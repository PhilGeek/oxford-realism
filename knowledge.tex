%!TEX root = /Users/markelikalderon/Documents/oxford-realism/oxford.tex
\section{Knowledge} % (fold)
\label{sec:knowledge}

The last section traced (somewhat speculatively) the most pregnant part of Austin's view of language, and of thought, to an idea of Cook Wilson's. But if Austin's view was so inspired, it was, plausibly, also inspired by need. Austin's view of language did not long survive him in Oxford itself. It was soon to be supplanted by what is commonly known as ``the Davidsonic boom''. But another idea, central in Cook Wilson, held a central place at Oxford until roughly the end of the century. It is an idea about knowledge which found applications to perception as well. As Austin saw things, that idea requires his view of language and thought in order to be viable. From some time in the '70s on, the dominant view in Oxford seems to have been that the idea about knowledge is perfectly fine with no help from that Austinian view. This section will set out the idea about knowledge and raise the question whether it is really true that no such Austinian help is needed.

The idea about knowledge can be stated simply. To know that \( P \) is no less than to have proof that \( P \) (or, perhaps, for \( P \) to be simply self-evident). A proof that \( P \) is something whose existence is \emph{absolutely} incompatible with things being otherwise than that \( P \). Having proof that \( P \) is, first, having (being entitled to) \emph{complete} certainty as to whether \( P \); and, second, appreciating adequately the proof one has available as the proof it is. So it is appreciating adequately the incompatibility of that which one sees as to how things are with it being otherwise than P. Cook Wilson expresses this idea as follows:
\begin{quote}
	In knowing, we can have nothing to do with the so-called `greater strength' of the evidence on which the opinion is grounded; simply because we know that this `greater strength' of evidence of \( A \)'s being \( B \) is compatible with \( A \)'s not being \( B \) after all. \ldots\ Belief is not knowledge and the man who knows does not believe at all what he knows; he knows it. \citep[100]{Cook-Wilson:1926sf}
\end{quote}
Prichard insists that knowledge is ``certainty'', and vice-versa. Certainty, for Prichard, is not a feeling. (As he insists, one might have any feeling, whether he knew or not.) Rather, it involves standing in a particular way towards the (mind-independent) world. He describes that way as follows:
\begin{quote}
	We should consider what has now become of the objection that our certainty that an \( A \) is \( B \) cannot be knowledge because an \( A \) need not in the real world conform to our certainty by being \( B \). The fact is that it has simply vanished. For now admittedly it is a condition of our being certain that an \( A \) is \( B \), that we know a certain fact in nature, \emph{viz}. that the possession by an \( A \) of a certain characteristic, a, necessitates its having the characteristic of being \( B \), and, knowing this, we cannot even raise the question `Need an \( A \) in nature have the characteristic \( B \)?', because we know that a certain definite characteristic which it has requires it to have that characteristic. \citep[103--104]{Prichard:1950tg}
\end{quote}
So to insist that knowledge is certainty in Prichard's sense is to endorse Cook Wilson's idea. To know that \( A \)s are \( B \) is to have proof: knowledge of some fact of nature which, one appreciates, is incompatible with things being otherwise. In Prichard's terms, one cannot even raise the question whether \( A \)s must be \( B \). At least one cannot intelligibly wonder this. There is no room for one to raise an intelligible doubt. So it is with sapience on Cook Wilson’s and Prichard's views.

The bite in this view begins to show in Cook Wilson's reference, above, to evidence. Can there be knowledge by, or on, evidence? One might think so. Has Sid been drinking? That loopy expression on his face is some evidence that he has. His slurred speech is a bit more. Now he comes close, and we smell his breath. Now we \emph{know} it. What has happened? One story might be: Sid's breath is (perhaps quite a bit) more evidence. Now the evidence has mounted so high, become so strong, that we may correctly take ourselves to \emph{know} that he has been drinking. So, in general, good enough evidence amounts to knowledge. Cook Wilson and Prichard reject this story. On their view, if all we have is evidence, even very strong evidence (but still, something evaluable in terms of strength or weakness), then, for all we have to show that Sid has been drinking, it is at least possible that he has not. It can make sense to ask whether he really has been.  So for all we know, perhaps not. But if one knows that Sid has been drinking, then not: for all he knows, perhaps not. So this is not knowledge. Which is not to say that one cannot come to know that Sid has been drinking by smelling his breath. But where one does this, one is aware of, as Prichard puts it, some fact of nature: Sid could have breath like that only if he had been drinking (his breath is that of one who has imbibed). In which case, that his breath smells thus does not stand to his having been drinking as evidence for this, but rather as proof.

Cook Wilson refers to knowing as a ``frame of mind''. Prichard concurs. One could use the term ``mental state'' here if one allows that whether one is in it depends, \emph{inter alia}, on how he stands towards the world. This last proviso points to something the two take great pains to stress: to see whether you know that P, do not try to examine your mental state (if that is some kind of pyschological, perhaps introspective, enterprise). Rather, turn your attention towards the things to be known---that Sid has been drinking, say, or that there is no largest prime---and see whether you have a proof of that in hand (whether you can see things being as they are to be incompatible with their being otherwise in that respect). If I know that Sid has been drinking, that is because, as I can appreciate, breath like that (at least in this case) can only mean that he has been drinking. To see by any other means whether I know that P would be, as Prichard points out (see 1950: 92-93), self-defeating (the start of an infinite regress). For if knowing were a mental state distinguished by some mark (which I might detect, say, by introspection), this would help me see whether I know that P only if I knew my current state to have, or to lack, that mark. But, on this plan for detecting knowledge, I could see myself to do that only if I could know my current state to have the distinctive mark of knowing that my state has the mark of knowing that P. And so on \emph{ad infinitum}. So if we were to call knowing a mental state, the way to see whether one was in it \emph{in re} P could only be by directing attention to its object, P. This idea, in more general form, has enjoyed a long life at Oxford. (See, e.g., Gareth Evans, ??)

Cook Wilson and Prichard also stress the further point that knowledge is not a particular variety of belief. In Prichard’s words:
\begin{quote}
	Knowing is not something which differs from being convinced by a difference of degree of something such as a feeling of confidence, as being more convinced differs from being less convinced \ldots\ Knowing and believing differ in kind as do desiring and feeling, or as do a red colour and a blue colour. \ldots\ To know is not to have a belief of a special kind, differing from beliefs of other kinds; and no improvement in a belief and no increase in the feeling of conviction which it implies will convert it into knowledge. \ldots\ It is not that there is a general kind of activity, for which the name would have to be thinking, which admits of two kinds, the better of which is knowing and the worse believing. (1932/1950: 87-88)
\end{quote}
Part of the point here is that knowledge is not \emph{analysable} in terms of belief (or, for both thinkers, in terms of anything). It is not as if knowing is believing with such-and-such further features added---thus, some special variety of believing, or of any other (non-factive) way of standing towards it being so (or its being so) that P. Such is now a widely held view, still at Oxford, and well beyond. But Cook Wilson also holds what is now generally seen as a stronger thesis: when you know that P, you do not believe it. (See above.) This is, to say the least, less widely held. It may \emph{seem} to be controverted by obvious facts---e.g., if Sid stands as he does towards Pia being the new dean, then it can be (depending on how he thus stands) that I, knowing that she is, may say, truly, ``Sid knows that Pia is dean'', while you, doubting that Pia could have been chosen, may say, also truly, ``Well, Sid \emph{thinks} that Pia is dean''. Each of us, it seems, states a truth about Sid's condition; truths which hold simultaneously, and, it seems, may hold of the same frame of mind, or mental state. Austin's view of language should make this seem a less convincing case against Cook Wilson's thesis. The thesis may then come to seem more plausible if we first recognise it as one version of disjunctivism---a denial of a certain sort of common factor in standing towards a thought that P as one might stand whether or not that thought is true, and standing towards a fact of its being so that P---and then apply J.M. Hinton's conception of what such a common factor---what would relevantly hold wherever the disjunction ``Sid believes, or he knows, that P'' would need to be. (See Hinton, 1967.) However, for reasons of space, we leave this here as mere suggestion.

There is a further feature of Cook Wilson’s and Prichard’s view. It is one they are at considerable pains to stress. Given their conception of a frame of mind, it seems to them simply to follow from the above conception of knowledge as proof---though \emph{perhaps} there is room to resist the inference. Cook Wilson sets up the inference by considering the possibility that there are two frames of mind---one knowing, the other merely being under the impression of knowing---which were such that if you were in the one, you might be unable to tell that you were in it rather than the other, so that, as he puts it:
\begin{quote}
	\ldots\ the two states of mind in which the man conducts his arguments, the correct and the erroneous one, are quite indistinguishable to the man himself. But if this is so, as the man does not know in the erroneous state of mind, neither can he know in the other state. (1926: 107)
\end{quote}
So a state of knowing,of actually having proof---if there is such a thing at all---cannot be indistinguishable to someone in it from an ``erroneous'' state---one of merely seeming to have proof; nor vice-versa. Prichard puts the conclusion here this way:
\begin{quote}
	We must recognize that whenever we know something we either do, or at least can, by reflecting, directly know that we are knowing it, and that whenever we believe something, we similarly either do or can directly know that we are believing it and not knowing it. (1950: 86)
\end{quote}
He insists on this point far more than just the once. For convenience, we will refer to this point as \emph{the accretion}.

There is no general thought here that if one is in a frame of mind, he can, by reflection, come to see that he is. Nor are Cook Wilson and Prichard endorsing some form of what has come to be known as ``semantic internalism''. The point is, or is meant to be, a quite special one about what knowledge, or what proof, is. The thought would go something like this. Suppose I am in a frame of mind in which I cannot, by reflection come to see (if I do not see already) whether this is one of having proof that P, or whether it is not. (One might plausibly think of this as my being unable to distinguish this from some other conceivable conditions I might be in in which I would not have proof---plausible, but optional for the present argument.) Then that frame of mind cannot be one of my actually having proof in the requisite sense of having proof. For, whatever grounds I may have for taking it that P, even if these are grounds which might, in fact, be incompatible with things being otherwise than P, they cannot be grounds which I appreciate as proving that P---as being incompatible with things being otherwise. For if I did so appreciate them, then I would see that my state could not be one of merely being under the impression that I had proof that P. So the imagined frame of mind is not one of knowing. Now contrapose. The core thought: I am either in the frame of mind, or I am not---I either have proof or I do not; having proof is the sort of thing such that if you do it, then you should be able to see yourself to do so.

At this point, the whole conception of knowledge as proof begins to appear on shaky ground. On this conception, for one thing, could I ever know such a thing as that a pig is in the sty? Conceivably, Dr. Zarco (call him) might build a ringer-pig which one, or I, could not tell, at least by sight, from the real thing. As I now stare at the pig in the pen, can I tell by \emph{mere reflection} that I am not in such a situation? What, in fact, from my present vantage point on the world, allows me to \emph{tell} that I am not in such a situation, but rather in one in which the thing before me is \emph{really} a \emph{genuine} pig? So knowledge by perception---knowing because, e.g., you see it---seems ruled out absolutely. Which is unlikely to leave a viable ``intellectualist'' conception of knowledge, on which mathematics is the paradigm (and more or less exhausts the field). One can be fooled by a bogus proof. I now take myself to have a genuine proof, say, of some proposition of number theory. Perhaps it is genuine. But can I tell whether it is by mere reflection? If there were a flaw in the proof, could I detect that by mere reflection? (And just what counterfactual is this?) If one conceives reflection as Cook Wilson and Prichard appear to, then, perhaps, the answer is ``Yes''. It is within the reach of human reason to detect such flaws. It is, perhaps, within my reach if one neglects limitations of memory, attention, patience, and enough other things which might block my seeing the flaw. But if one so conceives reflection, then, plausibly, mathematical knowledge, at least in a broad enough domain to take in my theorem, just is knowledge by reflection. There remains for all that what I do know and what I do not about number theory. Being able to detect flaws on an overly idealized conception of this will not draw the distinction. What, then, might?

It is time for Austin's entrance. Austin takes over several points from Cook Wilson. (But, we shall see, \emph{not} the accretion.) There is, first, the idea that there is no knowledge by evidence. This comes out in Austin as follows:
\begin{quote}
	The situation in which I would properly be said to have evidence for the statement that some animal is a pig is that, for example, in which the beast itself is not actually on view, but I can see plenty of pig-like marks on the ground outside its retreat. If I find a few buckets of pig-food, that’s a bit more evidence, and the noises and the smell may provide better evidence still. But if the animal then emerges and stands there plainly in view, there is no longer an question of collecting evidence; its coming into view doesn’t provide me with more \emph{evidence} that it’s a pig, I can now just \emph{see} that it is, the question is settled. (1962: 115)
\end{quote}
Evidence is distinguished from proof. Even very good evidence, like the noises and the smell (in the situation Austin envisions) is compatible with it not being so (in Austin’s case) that the animal is a pig. By contrast, there is another sort of thing which (to speak archly) might speak in favor of taking the animal to be a pig; another way in which the world might come to bear for you on the question whether there is a pig before you. It is illustrated, in Austin’s case, by the pig standing there in plain view. In this case, when I see the pig, I can, thereby, see there to be a pig before me; with which ``the question is settled'': the pig’s presence (which, in this case, I can see to be the presence of a pig) is absolutely incompatible with \emph{that} animal failing to be a pig. There is as little room for that as there is for a largest prime, given the proof that there is none. Where, as in this case, I can see a pig to be present, I appreciate what I thus have in hand as proof. So the situation is this: vision affords me awareness of something, the obtaining of which proves that I confront a pig (leaves no doubt as to this); my recognising what I am thus aware of as a pig being before me (my seeing it to be a pig before me) is my appreciating what I have (am thus aware of) as proof. The proof here is short: the pig’s presence proves that a pig is present. (Note the step here from what does not have the form of a proposition to what does.) Anyway, such is a model of knowledge gained through perception. It is one on which such knowledge is not based on evidence (as, on the present conception, no knowledge could be).

Pursuant to this point, Austin echoes Prichard in insisting that knowledge is distinct from belief, or being (even justifiably) very sure. He writes,
\begin{quote}
	Saying `I know \ldots\ is \emph{not} saying, `I have performed a specially striking feat of cognition, superior, in the same scale as believing and being sure, even to being merely quite sure': for there is nothing in that scale superior to being quite sure. (1946/1970: 99)
\end{quote}
What, then, \emph{is} the difference between knowing and believing or being sure? For Cook Wilson and Prichard, these are different ``frames of mind''; where one can tell which he is in by ``reflection''. Austin puts things in somewhat different terms:
\begin{quotation}
	\noindent When I say `I promise', a new plunge is taken: I have not merely announced my intention, but, by using this formula \ldots\ I have bound myself to others \ldots\ Similarly, saying `I know' is taking a new plunge. \ldots\ When I say `I know', I \emph{give others my word}; I \emph{give others my authority for saying} that `S is P'.
	When I have said only that I am sure \ldots\ I am not liable to be rounded on in the same way as when I have said `I know'. I am sure \emph{for my part}, you can take it or leave it \ldots\ that’s your responsibility. But I don’t know `for my part', and when I say `I know' I don't mean you can take it or leave it (though of course you \emph{can} take it or leave it). (1946: 99-100)
\end{quotation}
This particular point in Austin has attracted a large amount of criticism. There seem to be two main complaints. First, the verb `know' seems to have other uses in the first person than that Austin has in mind---e.g., ``It's hard to park near the beach in August''; ``I know, I know''. Second, waiving that point, even if ``I know'' does typically mark a special force attaching to words, ``I know that P'', still, to describe that force, even correctly and in detail, is not yet to tell us what knowledge is---what it is for someone to know something, for a given such thing, under just what conditions of the world it would be true that he did.

For all that, though, Austin may have made a good start on saying what knowledge is, insofar as there is such a thing as saying that. Suppose that there is the use of ``I know'' that Austin has in mind, and someone, Sid, makes that use of it on an occasion. He may have done so correctly or incorrectly. As usual in such matters, one needs to choose his notion of correctness. It might be that one in Sid's position, grasping what he would say speaking as Sid did, might be able so to speak with complete sincerity and honesty. He might thus be perfectly justified in so speaking. That is one notion of correctness. But there may also be a certain position which one must be in in order to use the words (``I know'') for what they are to be used for (on this use); and Sid uses the words while in this position. Then that is another notion of correctness. (Compare: I may say, ``Pigs grunt'', being perfectly justified in taking it that pigs grunt, thus correctly on one notion of correctness, and, further, I may so speak while, in fact, pigs grunt, and thus be correct on that other notion.) What Austin suggests is that to say ``I know that P'', on the use he has in mind, is to claim authority as to whether P; to offer oneself as authoritative on that point. The position one must be in to do this correctly, on our second notion of correctness, is to be, in fact, authoritative as to whether P. Suppose this is right. Now suppose I tell you, ``Sid knows that Pia is at the Dew Drop Inn''. From the account so far, we can extrapolate something I am thus committed to: Sid is in a position to offer himself as an authority on that point (should he care to). We now have in hand what begin to look like materials for a general account of what one says in saying N to know that P (with a bit of feeling for the different uses ``I know'', ``You know'', etc., in fact have)---perhaps not the most elaborate account one might wish for, but anyway an account of the right shape.

What is not yet in view is any particular point in putting things in these terms. For that we need to see Austin's most significant contribution to making Cook Wilson's conception viable. It is most neatly captured here:
\begin{quote}
	It seems to be fairly generally realised nowadays that if you just take a bunch of sentences \ldots\ impeccably formulated in some language or other, there can be no question of sorting them out into those that are true and those that are false; for \ldots\ the question of truth and falsehood does not turn only on what a sentence \emph{is}, nor yet on what it \emph{means}, but on, speaking very broadly, the circumstances in which it is uttered. Sentences are not as such either true or false. But it is really equally clear \ldots\ that for much the same reasons there could be no question of picking out from one’s bunch of sentences those that are evidence for others, those that are `testable', or those that are `incorrigible'. (1962: 110-111)
\end{quote}
So whether A is evidence for B (or it is true to say so), as opposed to being \emph{no} evidence, or as opposed to being proof, depends not just on what A and B are, but on the circumstances of, or for, so saying (or so counting things). So, correspondingly, whether N has proof, or merely has evidence, that P thus depends on circumstance. So, accepting the Cook-Wilsonian conception of knowledge as proof, whether N \emph{knows} that P (or it is true to say so) depends equally on circumstance. The model here should come from Austin's view of language, as per the last section. Is the sky blue? There are various things to be said in speaking of it, and saying, it to be blue, some true, some false. What one would say depends on the circumstances in which he so spoke. So, apart from an occasion for so speaking, the question has no answer, is ill-formed. Where there  is an answer, what it is depends on the occasion for giving it. Now the idea is: the same goes for knowledge. Suppose, with Austin, that we think of knowing that P as a matter of being authoritative on the subject, or, even more Austinianly, as being in a position to offer oneself as an authority. So to say that N knows that P is, at least in the central use, to say that N is in such a position. Now here is the idea, applied to Pia as she watches the (free range) pig emerge from its straw shelter and approach the barbed wire between them. Such are \emph{her} circumstances. Now, does she know that a pig approaches? There are many (possible) occasions on which one might say her to, or not to. For each of these, it would be true to say, on it, that she knows this if (but only if) it is correct to acknowledge her position as authoritative on that subject. For each of these, there is what it then would take to be thus authoritative. For some of these, Pia has what it would then take, so it would be true to say that she knows a pig approaches. For others it would not, so it would not be true to say this. Independent of these truths, and falsehoods, to be stated on occasions, there is no well-formed question as to whether she knows or not, no fact that, occasion-independently (on this use of ``occasion'') she ``really'' knows, or ``really'' does not. Such is knowledge on the view Austin proposes.

Let us apply (sketchily) the central idea here to questions of evidence. We can begin with Sid’s breath. Is this proof that he has been drinking, or merely (some more) evidence? The question, asked just like that, seems embarrassing. There is, after all, some gap in conceptual space between having breath like that and having been drinking. So, when the question is asked like that, there seems \emph{some} possibility that, for all of his breath being as it is, he has not been drinking; which suggests that his breath cannot be \emph{proof}. What might some of the ways be for the inference here to fail? Perhaps you can get breath like that from near-beer, or by kissing a drunk (or enough drunks), or from tasting and spitting, or from strong whisky-flavored gum (a good wheeze, or good for undercover). Sid’s breath is as it is. Those are his circumstances. But there are many occasions for taking it (or not) as evidence (or more) of drinking. Suppose that Sid is a well-known wine taster, and it is reasonable to suppose that he has been practising his profession. Then, perhaps, his breath is no evidence at all that he has been drinking (in the meaning of the act). Or, again, suppose it is unlikely, but not entirely ruled out, that Sid has been chewing that special gum. Then his breath may be evidence, but hardly proof. But suppose there is simply no question of Sid having come by his breath in any such unusual way. Then to smell his breath is to know what he has been up to; his breath is proof. What varies here is our (or one's) circumstances on an occasion for making something of Sid's breath; for taking it as proof, or mere evidence, or as not even that. What varies with that is whether, in those circumstances, it is true to say that Sid's breath is evidence (or etc.), whether it so counts.

Consider now Pia, across the fence from the approaching pig. Does she know that a pig approaches? Hers is not much of an occasion for her to say, either that she does know, or that she does not. That is, she is not, most likely, in circumstances which determine any answer to that question (as, in general, circumstances are always liable to do wherever Austin’s core point applies). She has, as one says, the evidence of her eyes, for whatever that is worth. But does this amount to knowledge? If what she sees is the pig approaching, \emph{if} she can recognise this---if, in those circumstances, she can recognise a pig by sight---so that she can see that a pig approaches, then, trivially, what she sees, in seeing what she does, is proof, and not mere evidence, for her that a pig approaches. So she knows this. If not, not. At best, the ``evidence of her eyes'' is merely evidence. \emph{Are} these conditions satisfied? Whether they \emph{count} as such depends on the occasion for the counting, as per Austin’s core idea. There might, e.g., on some such occasion, be reason to doubt whether Pia can really tell \emph{pigs} from certain other animals, or whether the beast in question might be some sort of monster, porcine on its visible side only, and so on. (A comparison. Suppose the question were whether Pia knows that the approaching pig is a \emph{bísaro}---a particular kind of pig, marked by long rear legs, and large, floppy ears. Can Pia really tell \emph{bísaros} by sight? Did she really get a good view of the hind legs? Or did she only see the front part of the pig? Etc.) In such cases, it might be true to say Pia \emph{not} to know that a \emph{pig} approaches. But Pia has seen pigs before, and on many occasions counts as being able to tell a pig by sight. On some of these occasions, she may be said, truly, to know that a pig approaches. On such occasions, the evidence of her eyes is, for her, not merely evidence; it is proof. Such is a sketchy illustration of Austin’s idea applied to knowledge by perception.

How does Austin's idea apply to the accretion? The idea which moves the accretion is very briefly this. Suppose I cannot tell, on reflection, that I have proof that P. Then, for all I know, I do not have proof. But then I do not know. The response to that idea now takes this form: the question which, on it, I am supposed to answer on reflection, if I know that P is ill-formed; not a question with an answer at all. Again consider Pia and the approaching pig. She stands as she does towards the pig. Pia stands as she does towards the pig. On some occasions for considering her position, this would count as her having proof that a pig approaches. On others it would not. (Compare, again, Pia and Sid's breath.) There is no further fact as to Pia ``really'' having, or ``really'' lacking proof. So what should Pia be able to tell on reflection? Presumably not that, on the occasion of Sid and Zoë discussing her situation, it would be true for \emph{them} to say that she had proof. Why pick that situation, or impose on her the burden of seeing how \emph{their} circumstances would matter to whether she then counted as having proof? Nor, presumably, whether it would be true for her to say, at the moment of her gazing, that she had proof. First, there is likely to be no such thing to be said either truly or falsely at that moment. Second, why pick on that moment, when knowledge is, grammatically, a state---something which persists whether or not you are \emph{talking} about the matter in question. Third, whether or not she were able to say, truly, on her occasion, that she had proof, this would not settle those questions to be settled on other occasions in then so asking. But there is no further question besides such special questions as these. So there is no question as to having proof such that whether one knows that P turns on whether, on reflection, he can answer it. It is not as if, on this account, one may know that P while being ignorant as to whether he has proof. It is that no sense is to be made of what it is that on reflection one is supposed to see (on that idea which motivates the accretion).

Where Pia, as she stares across the fence, counts as knowing that a pig approaches, she counts as appreciating adequately as proof the proof at her disposal. But this may just come to her counting as seeing the pig, and as able, in this situation, to tell a pig, or this one as a pig, at sight. Perhaps there is also some requirement as to her actual convictions \emph{in re} it being a \emph{pig}, which is \emph{approaching}, though, as evidenced in the literature, there is room for dispute as to just what this requirement might be. Here we bracket that discussion.

So Austin's core idea leaves us with Cook Wilson's conception minus the accretion. It also leaves us, in the domain of knowledge, with a very significant form of disjunctivism. To find it, we can begin from an argument not unlike that for the accretion. This argument takes a case where all is well---say, where that pig is approaching Pia, and pairs it with a ringer for it (in fact, some one of indefinitely many different ringers). What makes for a ringer is this: if Pia were in the ringer situation, she would not be able to tell that she was in it rather than in her actual situation (or, more exactly, one in which a pig was approaching). Everything would be, so far as she could tell, without changing that situation, just as it in fact is with the pig approaching. But no pig would be approaching. Though we skip further details, ringers are always conceivable. In the ringer situation Pia would not have proof that a pig was approaching, since none would be. At best (according to the argument) she would have whatever reasons--\emph{nota bene} inconclusive ones---for supposing there to be a pig approaching. Such-and-such (according to the argument) would be her evidence for that, but no more than evidence. Now we shift to the non-ringer (the actual) situation. Here, according to the argument, she would have just the evidence she had in the ringer situation. But if, in this case, she knew that a pig approached, she would have to have something more as well; something which ruled out her being in the ringer situation---that is, which allowed her to distinguish her actual situation from the ringer. By hypothesis, though, there is no such thing. If there were, then the ringer would not be a ringer. So in the actual case she does not know that a pig approaches. So, ringers always being conceivable, knowing such things about the world around one is impossible.

If Austin's central point is correct, then there is more than a little wrong with this argument. First, one cannot suppose that there is such a thing as ``the evidence Pia has'' in the ringer situation. If she \emph{were} in some situation which was a ringer for a pig approaching her, then there would be indefinitely many occasions for discussing her status there. On different of these, there would be different things to be said truly as to what her evidence was. What counted as her evidence on one such occasion for discussing her predicament might not do so on some other. Similarly, in the real situation there is no such thing as ``her evidence'' \emph{tout court}, but only what, on some particular occasion, might be said truly as to what evidence she had. Which leads to a more important second point. In the ringer situation (of course) she could never count as having any more than evidence on any occasion for discussing her plight, since there cannot be proof of what is not the case. But it does not follow that she can never have any more than evidence in the actual situation, nor that what she can have in the actual situation is restricted to what she would have in the ringer situation plus some addition. In the ringer situation, perhaps (depending on how the situation is set up), something uncannily porcine-looking approaches. That a porcine-looking thing approaches can sometimes be evidence, and normally no more, that a pig approaches. In the ringer situation it gives Pia some reason to suppose that a pig approaches, but, of course, no more. In the actual situation a pig does approach. If Pia can see it approaching, and if she can recognise what she thus sees as that (that is, as a pig approaching), then may have as her reason for supposing that a pig approaches (if one chooses to put things so archly) that she \emph{sees} this. And, on some occasions for considering her actual plight, this would count as her reason. On such occasions, it is not as though she still counts as having, as evidence for so supposing, that something porcine-looking approaches. There is, on such an occasion, no room for this to figure in her reasons at all. Nor is there any clear way of evaluating it as evidence, in the terms of evaluation to which evidence is subject. Exactly how strong or weak is it, in the circumstances? Given that she sees the pig, and can see herself to do so, how can it matter that, moreover, the pig actually looks like a pig? Such is part of the point of insisting, with Austin, that what sometimes may be evidence is, other times, not so much as any evidence at all.

Summing up, then, in the actual situation, and on a favourable (but possible) occasion on which to consider Pia’s plight in that situation, Pia’s reasons for supposing that a pig approaches do not consist in the evidence she would have in the ringer case plus something else to rule out her being in that ringer case. Her reasons do not include such ringer-case evidence at all; and, they being what they are, no reason in addition to them is needed to rule out her actual case being the ringer. At which point, the argument as set out collapses. There are, then, two kinds of case: a case in which knowledge is in reach, and, on some occasion for considering it, counts as possessed; and a ringer case in which knowledge is not in reach. The reasons one has for supposing, falsely, in the ringer case something which is so in the first case are not a factor in common to both sorts of case. The first sort of case is thus not a ringer-case plus some addition. Such is one form of the disjunctivism for which Oxford later came to be well-known.

At Oxford, as noted, Austin's view of language and of thought did not long outlive his death. But Cook Wilson's conception of knowledge (most often with the accretion suppressed) continued to have its champions, most notably John McDowell. (See his two landmark essays, (McDowell: 1982, 1995).) McDowell’s main concern in those essays was to resist a picture in which knowledge is a sort of construct out of belief (or some other non-factive condition) plus some additional factors which might obtain or not without the knower’s awareness of this (a view which he refers to as ``the hybrid conception''). His position is thus far very much Cook Wilson's. Further, he resists the argument just canvassed by denying its conception of a common factor between cases of knowing and ringers for them, in line with the conclusion just suggested (though not quite for the same reasons). McDowell, though, does not accept that view of thought and its expression which Austin (mistakenly) characterised as ``fairly generally realised nowadays''---the central point in the above story. It remains a good question how Cook Wilson’s view can be viable without this. McDowell writes,
\begin{quote}
	Whether we like it or not, we have to rely on favours from the world \ldots\ that on occasion it actually is the way it appears to be. But that the world does someone the necessary favour, on a given occasion, of being the way it appears to be is not extra to the person’s standing in the space of reasons. \ldots\ once she has achieved such a standing, she needs no extra help from the world to count as knowing. (1995: 406)
\end{quote}
So if a pig actually is approaching Pia, then the needed favour has been done. For McDowell, no further favours are needed for her to count as \emph{knowing}. She must, of course, but need only, be able to appreciate her situation for, in this respect, what it is. Where a pig approaches, there is something \emph{for} her to appreciate as to what her situation is, which is not there to be appreciated in any situation in which no pig is on the way. So she may stand towards her situation in a way in which she could not in a deceptive case. She may, if all is well, have as her reason for taking a pig to approach that she sees one to. Such is the core of a disjunctivism on the model of Austin's (but without that central idea which, to Austin’s eye, makes the disjunctivism viable. The question now for McDowell is how such disjunctivism can be viable.

Suppose a pig is approaching Pia. So the world has done its favour. Still, on the conception of knowledge which McDowell endorses---one on which ``the unconnected obtaining of [that] fact'' cannot ``have any intelligible bearing on an epistemic postion'' (vide 1995: 403)---Pia must relate to that fact in the right way---one which draws on her cognitive capacities---if she is to know it. So there remain two sorts of case. Pia may, or may not, have proof at her disposal; and, if so, may, or may not, be able to appreciate what is at her disposal as the proof it is. In the case at hand, she may or may not be able to see the pig approaching; and, if she does, then she may, or may not, be able to recognize what she thus sees as what it thus is. Without Austin's idea in place, it is fair to ask: In what cases of viewing does Pia see the pig, and in what does she not---e.g., in what does she merely see a porcine front half of an animal which is approaching? Thompson Clarke (1965) offers principled reasons for finding that question more than just difficult to answer. Then, if she does see the pig, there is the question whether she is able to recognize what she sees as that. Consider all the ways in which she might be viewing a scene (not necessarily this one) where no pig approaches---\emph{inter alia}, all the ways for the world to have failed to do its present favour. For example, there are those situations in which a shaved goat, or in which a tapir, would be approaching. Without Austin's means, we must say: for some of these, if Pia could not distinguish her actual situation from them---perhaps, e.g., if she could not tell pigs from tapirs---then she would not count as able to tell, in her situation, that a pig approaches. For other---perhaps, say, for Dr. Zarco’s miracle mechanical pig---no such conditional holds. What is needed now is some kind of principled, or at least recognizably correct, way of drawing the distinction---of saying when Pia would have done her bit to earn the relevant status in ``the space of reasons''. Austin offers a principled way to reject the questions, and to be unsurprised when the pursuit of an answer leads only to bafflement. McDowell has no such means. Thus, though we may commend him for the conception of knowledge which he offers, and for his demonstrating the unviability of the alternatives, the question remains how he can make Cook Wilson’s conception viable.

% section knowledge (end)
