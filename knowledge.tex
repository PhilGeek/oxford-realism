\section{Knowledge} % (fold)
\label{sec:knowledge}

% section knowledge (end)

The last section identified a distinctive core in J. L. Austin’s view of language and thought, and traced this view, intellectually, at least, to some at first sight peculiar ideas in Cook Wilson. Though Austin’s ideas on language were the most distinctive, and even, in some sense, the most distinctively Oxonian, in 20th century Oxford, it cannot be said that they survived there long, at least as a major presence. They were soon to be supplanted there by what has come to be known as the ‘Davidsonic boom’. No doubt Austin’s interest in language was intrinsic. But, as already suggested, it was motivated by commitment to another distinctively Oxonian view---perhaps the most distinctively Oxonian of all, and this time one which did flourish throughout the century. This is, in the first instance, a view of knowledge---though it has become almost a tradition (as evidenced, e.g., in very different forms, in Prichard, Austin and, later, John McDowell) to apply it to perception as well. There is something very compelling about the idea from which this idea begins. Yet it can immediately seem (as good as) impossible to reconcile it with (what seem) undeniable facts. What Austin was the first to see (and saw clearly) is that there is only one satisfactory resolution to this dilemma. It requires invoking the view of language presented in the last section. (As recent work has shown, it also requires some delicacy in the application.)

The core idea can be brought into proper focus by asking whether there is such a thing as knowledge based on evidence. The answer can seem clearly to be, ‘Yes’. That loopy expression on Sid’s face is some evidence that he has been drinking. His slurred speech is a bit more. His errant gait yet a bit more. Then he comes close; we smell his breath. Finally the evidence has mounted to the point where we can say we know he has imbibed. Or again, zinc stops colds. It worked for Sid and Pia. That is a bit of evidence. We broaden our study, perhaps introducing a control group. The evidence holds up. We broaden some more. The evidence continues to mount. If it does, then at a certain point we can say: we know zinc stops colds. But is this right? Suppose that all I have is evidence that Sid is drunk. Evidence may make this very probable. The idea here seems to be: if it is probable enough (say, very, very probable) then we know. If this is the idea, then there is the following to say: for it to be merely very, very probable that Sid is drunk is for there to be some chance that he is not. So if we reflect on our position in having very strong evidence that he is drunk, we come to see that, for all the evidence at our disposal, Sid might not be drunk. But to admit that Sid might not be drunk is to admit that we do not know he is. I cannot know he is if, for all I know, it is possible for him not to be. This, in one form, is the core idea. That idea can also be put this way: merely to have evidence, no matter how high mounted up, is not to have proof. But to know something is to have proof. So there is no such thing as knowledge by evidence. And, admittedly, that claim can seem absolutely incredible (and/or simply the assertion of scepticism).

Still, though, it is a compelling thought that to know is to have proof, to see how things stand in the relevant respect; and that one who merely has evidence does not see how things stand---or, conversely, if he does see, then he does not need evidence (and even, perhaps, that, in that position, nothing could be evidence for him). One way to know that Sid is drunk, e.g., is to witness his drunkenness (and to recognise what it is that one thus witnesses). One cannot do that just be having evidence. For, whatever the evidence is, it is something distinct from the fact of Sid’s drunkenness, so that awareness of it cannot be awareness of (witnessing) that drunkenness. And, conversely, if I am in that happy position just described, then evidence for his drunkenness, whatever it might be, is otiose. It cannot provide my reason for thinking he is drunk. True, I may know that Sid is drunk without witnessing his drunkenness. But this is not to say that I can know that Sid is drunk while enjoying any lesser epistemic status in re the question whether he is drunk than I enjoy where I am witness—that is, where I do any less than being acquainted with the fact of his drunkenness, so as to have that as the grounds on which I know. Such, expanded, is the idea that to know is to have proof.

Cook Wilson puts the point thus far as follows:
\begin{quote}
	In knowing, we can have nothing to do with the so-called `greater strength' of the evidence on which the opinion is grounded; simply because we know that this `greater strength' of evidence of \( A \)’s being \( B \) is compatible with \( A \)’s not being \( B \) after all. \ldots\ Belief is not knowledge and the man who knows does not believe at all what he knows; he knows it. \cite[100]{Cook-Wilson:1926sf}
\end{quote}

Indeed, he holds that knowledge is not analysable in any terms at all. It is not a form of belief, or of anything else, distinguished from other such forms by the present of such-and-such feature(s). Nor could one test to see whether the condition he was in was one of knowing that P by seeing whether it was a mental state with some feature, \( F \). (Indeed, though he speaks of knowledge as a frame of mind, or, sometimes, as a mental condition, Cook Wilson begins to point towards a sense in which knowledge is not a mental state at all.) To see whether you know that \( P \), direct your attention, not at yourself, but rather at the question whether \( P \), and attend to the answer to this that you have in hand. If you know that \( P \), then that which convinces you is, recognisably, absolutely incompatible with P not being so (in the way that a peccary’s presence on your path is absolutely incompatible with it failing to be so that there is a peccary there—though this is not a very Cook-Wilsonian case).

Somewhat later, Cook Wilson’s student, H. A. Prichard, expressed the view thus far as follows:
\begin{quote}
	Knowing is not something which differs from being convinced by a difference of degree of something such as a feeling of confidence, as being more convinced differs from being less convinced \ldots\ Knowing and believing differ in kind as do desiring and feeling, or as do a red colour and a blue colour. \ldots\ To know is not to have a belief of a special kind, differing from beliefs of other kinds; and no improvement in a belief and no increase in the feeling of conviction which it implies will convert it into knowledge. \ldots\ It is not that there is a general kind of activity, for which the name would have to be thinking, which admits of two kinds, the better of which is knowing and the worse believing. (1932/1950: 87-88)
\end{quote}

Here Prichard also stresses the idea that knowledge, or more precisely, its object, is neither true nor false. To think, say, that that peccary is bristly is to take a stand on (or posture towards) a certain question; to relate to a thought---a particular way of being right or wrong as to how things are. Whereas to know that that peccary is bristly is, intrinsically and irreducibly, to relate to the fact of that peccary’s being bristly---if there is no such fact, then \emph{ipso facto} there is no knowledge.

Both Cook Wilson and Prichard adjoin to the above a further view. Cook Wilson points towards this addition here:

A correct way to put the case before us seems to be that the two processes, the two states of mind in which the man conducts his arguments, the correct and the erroneous one, are quite indistinguishable to the man himself. But if this is so, as the man does not know in the erroneous state of mind, neither can he know in the other state. (1926: 107)

Prichard works this worry into the following idea:
\begin{quote}
	We must recognize that whenever we know something we either do, or at least can, by reflecting, directly know that we are knowing it, and that whenever we believe something, we similarly either do or can directly know that we are believing it and not knowing it. (1950: 86)
\end{quote}
I need only reflect, the idea is, on what makes me feel convinced that \( P \)---what convinces \emph{me}---and I can see immediately either that the way I stand towards \( P \) is knowing it, or, again, that it is not. So that, on reflection, one cannot mistake one’s own belief for knowledge, nor vice-versa.

We turn now to Austin. Austin sees something importantly right in Cook Wilson’s and Prichard’s view of knowledge. But it faces an obvious problem. It is Austin’s response to that problem which is the centre of interest here. For a start, Prichard stresses, above, that the difference between knowing something and merely being convinced of it is not a difference of degree. Austin concurs. As he puts it:
\begin{quote}
	Saying `I know' \ldots is not saying, `I have performed a specially striking feat of cognition, superior, in the same scale as believing and being sure, even to being merely quite sure': for there is nothing in that scale superior to being quite sure. (1946/1970: 99)
\end{quote}
Austin’s treatment of this point represents a crucial modification in the Cook-Wilsonian conception of knowledge. We will return to this presently. But first to introduce the problem.

Just how does knowledge differ from a maximal point on a scale of being sure? The obvious answer is: being sure, even \emph{very} sure, of something is compatible with it not being so. Knowing is not. As Cook Wilson and Prichard conceive that incompatibility, where one knows something, that \( P \), he is aware of what is (and what he could recognise to be) absolutely incompatible with it not being so that \( P \), which rules this out absolutely; he is thus in a situation which is absolutely incompatible with P not being so. In short, to know is (nothing less than) to have proof; which differs from having even very strong evidence in the way Cook Wilson points to. On this view, where one \emph{knows} that \( P \), there is a kind of relation between what answers the question \emph{how} you know this---what it is that reveals to you that \( P \) is so---and what it is that you know, that \( P \) is so. It is the kind of relation which holds between what is cited in a proof that there is no largest prime and there being no largest prime: what is cited in the proof rules out \emph{absolutely} there being such a prime. Suppose you see a peccary before you. That peccary’s presence before you, and, also, the fact of your seeing it before you, are as incompatible with it not being so that there is a peccary before you as the proof that there is no largest prime is incompatible with there being a largest prime. There is no room in conceptual space for the one thing---the peccary’s presence, e.g.,---without the other---its being so that a peccary is present. The peccary’s presence, and your seeing it, do not have the form of a fact that such-and-such (though that you see the peccary does). But the incompatibility here is one we are as well equipped to recognise as that between the facts cited in a mathematical proof and the failure of its conclusion to hold.

So if I am (say, visually) aware of a peccary before me, and can recognise what it is I am thus aware of, then, as we can all recognise, I am aware of what is absolutely incompatible with it not being so that there is a peccary before me. So, if I am in that condition, then, it would seem, on Cook Wilson’s apparently demanding conception of knowledge, I know, or can know, that there is a peccary before me. Such would be a way in which vision can afford me knowledge, even on the most demanding conception of what knowledge would be. But now we may ask whether vision really ever does place me in such positions. Take a situation in which I do see a peccary before me, in plain view. I can recognise, sure enough, that a peccary before me is incompatible with it failing to be so that there is a peccary before me. But can I also recognise that I am aware of what is incompatible with that? And if not, can I meet Cook Wilson’s demanding standard for knowing? After all, a ringer for my situation is conceivable: a situation in which, for all I can tell, I see a peccary before me, but in fact I do not (e.g., because it is a ringer-peccary). And now, despite its being a peccary before me that I am in fact aware of, and despite my preparedness to take what I am aware of for that, am I really in a position to swear, hand on heart, that there is no chance that I am in a ringer situation, that, given all I can see as to what my situation is, it is just inconceivable that my situation is such? Suppose one answers, ``Yes'': there are situations which you can just see not to be ringers, but to be genuine situations of seeing a peccary---and, of course, since there are ringers, situations in which you cannot see this. Then the question is when you are in the one sort of situation, when you are in the other---for, it seems, this is something you must be able to tell. To say the least, there are no ready answers to such questions.

It is, it seems, for just such reasons that Prichard in fact makes it doubtful that perception can be the sort of source of knowledge here envisaged. He says, e.g.:
\begin{quote}
	When knowing, for example, that the noise we are hearing is loud, we do or can know that we are knowing this and so cannot be mistaken, and when believing that the noise is due to a car we know or can know that we are believing and not knowing this. (1950: 89)
\end{quote}
And he suggests as a way for me to know that I do not know this that, on reflection, I see that ``such a noise can be caused by something other than a car, say, an aeroplane.'' (1950: 79) So if we think as above, it can seem at best questionable whether Cook Wilson’s conception of knowledge can make room for vision as a source of knowledge of such things as the whereabouts of peccaries. Which may seem to force on us a highly revisionary view of perception. But, as Cook Wilson himself saw, the trouble does not end here. (Thus, swallowing such a view of perception will not buy us Cook Wilson’s view of knowledge.) For, if one can be under the illusion of seeing a peccary on the path—if there are such ringer situations—one can also be under cognitive illusions. I may be as convinced as can be that I see before me a proof of the Pythagorean theorem when what I in fact see is a bogus proof. I would be willing to swear that what I see is incompatible with the Pythagorean theorem failing to hold. But I would be wrong. Same problems all over again; this time ones which Cook Wilson worried about quite a lot. Perhaps, as he insists, I could (in principle), if I reflected, come to see that what is before me is not a proof. But if I am as certain as can be that I have before me a proof that the continuum hypothesis is independent of the axiom of choice, it is cold comfort to be told that if this is cognitive illusion, it might, in fact, go away if I reflected enough. Do I now know the fact in question, or do I not? And just which cases of my so standing would be ones of my knowing this, which cases not?

A satisfactory account of knowledge should allow that, sometimes, one knows there is a peccary before him because he sees it. This might have been enough motivation for the reworking of Cook Wilson’s view which Austin proposes. But, for reasons Cook Wilson himself acknowledges, that view is in enough trouble anyway. Now, then, for the modifications. Two are crucial. A third follows of its own accord. One of the two crucial points concerns the right way to understand the idea that knowing is not some very high point on a scale of being sure. Austin puts the point in first person terms, making a comparison between the sort of force (typically) indicated by prefacing a statement with ‘I know’ and that indicated by ‘I promise’, used to make a promise. Comparing promising to stating an intention, he says:
\begin{quotation}
	\noindent When I say `I promise', a new plunge is taken: I have not merely announced my intention, but, by using this formula \ldots\ I have bound myself to others \ldots\ Similarly, saying `I know' is taking a new plunge. \ldots\ When I say `I know', I give others my word; I give others my authority for saying that `\( S \) is \( P \)'.
	\\
	When I have said only that I am sure \ldots\ I am not liable to be rounded on in the same way as when I have said `I know'. I am sure for my part, you can take it or leave it \ldots\ that’s your responsibility. But I don’t know `for my part', and when I say `I know' I don’t mean you can take it or leave it (though of course you can take it or leave it). (1946: 99-100)
\end{quotation}
This move of Austin’s has probably attracted more, and more vehement, criticism than any other he made. The complaint, briefly, is that saying what one does in saying `I know' (at that, only on one special sort of occurrence of it) is not offering an account of what knowledge is. Perhaps not. But, considered as a modification of Cook Wilson, it is an interesting move in that direction. The shift it highlights is this. Cook Wilson refers to knowledge as a `frame of mind'. The point about scales, in that context, is then that knowledge is a characteristically different frame of mind from any degree of being sure, or, in Cook Wilson’s terms, from believing, or being of an opinion. The question then would be just what state of mind it is. Presumably, in any case, for any potential object of knowledge (any fact), and any time, presumably a frame of mind which I am in at that time relative to that fact, or not, punkt. Either I am, or I am not, enjoying that sort of awareness which, on the Cook Wilson view, I am supposed to enjoy if I know. Whereas Austin here points us towards viewing knowing as a sort of status which I may be accorded, or enjoy. In saying, `I know that \( P \)', I pretend to a certain sort of authority—to being an authority on, having proof of---\( P \). The question then is whether I am really to be counted as having such authority. This is a question as to what I have done to earn that status; and what would need to be done to earn it. It naturally points in different directions than a question understood as one as to what frame of mind I am in.

Perhaps stating makes a better object of comparison for Austin’s purpose here than promising. (It is, anyway, another which Austin makes.) To state something is to represent oneself (truly or falsely) as a certain way; for a start as oneself judging that which is stated. If I state that that is a peccary beneath the oak, I represent myself as saddled by the world with so thinking; as being unable to think otherwise on that score while pursuing the goal \emph{truth}. I thus offer you relief from certain intellectual (and perhaps other) labour. I say that the cat has been put out. I thus offer you relief from looking for yourself. You may, of course, take up my offer or not, depending on what you think of me—just as Austin says you may, or may not, take my word when I say I know. And you may or may not be right to take up my offer, or to refuse it, depending on the work of discovery I actually have done, and on the work needed. Modulo nuances, so it also is if I say that I \emph{know} that the cat has been put out. There is work that must be done if one is to enjoy such status (\emph{what} work being liable to depend on circumstances). I may have done the needed work or not. I may accordingly be to be accorded, or not, the status in question.

In fact, this may be seen as picking up on another element in Cook Wilson and Prichard. Both insist that to see whether I know I must not try to examine my own mental state to see whether it has some feature which marks it as a state of knowledge---whether, e.g., I am `perceiving clearly and distinctly', where that is understood as an introspectibly detectable sort of awareness for one to have. Rather, I must turn my attention to the objects of my awareness---to the proof, say---and ask myself how they bear on the relevant candidate for knowledge—say, that there is no largest prime. As Prichard puts it, ``there is \ldots\ no special distinguishing characteristic of a belief which is true \ldots\ there is no way of discovering whether some belief is true except that of first obtaining knowledge of the fact to which the belief relates, that knowledge therefore necessarily not having been obtained by considering the truth of the belief'' (1950: 92-93); ``What is known \ldots\ is some part of an independent world \ldots\ The only way in which the nature of anything in this independent world can come to be known is \ldots\ by perceiving it \ldots'' (1950: 101).  Talk of frames of mind tends to lead away from this insight.

The second main point is the application of Austin’s view of language, as in the last section, to the general area of epistemic status. The most concise and full statement of that application is in lecture 10 of \emph{Sense and Sensibilia}:
\begin{quote}
	It seems to be fairly generally realised nowadays that if you just take a bunch of sentences \ldots\ impeccably formulated in some language or other, there can be no question of sorting them out into those that are true and those that are false; for \ldots\ the question of truth and falsehood does not turn only on what a sentence is, nor yet on what it \emph{means}, but on, speaking very broadly, the circumstances in which it is uttered. Sentences are not as such either true or false. But it is really equally clear \ldots\ that for much the same reasons there could be no question of picking out from one’s bunch of sentences those that are evidence for others, those that are `testable', or those that are `incorrigible'. (1962: 110-111)
\end{quote}
Nor, Austin adds, are there sentences which are intrinsically in need of evidence, or knowledge of which must, intrinsically, rest on such-and-such grounds. All of which also applies to what speaks of someone \emph{knowing} such-and-such (an application elaborated in more detail in our other source here, “Other Minds” (1946)). The general point—Austin’s development of Cook Wilson---can be put as follows. Consider a sentence such as `Fawns gambol', or `There is red meat on the white rug'. The first, by virtue of meaning what it does, speaks of, or represents, (on some aspect of those verbs) Fawns as gambollers. But this leaves room for different things to be said—for indefinite variety in what is said—on different occasions, in using that sentence to say fawns to be gambollers; things each of which would vary in the conditions under which they would be true. For example, it may matter to the truth of some such things, but not to that of others, how fawns would behave on ritalin. Similarly, for different uses of that second sentence, raw liver, or lightly seared ribeye, may, or may not, count as red meat; ribeye in butcher’s paper may or may not count as on the \emph{rug}. Similarly, then, for such ascriptions of epistemic status to a fact as that it is evidence for such-and-such, or to a person as that he knows such-and-such. Take, for example, ‘Sid knows that the cat has been put out.’ By virtue of what it means (and reference to Sid, a time, and a given cat), these words speak of \emph{Sid} then knowing that that cat has been put out. But, the point is, in using them to speak of that, one \emph{might} say any of an indefinite variety of things to be so; \emph{would} say different things to be so on different occasions for such use, where each such thing \emph{would} be so under different ranges of conditions. So that some such things may, in fact, be true while others are false. So it is not, in general, simply true that Sid’s condition---his being as he is---\emph{is} his knowing that \( P \), or is his not knowing this, independent of any occasion for asking what he knows.

As it stands, this is a bare skeleton of a position. Again, the flesh Austin places on those bones is, for the most part, in (1946). But it already allows us to understand something, perhaps otherwise puzzling, which Austin says about evidence. He remarks,
\begin{quote}
	The situation in which I would properly be said to have evidence for the statement that some animal is a pig is that, for example, in which the beast itself is not actually on view, but I can see plenty of pig-like marks on the ground outside its retreat. If I find a few buckets of pig-food, that’s a bit more evidence, and the noises and the smell may provide better evidence still. But if the animal then emerges and stands there plainly in view, there is no longer an question of collecting evidence; its coming into view doesn’t provide me with more \emph{evidence} that it’s a pig, I can now just see that it is, the question is settled. (1962: 115)
\end{quote}
The notion of evidence in question here---what Austin takes to be the notion which the English `evidence' expresses---is one on which evidence admits of being stronger or weaker, better or poorer; accordingly one on which even the best evidence is compatible with the non-obtaining of what it is evidence for. Just so, then, not merely seeing the pig, but being able to see \emph{that} there is a pig there, is \emph{not} evidence in this sense. Where I am in this position, there is no logical gap between \emph{how} I know there is a pig there---I see (by sight) that there is---and its being so that there is a pig there, no room at all for me to be in that position while there fails to be a pig. (Nor, where I plainly see the pig before me, is there clearly an answer to the question just how much \emph{evidence} for this those pig-droppings now in fact supply. Against just what sorts of pigless cases of their presence are we supposed to measure the strength of such supposed evidence?) There is here, then, the same relation between my grounds for saying so and what I would thus say to be so as there is in the case of the proof that I grasp that there is no largest prime and there being no largest prime. So if I can be in this position, then I can know that there is a pig before me on Cook Wilson’s demanding conception of what knowledge is. At which point it becomes completely unstartling that, as Cook Wilson, Prichard and Austin all hold, there is no such thing as knowledge by evidence. To deny that there is such a variety of knowledge is now \emph{not} to deny that one may know such things as that there is a pig before him.

Here Austin’s point about occasion-sensitivity applies. What may sometimes count as my seeing that there is a pig before me may also, sometimes not. (‘Sometimes’: on some occasions for the counting.) So \emph{I}, in being as I now am (where I stand before the pig in plain view) may sometimes count, and sometimes not, as seeing that there is a pig before me. There are, accordingly, different things to be said in saying me to see this, and, correspondingly, different things to be said in saying me to know it. For some of these, epistemic status is so to be understood that my position \emph{is} one of seeing that there is a pig before me, for others it is to be so understood that my position is not that. There \emph{are} circumstances in which it matters to an answer to the question whether I can see that there is a pig before me whether I can, by sight, rule out (certain sorts of) ringers. Thus it is that worries arise as to how Cook Wilson’s conception of knowledge leaves room for knowledge of the world we all co-inhabit at all. But there are also circumstances in which the answer to that question does not turn on such things: where it comes to measuring my status, ringer pigs are just not in the cards.

As noted already, this is not just a point about knowledge gained through perception. If I grasp a proof that there is no largest prime---if I see how, and that, what the proof cites proves this---then there is no gap between how I know---what it is which reveals this to me—and the fact that there is no largest prime. But do I see how, and that, the proof proves? Do I see what rules out all possibility of a ringer-proof? Within Austin’s general framework, the answer to that is an occasion-sensitive matter. Which means that the fact that one might sometimes need to worry about ringer-proofs—that ringers always are \emph{conceivable}---does not show that one can never count as seeing of a proof that it is a proof. That idea of occasion-sensitivity which marks Austin’s conception of language and its relation to thought is at work throughout in making Cook Wilson’s conception of knowledge unproblematic, at least in such ways.

Similarly, occasion-sensitivity allows us to understand Austin’s response to Ayer’s idea of incorrigible statements. Here Austin remarks,
\begin{quote}
	If, when I make some statement, it is true that nothing whatever could in fact be produced as a cogent ground for retracting it, this can only be because I am in, have got myself into, the very best possible position for making that statement—I have, and am entitled to have, \emph{complete} confidence in it when I make it. But whether this is so or not is not a matter of what \emph{kind of sentence} I use in making my statement, but of what \emph{the circumstances are} in which I make it. If I carefully scrutinise some patch of colour in my visual field, take careful not of it, know English well, and pay scrupulous attention to just what I’m saying, I may say, `It seems to me now as if I were seeing something pink'; and nothing whatever could be produced as showing that I had made a mistake. But equally, if I watch for some time an animal a few feet in front of me, in a good light, if I prod it perhaps, sniff, and take note of the noises it makes, I may say, `That’s a pig', and this too will be `incorrigible', nothing could be produced that would show that I had made a mistake. (1962: 114)
\end{quote}
Being completely entitled to one’s confidence that \( P \) is an epistemic status there is for one to enjoy (or fail to). Austin’s point then is: on any understanding as to what status this is on which I can count as enjoying it with respect to its being, for me now, as if I were seeing something pink, it is also one I can (or might) count as enjoying with respect to there being a pig before me now---and with anything else one might ever know, or of which one might be ignorant. My condition as I now view the pig before me is, quite likely, one which would sometimes count, and sometimes not, as my enjoying that status---depending on the occasion for the counting. But then, if its being for me as if I were seeing something pink is, genuinely, something to be known, or of which to be ignorant, then the same goes for that. Which makes room for knowing there is a pig before me to be an attainable status.

It is important here to combine the two main points just made. Thinking of knowledge as a frame of mind at least tempts us to think this way: I am now in the frame of mind I am in; either \emph{this} frame of mind is one of knowing there is a pig before me, or it is not. Which is so can only depend on how I am in being as I am. If we take a broad view of frames of mind, then, perhaps how \emph{I} now am depends on the broader circumstances in which I find myself. I may or may not, e.g., have views on water depending on what those broader circumstances are. But this is \emph{all} it can depend on. But suppose, instead, that knowing is a status I may or may not enjoy. To enjoy it, e.g., \emph{in re} whether there is a pig before me is to have done the needed work to count as authoritative on that subject. Whether I have done that work depends on what the needed work would be. And now if we ask just what work \emph{would} be needed (or sufficient) for this, an answer is more than likely to depend on circumstances. Not the circumstances in which \emph{I} find myself. That is not the view of knowledge now on offer. But rather the circumstances in which that question is raised (or might be raised). It is of \emph{me}, as I am now \emph{in re} the pig in plain view before me that there is a variety of things to say to be so in saying me to know that there is a pig before me.

[Which forestalls a common objection to occasion-sensitivity about knowledge. Abstracting from some irrelevant rhetoric about `high' and `low' standards for knowledge, the form of the objection is: as Sid stands before the pig, his circumstances (or `the' ones which then obtain) determine what, in them, would count as knowing there is a pig before him. Suppose, as it happens, they determine standards on which he does know this. (Such had better sometimes happen if occasion-sensitive is going to gain for us knowledge of the empirical world at all. Or so goes the objection.) But then there will be other circumstances which determine other standards, such that, finding ourselves in them, in which we, privy to all that Sid then was, would have to say that we did not know whether there was a pig. (Or, transforming the case, we may be just as well off in our circumstances as Sid was in his, except that, by the standards for knowledge set by our circumstances, we do not count as knowing there is a pig before us.) How can we, not counting as knowing whether there was a pig before Sid (or is one now before us) credit Sid with knowing this? For one thing, if we know that Sid knew there was a pig before him (to take that version) then, presumably, we know that there was a pig before him. But this contradicts the supposition. So, it seems, it is incoherent to suppose knowledge to be an occasion-sensitive matter.]

The first mistake here is to suppose that it is the circumstances I am in which determine what would count as \emph{my} knowing whether there is a pig, and the circumstances you are in which determine what would count as your knowing this. Such a supposition is inconsistent with the present story. On it, it is the circumstances in which knowledge is attributed which determine, first, what would be said to be so in attributing it, so, second, when such an attribution would be correct. There are both true and false things to be said of me, in my circumstances, in saying me to know that there is a pig before me. What remains true is this: someone may have said me to have known, on a certain occasion, that there was a pig before me, and thus have spoken truth, while someone else, at the same time as that first person, or at a later one, may have said me not to have known, in the situation of which the first person spoke, that there was a pig before me, and thus also have spoken truth. Such is allowed for, on the account now on offer, by the fact that what the first of these people said to be so in saying me to have known such-and-such may not be what the second said not to be so in saying me not to know this. The two remarks may be consistent. It is the same here as where one person says there to be meat on the rug, steak in butcher paper counting as on the rug for the purpose of what he says, and another says there to be no meat on the rug, steak in butcher paper \emph{not} counting as on the rug for purposes of what he said.

If we think of knowledge as a status rather than a frame of mind, one aspect of Cook Wilson’s, and Prichard’s story does drop out. Their story makes it essential to knowing that if I know something, I can, by reflection alone, come to see that I know it; and if I do not know something (even if I am fooling myself as to this), I could (in principle) by reflection alone, come to see that this is so. That idea helps show how Prichard and Cook Wilson think of frames of mind. For it takes a certain conception of this for the idea to make sense. Sid faces the pig in plain view. It is not as if he could, by reflection alone, see whether or not, on our occasion for answering the question whether he knows there is a pig, he would (ought to) count as enjoying the relevant status---as having done the work needed to count \emph{there} as authoritative. For that, he would need to be acquainted with \emph{our} circumstances, which, in general, would not be there for him to be acquainted with at all. Nor need it be so that, by reflection alone, he could tell whether, in his circumstances, one would need, e.g., to have taken such-and-such precautions against ringers if one were to count as enjoying the relevant status. Rather, he has done the work he has done; and, as it may be, in given circumstances that does count as work enough for achieving that status. It neither so counts, or fails to, of course, unless those circumstances are such as to make relevant questions as to what Sid knows arise.

Austin thus offers a way of understanding knowing as (nothing less than) having proof, in the way Cook Wilson and Prichard do, while making its instancing coincide with what we are prepared to acknowledge its instancing to be---as envisaged by Moore, though matched with a new view as to what it is that we are prepared to recognize. Austin’s reworking of Cook Wilson, along with his ideas about language and thought, were things up with which (most of) Oxford was not for long prepared to put. There are probably several intersecting reasons for this, among which the rapidly accelerating Americanization of everything, including philosophy. Whatever Austin’s fate, though, Cook Wilson’s idea hardly disappeared. Rather, it, or its main part, lived on, most notably in the work of John McDowell. There, touched by different Austinian ideas, it appeared as an anti-hybridism, or (the same thing) disjunctivism about knowledge. McDowell, like Cook Wilson and Prichard, rejects the idea that knowledge that \( P \) could be some sort of condition in which all that one was actually aware of, so far as that went---whatever it was that might answer the question how he knows, what it is that reveals this to him---was compatible with \( P \) not being so.

McDowell shares Cook Wilson’s conception of knowing as nothing less than having proof---as having that \( P \) revealed to one by that which leaves no room for \( P \) not to be so. McDowell arrives at this point by considering the idea that ``we ought to be able to achieve flawless standings in the space of reasons by our own unaided resources, without needing the world to do us any favours'' (1995/1998: 395-396). This, as he sets things out, is the first step to a hybrid conception of knowledge. Knowledge, on this idea, would consist---at least in part---in having conducted our cognitive affairs according to the highest standard rationality imposes on this, where the responsibility for having so conducted ourselves is \emph{entirely} ours; there is \emph{no} demand for the world to be obliging if we are to reach the mark. But knowledge is factive: you do not know that \( P \) if \( P \) is not the case. So we may ask, for given \( P \), whether there is any way of conducting one’s rational affairs which one can guarantee for himself---see by mere reflection---that he has engaged in---a way for which there simply are no ringers---which, where it has a given terminus, leaves no room whatever for \( P \) not to be the case. Descartes asked that question. On reflection, the answer seems to be ``No''; not just in the case of perceptually based knowledge---most of us know all too well that it is possible for it to seem to one that he has, unmistakably, a proof of a proposition in geometry or number theory when he does not.

So the above requirement on knowing cannot be the \emph{only} requirement. There must be an extra requirement, over and above any such requirement on our conduct of our cognitive affairs, which is, at least in part, that \( P \) be the case. Thus a hybrid conception: knowing that \( P \) is part our own responsibility, in the way just sketched, and part the world’s. McDowell then has this to say about that conception:
\begin{quote}
	In the hybrid conception, a satisfactory standing in the space of reasons is only part of what knowledge is; truth is an extra requirement. So two subjects can be alike in respect of the satisfactoriness of their standing \ldots\ although only one of them is a knower, because only in her case is what she takes to be so actually so. \ldots\ Its being so is conceived as external to the only thing that is supposed to be epistemologically significant about the knower herself, her satisfactory standing in the space of reasons. That standing is not itself a cognitive purchase on its being so \ldots\ Then how can the unconnected obtaining of the fact have any intelligible bearing on an epistemic position that the person’s standing \ldots\ is supposed to help constitute? (1995/1998: 403)
\end{quote}
McDowell concludes that it cannot. It is as with evidence, conceived as what is liable to be stronger or weaker. As Cook Wilson notes, when we reflect on our position where such a thing is how we know that \( P \), we see that our reasons for taking it that \( P \) are consistent with \( P \) not being so; hence that we do not know that \( P \).

McDowell’s remedy (what has come to be known as disjunctivism) is to suppose that there are two sorts of case. No one, of course, can know that \( P \) if \( P \) is not so. But suppose that \( P \) \emph{is} so. Then there are still two (generally attainable) ways of standing towards \( P \) being so. In one way, as McDowell puts it, the fact that \( P \) is made apparent to one: What reveals to you that \( P \) is nothing less than what (recognizably) leaves no gap between itself and \( P \) being so---e.g., I know that there is a pig in the pen because I see it for myself. In the other case, this is not so. I have at best defeasible reasons for \emph{thinking} that there is a pig in the pen. It is thus not \emph{apparent} to me that \( P \); things merely so appear (vide 1982/1998: 386-387). The first sort of case---its being apparent that \( P \)---is knowledge. The second sort---its merely appearing that \( P \)---is not. If I can just \emph{see} the pig in the pen, and can recognise what I see for what it is, I need not worry about the strength of any reason I may have for taking there to be a pig there. We thus make room for experience to be a source of knowledge, within Cook Wilson’s conception of what knowledge is.

But suppose there is now, in plain view before me, a pig in the pen. McDowell suggests that there are two conditions I might thus be in. One of these is seeing for myself, hence knowing, that there is a pig in the pen, the other is not. Which of these conditions \emph{am} I in, in this case now? Suppose that being in the knowledgeable condition required me to be able to tell whether I am actually seeing a pig in a pen, or rather in some sort of mere ringer for this. Ringers being what they are, I would fail this requirement. So my condition would not be one of knowing. But, if McDowell is right, there are again, with respect to the ringers in question, two sorts of case. I might be required to be able to tell whether I am in such a ringer situation or not. But I might not. No such ringer may need ruling out by me in order for it to count as just apparent to me that there is a pig before me. Well, then, which condition am I in? And what guides towards an answer to this does McDowell make available?

Austin offers an answer to this question. It is, ``Bad question''. The question is a bad one for a quite specific reason. It supposes that there are two conditions for me to be in, \emph{in re} that pig in the pen, and that my being as I am just is my being in the one condition or the other. Whereas for Austin---knowledge being the occasion-sensitive sort of phenomenon it is---it is not \emph{my} circumstances as I stand there which decide what it is right (true) to say, but rather the circumstances in which I might be credited with, or denied, knowledge (that the pig is in the pen). To repeat, if you and I are confronting the question whether Sid is an authority as to pigs now being in the pen---if there is now some determinate such question for us to confront---then we confront the issue what would it be, now, for someone to be an authority as to that. Depending on our circumstances, this may or may not involve distinguishing porcine-infested pens from certain sorts of ringers for them. Following Austin thus gives us a principled reason for rejecting the question. But McDowell does not follow Austin, at least on this point. Oxford views of language had, in the interim, moved on. The question is, no doubt, still a bad one on McDowell’s view. But it remains an open question precisely why this is so.
