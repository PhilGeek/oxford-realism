%!TEX root = /Users/markelikalderon/Documents/oxford-realism/oxford.tex

\section{Perception} % (fold)
\label{sec:perception}

A concern for realism motivates a fundamental strand of Oxford reflection on perception. Begin with the realist conception of knowledge.  The question then will be: What must perception be like if we can know something about an object without the mind by seeing it? What must perception be if it can, on occasion, afford us with \emph{proof} concerning a subject matter independent of the mind? The resulting conception of perception is not unlike the conception of perception shared by Cambridge realists such as Moore and Russell. Roughly speaking, perception is conceived to be a fundamental and irreducible sensory mode of awareness of mind-independent objects, a non-propositional mode of awareness that enables those with the appropriate recognitional capacities to have propositional knowledge concerning that subject matter. 

The difference between Oxford and Cambridge realism concerns the extent of this fundamental sensory mode of awareness. Whereas Oxford realists maintained that perception affords us this sensory mode of awareness, Cambridge realists maintained that this mode of awareness has a broader domain. Let experience be the genus of which perception is a species. Cambridge realists maintained that \emph{all} experience, and not just perception, involves this non-propositional sensory mode of awareness. Cambridge realists are thus committed to a kind of \emph{experiential monism} (in Snowdon's \citeyear{Snowdon:2008oz} terminology)---the thesis that experience has a unitary nature. Specifically, all experience involves, as part of its nature, a non-propositional sensory mode of awareness. Even subject to illusion or hallucination, there is something of which one is aware. And with that, they were an application of the argument from illusion, or hallucination, or conflicting appearances away from immaterial sense data and a representative realism that tended, over time, to devolve into a form of phenomenalism.

Framing the discussion is the fundamental realist (or anti-idealist) commitment common to Cook Wilson and Moore---that the objects of knowledge are independent of the act of knowing. Suppose that in seeing the pig Sid is in a position to know various things about it. The pig is the object of Sid's knowledge in the sense that Sid knows something about \emph{it}---that the pig is before Sid, or that the pig is black, say. According to the fundamental realist commitment, the pig is the object of Sid's knowledge only insofar as it exists independently of Sid's knowing. 

This is a thesis about knowledge, not perception. What connects this thesis to perception is a doctrine whose slogan might be---\emph{perception is a form of knowing}. Perception, conceived as a form of knowing, is a sensory mode of awareness that makes the subject \emph{knowledgeable} of its object. In being so aware of an object, the subject is in a position to know certain things about it, depending, of course, on the subject's possession and exercise of the appropriate recognitional capacities in the circumstances of perception. The subject is knowledgeable of the object of perception in the sense that knowledge is \emph{available} to the subject in perceiving the object, whether or not such knowledge is in fact ``activated'' (in Williamson's \citeyear{Williamson:1990uq} terminology).

Suppose, then, that perception is a form of knowing in the sense that it makes the subject knowledgeable of its object. The objects of perception are then at least potential objects of knowledge. If knowledge is always knowledge of a mind-independent subject matter, and the objects of perception are at least potential objects of knowledge, then it follows that the objects of perception are themselves mind-independent and so independent of the act of perceiving. In this way the doctrine that perception is a form of knowing allows the realist conception of knowledge to have implications for how perception is properly conceived in light of it.

Working out the demands of the realist conception of knowledge on the nature of perception was subject to internal and external pressures. 

Internally, the core features of the realist conception of knowledge get differently conceived by different authors, in a process of refinement and extension, and so the demands that conception of knowledge places on the nature of perception are themselves reconceived. Importantly, an independent aspect of Cook Wilson's conception of knowledge, \emph{the accretion}, an aspect endorsed by Prichard and rejected by Austin, turns out to be inconsistent with the idea that perception makes the subject knowledgeable of a mind-independent subject matter. So the development of the realist conception of knowledge involved not merely refinement and extension, but elimination as well.

Externally, Oxford reflection on perception is subject to alien influences, in particular, Cantibrigian and Viennese influences. Thus Price comes to Oxford from Cambridge where he was Moore's student. Paul comes to Oxford from Cambridge as well but studied with Wittgenstein. And Ayer, given Ryle's encouragement, studied for a time with the logical positivists in Vienna. Incorporating the insights and resisting the challenges posed by these alien influences play an important part in the development of philosophy of perception in Oxford.

Cook Wilson never published on perception. The main source of Cook Wilson's \citeyearpar[764--800]{Cook-Wilson:1926sf} views on perception is a letter of July 1904 criticizing Stout's \citeyearpar{Stout:1903zl} ``Primary and Secondary Qualities''. To highlight the connections between his realist conception of knowledge and his views about perception, it is useful to begin, however, with Cook Wilson's \citeyearpar[801--808]{Cook-Wilson:1926sf} earlier letter of January 1904 to Prichard. There Cook Wilson discusses two variants of a fundamental fallacy concerning knowledge or apprehension.

The first variant is the idealist attempt to understand knowledge as an activity. If knowledge is an activity, then in knowing something a subject must \emph{do} something to the object known. But this, Cook Wilson claims, is absurd. The object of knowledge must be independent of the subject's knowing it, if coming to know is to be a discovery: 
\begin{quote}
	You can no more act upon the object by knowing it than you can `please the Dean and Chapter by stroking to dome of St. Paul's'. The man who first discovered the equable curvature meant equidistance from a point didn't supposed that he `produced' the truth---that absolutely contradicts the idea of truth---nor that he changed the nature of the circle or curvature, or of the straight line, or of anything spatial. \citep[802]{Cook-Wilson:1926sf}
\end{quote}

The second variant is the representative realist's attempt to understand knowledge and apprehension in terms representation. Whereas the idealist attempts to explain apprehension in terms of \emph{apprehending}, the representative realist attempts to explain apprehension in terms of \emph{the object apprehended}, in the present instance, an idea or some other representation. The problem is that this merely pushes the problem back a level:
\begin{quote}
	The chief fallacy of this is not so much the impossibility of knowing such image is like the object, or that there is any object at all, but that it assumes the very thing it is intended to explain. The image itself has still to be \emph{apprehended} and the difficulty is only repeated. \citep[803]{Cook-Wilson:1926sf}
\end{quote}

How are the fallacies of explaining apprehension in terms of apprehending and in terms of the object of apprehension variants of the same fallacy? Both attempt to \emph{explain} knowledge or apprehension:
\begin{quote}
	Perhaps most fallacies in the theory of knowledge are reduced to the primary one of trying to \emph{explain} the nature of knowledge or apprehending. We cannot \emph{construct knowing}---the act of apprehending---out of any elements. I remember quite early in my philosophic reflection having an instinctive aversion to the very expression `\emph{theory} of knowledge'. I felt the words themselves suggested a fallacy---an utterly fallacious inquiry, though I was not anxious to proclaim <it>. \citep[803]{Cook-Wilson:1926sf}
\end{quote}
This is a clear statement of the anti-hybridism or anti-conjunctivism about knowledge that \citet{McDowell:1982kx} and \citet{Williamson:2000lr} will later defend. So conceived, knowledge is not a hybrid state consisting of an internal, mental state and the satisfaction of some external conditions. Cook Wilson's aversion to the ``theory of knowledge'' is just an aversion to explaining knowledge by constructing it out of elements, and this skepticism will be echoed by Prichard, Ryle, and Austin and in precisely these terms.

Suppose that perception makes the subject knowledgeable of a mind-independent subject matter. Suppose further that the knowledge the subject is in a position to acquire cannot be explained or constructed out of elements. What must perception be like to make us knowledgeable of the environment in that sense? Must perception itself be non-conjunctive? Does Cook Wilson himself endorse anti-hybridism about perception? In his letter to Stout he does defend a conception of perception as the direct apprehension of objects spatially external to the perceiving subject. And in the letter to Prichard he does at one point speak indifferently of knowledge, apprehension, and perception. If the main conclusions of that letter are meant to apply to all three, then Cook Wilson endorses anti-hybridism about perception. Neither consideration is decisive. More telling, however, is that the variant fallacies of explaining apprehension in terms of apprehending and the object apprehended are echoed in the letter written later that year to Stout on perception and, indeed, form the core of its content. In particular, both idealist and representative realist accounts of perception are criticized in line with the two variant fallacies concerning knowledge or apprehension. Let's consider these in turn.

First, like \citet{Moore:1903uo}, Cook Wilson emphasizes the distinction between the object of perception and the act of perceiving. In perceiving an object, the object appears to the subject, and so the subjective act of perceiving is sometimes described as an \emph{appearance}. Given the distinction between the object perceived and the act of perceiving, an appearance, so understood, is necessarily distinguished from the object. However, Cook Wilson warns against a misleading ``objectification'' of appearing:
\begin{quote}
	But next the \emph{appearance}, though properly the appear\emph{ing} of the object, gets to be looked on as itself an object and the immediate object of consciousness, and being already, as we have seen, distinguished from the object and related to our subjectivity, becomes, so to say, a mere subjective `object'---`appearance' in that sense. And so, as \emph{appearance} of the object, it has now to be represented not as the object but as the phenomenon caused in our consciousness by the object. Thus for the true appearance (=appearing) to us of the \emph{object} is substituted, through the `objectification' of the appearing as appearance, the appearing to us of an \emph{appearance}, the appearing of a phenomenon caused in us by the object.  \citep[796]{Cook-Wilson:1926sf}
\end{quote}

If perceptual appearances are ``the appearing of a phenomenon caused in us by the object'', then it would be impossible for a subject to come to know about the mind-independent object on the basis of its perceptual appearance and hence impossible to discover how things stand with a mind-independent subject matter by perceiving:
\begin{quote}
	It must be observed that the result of this is that there could be no direct perception or consciousness of Reality under any circumstances or any condition of knowing or perceiving: for the whole view is developed entirely from the fact that the object is distinct from our act of knowing it or recognizing it, which distinction must exist in any kind of knowing it or perceiving it. From this error would necessarily result a mere subjective idealism. Reality would become an absolutely unknowable `Thing in Itself', and finally disappear altogether (as with Berkeley) as an hypothesis that we could not possibly justify. \citep[797]{Cook-Wilson:1926sf}
\end{quote}
This straightforwardly parallels the fallacy of explaining apprehension in terms of apprehending. 

Second, Cook Wilson criticizes Stout’s \citeyearpar[144]{Stout:1903zl} representative realism, in particular his claim that the sensations which mediate knowledge of external qualities such as extension do so only in so far as ``they represent, express, or stand for something other than themselves''. The basis of of his criticism involves negative and positive claims about the nature of representation. The negative claim is that nothing is intrinsically representational: ``Nothing has \emph{meaning} in itself'' \citep[770]{Cook-Wilson:1926sf}. The positive claim is put as follows: ``Representation is our subjective act. ... It is \emph{we} who mean'' \citep[770]{Cook-Wilson:1926sf}. According to Cook Wilson, then, representation is personal. It is we who mean. So conceived, representation is something that the subject does. 

How, according to Stout, might the sensation of extension ``represent, express, or stand for'' extension? Plausibly in two ways: by resembling extension or by necessarily covarying with the presence of extension. However, the natural relations of mimesis and necessary covariation are \emph{impersonal}---they obtain independently of anything that the subject does. And since they are \emph{symmetric}, this has the surprising consequence that external qualities represent sensations. However, if it is \emph{we} who mean, if representation is something that a subject does, then the natural relations of mimesis and necessary covariation could not make a sensation represent an external quality (let alone make an external quality represent a sensation, for plausibly nothing does). These are not two analyses of different notions of representation; at most, mimesis and necessary covariation are merely natural relations that \emph{incline} us to represent things by means of them---they are merely relations that can be exploited by a subject's representational ends:
\begin{quote}
	It is we who make the weeping willow a symbol of sorrow. There may of course be something in the object which prompts us to give it a meaning, e.g., the resemblance of the weeping willow to a human figure bowed over in the attitude of grief. But the willow in itself can neither `mean' grief, nor `represent' nor `stand for' nor `express' grief. \emph{We} do all that.  \citep[770]{Cook-Wilson:1926sf}
\end{quote}
% The weeping willow resembles a human figure bowed over in the attitude of grief. This presents a subject with an opportunity to exploit that resemblance for their own representational ends, at least if they are apprised of that resemblance. 
In using the willow to represent grief, the subject must apprehend the content of that representation. And that, according to Cook Wilson, is precisely what prevents representation from figuring in an explanation of perceptual apprehension. Any such explanation would be circular and, hence, no explanation at all. This straightforwardly parallels the fallacy of explaining apprehension in terms of the apprehension of an idea or representation more generally.

Thus Cook Wilson's discussion of perception in his letter to Stout, parallels his discussion of knowledge in his letter to Prichard. In particular the two fallacies of explaining apprehension in terms of apprehending and in terms of the object apprehended (a representation) arise in the perceptual case as well. This raises the question whether in the perceptual case these fallacies are variants of the fundamental fallacy of trying to \emph{explain} perception in more fundamental terms. Just as knowledge cannot be explained in terms of belief that meets further  external conditions, perhaps perception cannot be explained in terms of, say, experience or appearance that meets further external conditions. Cook Wilson expresses his skepticism about such explanations in the case of knowledge by denying that there is any such thing as a theory of knowledge. Farquharson in the postscript to \emph{Statement and Inference} reports a similar attitude in the perceptual case: ``He came to think of a theory of Perception as philosophically preposterous'' \citep[882]{Cook-Wilson:1926sf}. 

The evidence is not decisive. However, even if we were convinced that Cook Wilson accepted an anti-hybridist conception of perception, we would remain unclear why the realist conception of knowledge requires this. A reason begins to emerge with Prichard's case \emph{against} the idea that perception is a form of knowing. While Prichard opposes the doctrine that links the realist conception of knowledge with the nature of perception, his discussion reveals some of what is required if one were to retain the doctrine distinctive of twentieth century realists that perception makes us knowledgeable of a mind-independent subject matter.

Cook Wilson provides neither a theory of perception nor of the nature of appearances. However, Prichard's \citeyearpar{Prichard:1906gf,Prichard:1909yg} theory of appearing builds on some of Cook Wilson's insights. Following Cook Wilson, Prichard holds that the object of perception, like the object of knowledge, must be independent of the act of perceiving, and that an appearance is properly understood as an appearing of a mind-independent object to the perceiving subject. \citet{Prichard:1909yg} thus opposes any conception of appearance where appearances are states of a subject produced by external objects. However, from at least since ``Seeing Movement'' written in 1921, Prichard abandons the theory of appearing. Specifically, he comes to deny that the objects of perception are mind-independent objects located in space, coming to favor, instead, a Berkelean conception of perception where the objects of perception depend on our perceptual experience of them. At the heart of this change of mind is a doubt about whether perception could be a form of knowing.

% (For criticism see Price \citeyear{Price:1932fk}; the theory of appearing is subsequently defended by Alston \citeyear{Alston:1993zl}, Chisholm \citeyear{Chisholm:1950rj}, and Langsam \citeyear{Langsam:1997md})

The central argument occurs in Prichard's \citeyearpar{Prichard:1938ve} ``Sense Datum Fallacy''. His main target is the sense datum theory of Cambridge realists such as Moore and Russell. Like their Oxford counterparts, the Cambridge realists held that the object of knowledge is independent of the act of knowing, and that perception is a form of knowing. Cambridge realism departs from Oxford realism in its adherence to a further thesis. Cambridge realists held, in addition, that there is something of which a subject is aware in undergoing sense experience whether perceiving or no. According to the theories of \citet{Moore:1953nx}, \citet{Russell:1912uq}, and \citet{Price:1932fk}, sense data are whatever we are aware of in sense experience. This characterization of sense data is \emph{neutral} in the sense that it assumes nothing about the substantive nature of objects that play this epistemic role. Further argument is required to establish substantive claims about the nature of sense data. We have already noted how the sense data theory is committed to an experiential monism---all experience involves, as part of its nature, a non-propositional sensory mode of awareness. A further commitment is presently important. For so conceived, sense data are objects whose substantive nature is open to investigation independent of our acts of awareness of them. It is this consequence of the conjunction of the realist conception of knowledge, the conception of perception as a form of knowing, and the sense datum theory that is Prichard's primary target. And Prichard's central thought is that perception could not make one knowledgeable of its object, since the object of perception depends on the subject's experience of it in a way that the object of knowledge could not.

Much of Prichard's case is a variant of Berkeley's \citeyearpar{Berkeley:1734fk,Berkeley:1734zp} critique of \citet{Locke1690An-Essay-Concer}. However, two arguments go beyond the familiar Berkelean critique. The first derives from a peculiar feature of the Cook Wilsonian epistemology, the accretion, and the second is explicitly derived from \citet{Paul:1936kd}. Both present important morals for Oxford realism. The moral of the first argument is that the accretion must be abandoned if Oxford realism is to be sustained. The moral of the second argument is that the realist conception of knowledge and the conception of perception as a form of knowing requires abandoning the Cambridge realist's commitment to experiential monism (though it will take the work of \citet{Austin:1962lr} and \citet{Hinton:1973js} to begin to vindicate this).

The first argument can seem like a variant of the argument from illusion though it really has a very different character:
\begin{quote}
	\ldots\ if perceiving were a kind of knowing, mistakes about what we perceive would be impossible, and yet they are constantly being made, since at any rate in the cases of seeing and feeling or touching we are almost always in a state of thinking that what we are perceiving are various bodies, although we need only to reflect to discover that in this we are mistaken. \citep[11]{Prichard:1938ve}
\end{quote}
The passage is frustrating in its lack of explicitness. Indeed in the last line Prichard seems to echo Hume’s \citeyearpar[§XII]{Hume:1740lr} contention that it takes the slightest philosophy to show naïve realism to be false. 

Suppose a pig is in plain view of Sid, and Sid can recognize as a pig the animal that he sees. It might seem that what Sid is thus aware is incompatible with there not being a pig before him. In which case, perception affords Sid something akin to proof of a porcine presence. In this way, perception can seem to make the subject knowledgeable of a mind-independent subject matter. Prichard's insight is that this picture is incompatible with a further feature of Cook Wilson's conception of knowledge, \emph{the accretion}. % Prichard understands the accretion as follows:
% \begin{quote}
%   We must recognize that whenever we know something we either do, or at least can, by reflecting, directly know that we are knowing it, and that whenever we believe something, we similarly either do or can directly know that we are believing it and not knowing it. (1950: 86)
% \end{quote}
If Sid knows that P, Sid can know upon reflection that he knows that P. And if Sid has some attitude other than knowledge to that proposition, then Sid can know upon reflection that his attitude is something other than knowledge. Knowledge admits of no ringers---a state indiscriminable upon reflection from knowledge just is knowledge. What would it take for perception to make us knowledgeable of a mind-independent subject matter if there are no ringers for knowledge? If Sid's seeing the pig makes him knowledgeable of the pig's presence, then Sid must recognize that what he is aware of in seeing the pig is incompatible with the pig's absence. But is Sid in seeing the pig in a position to recognize that? After all, there are situations indiscriminable upon reflection from seeing a pig that do not involve the pig's presence. Sid's hallucination of the scene would be indiscirminable upon reflection from his perceiving it. If what Sid is aware of in seeing the pig is not discriminable upon reflection from what, if anything, he is aware of in hallucinating the pig, then it could seem that he is not in a position to recognize that what is aware of in seeing the pig is incompatible with the pig's absence. He would lack proof of a pig before him. Since perception admits of ringers, it could not be a source or form of ringerless knowledge.

This argument reveals a tension within the Oxford realism of Cook Wilson and early Prichard. If Cook Wilson and early Prichard were right in claiming that the objects of knowledge are mind-independent objects, and the objects of perception are at least potential objects of knowledge, then these claims can only be sustained by abandoning the accretion. Indeed, it is telling that Austin jettison's just this feature of Cook Wilson's epistemology.

Prichard's second argument derives from \citet{Paul:1936kd}. Arguably it has ancient roots as well. At the very least, it is a variant of Berkeley's interpretation of the \emph{Theatetus} (\emph{Siris} §§ 253, 304-5). On the Berkelean interpretation, the objects of perception are in a perpetual flux of becoming. In perception, every subject is aware of the sensible qualities whose coming and going constitute the flux since every subject is the ``measure'' of what they perceive. Though perception affords us with awareness of its objects, this mode of awareness could not constitute knowledge since knowledge pertains to \emph{being}, not \emph{becoming}. More prosaically, the objects of perception could not have a continuing identity through time, if every feature they manifest is relativized to a perceiver at a time. Nor could the objects of perception be publicly accessible to different perceivers. But this would preclude the objects of perception from being objects of knowledge if knowledge is to have a mind-independent subject matter \citep[see][for further discussion of the Berkelean interpretation]{Burnyeat:1990dp}. Paul's discussion of sense data is of a piece. Paul, and Prichard following him, emphasize our inability to decide key questions about the persistence and publicity of sense data. If sense data are meant to be objects open to investigation independent of our awareness of them, then such questions should be settled by looking to the sense data themselves. But our inability to decide such questions belies this thought. At best, sense data are shadows cast by experiences that can be elicited by suitably affecting the mind. So conceived, open questions about the nature of sense data are resolved not by investigation but by linguistic decision. Paul is clearly influenced by Wittgenstein's discussion of sense data in \emph{The Blue Book}:
\begin{quote}
    Queerly enough, the introduction of this new phraseology has deluded people into thinking that they had discovered new entities, new elements of the structure of the world, as though to say “I believe that there are sense data” were similar to saying “I believe that matter consists of electrons”. \citep[70]{Wittgenstein:1958rr}
\end{quote}

Suppose the central claim here is right---that sense data do not have a substantive nature open to investigation independent of our awareness of them in sense experience. There are at least three potential morals:

\begin{enumerate}
	\item One might claim that sense data constitutively depend on our awareness of them in sense experience. Sense data would be in this regard like Berkelean ideas. Sense data would lack a substantive nature independent of our awareness of them. Though, Ayer, at least, would regard this Berkelean alternative as piece of substantive metaphysics on a par with Moorean sense data. (Though neither deploy the sense-data vocabulary, Berkeley and later Prichard adopt this alternative.)
	\item One might deny that there are any substantive facts about the nature of sense data that are open to investigation independent of our awareness of them in sense experience. (Wittgenstein, Paul, and Ayer adopt this alternative,)
	\item One might retain the conception of perception, common to Oxford and Cambridge realists, as a sensory mode of awareness that makes one knowledgeable of a mind-independent subject matter by abandoning the fundamental claim of the sense-datum theory---that there is an object of which we are aware whenever we undergo sense experience---and the experiential monism that came in its wake. (Austin and Hinton adopt this alternative.)
\end{enumerate}

There have been relatively few takers for the Berkelean alternative (though see Foster \citeyear{Foster:00ny} for a recent defense). We will set it aside and focus, instead, on the second and third alternatives, as represented by the work of Ayer and Austin respectively.

In the \emph{Foundations of Empirical Knowledge}, \citet{Ayer:1958kx} takes over from the logical positivists the general idea that there is no substantive metaphysics and that metaphysical disagreements are better understood as practical disagreements about what language or conceptual scheme to adopt. Ayer applies this idea to sense data and suggests that talk of sense data is just an alternative way of talking about facts that all of us can agree about, namely, facts about appearances. Ayer cites \citet{Paul:1936kd} as an antecedent. However, as previously noted, the most likely proximate influence on Paul is the middle period Wittgenstein and not the logical positivists. Moreover, it is clear that Paul's attitude toward this claim is more ironic than Ayer's:
\begin{quote}
    The important point is whatever we do is not demanded by the nature of objects which we are calling `sense-data', but that we have a choice of different notations for describing observations, the choice being determined only by the greater convenience of one notation, or our personal inclination, or by tossing a coin. \citep[74]{Paul:1936kd}
\end{quote}

Ayer understands the argument from illusion to establish not that there are sense data, distinct from material objects, that are the objects of sensory awareness, if this is to be understood as a substantive metaphysical claim; rather, the argument from illusion highlights the practical need to regiment our perceptual vocabulary. According to Ayer, ``see'', ``perceive'', and their cognates have readings that implicate the existence of the object seen or perceived \emph{and} readings that fail to so implicate. Sense-datum theorists, as Ayer understands them, simply regiment in favor of the existential reading. The practical need for talk of immaterial sense data arises in the context of an epistemological project:
\begin{quote}
    For since in philosophizing about perception our main object is to analyse the relationship of our sense-experience to the propositions we put forward concerning material things, it is useful for us to have a terminology that enables us to refer to the contents of our experiences independently of the material things they are taken to present. \citep[]{Ayer:1958kx}
\end{quote}

That project involved two central claims:
\begin{enumerate}
	\item (non-analytic) sentences about material objects are empirically testable but do not admit of conclusive verification while 
	\item (non-analytic) sentences about sense data are \emph{observation} sentences---\-they furnish evidence for other sentences and are themselves incorrigible. 
\end{enumerate}
Each of these claims are instances of more fundamental commitments that are independent of Ayer's positivism. Moreover, each stands opposed to fundamental claims in Cook Wilsonian epistemology and philosophy of language, at least as extended and refined by Austin.

The first claim involves a commitment to a \emph{Lockean conception of knowledge}:
\begin{quote}
    I believe that, in practice, most people agree with John Locke that ``the certainty of things existing \emph{in rerum natura}, when we have the testimony of our sense for it, is not only as great as our frame can attain to, but as our condition needs.'' \citep[1]{Ayer:1958kx}
\end{quote}
The Lockean conception of knowledge is opposed to the Cook Wilsonian conception of knowledge as proof. According to Cook Wilson, knowing that P is akin to having a proof that P since a subject only knows that P when he is in a state that is absolutely incompatible with not-P. However, if knowledge only requires as much certainty as our frame can attain to and as our condition needs, then such certainty can, and most certainly will, fall short of proof (as Ayer acknowledges in conceding that material sentences do not admit of conclusive verification.) In this way, this dispute replays key elements of the early modern dispute between Hobbes and Boyle on the epistemic status of experimental philosophy \citep[see][for discussion]{Shapin:1985ad}.

The second claim involves a commitment to \emph{a form of foundationalism} according to which there are a subclass of sentences (observation sentences, in the present instance, sentences about sense data) that can be incorrigibly known to be true. Moreover, these sentences can serve as the basis of an inferential transition to less certain sentences (sentences about material objects) that can nevertheless be known to be true on the basis of the evidence they provide. However, foundationalism, so conceived, conflicts with a fundamental claim in Cook Wilsonian philosophy of language, at least as extended and refined by Austin. 

Suppose that Sid sees a pig in plain view. The pig that Sid sees is a material object, and for Ayer statements about material objects do not admit of conclusive verification. His thought seems to be this. Contrast Sid seeing a pig in plain view with a perfect matching hallucination---Sid seeming to see a pig but where there is no pig to be seen and where the Sid's seeming to see a pig is, at least in this instance, indiscriminable upon reflection from seeing a pig. While the statement ``There's a pig'' is true in the good case, it is false in the bad case. Since from Sid's perspective the bad case is a ringer for the good case, Ayer concludes that the possibility of Sid's mistakenly judging that a pig is before him in the bad case means that he cannot be certain that there is a pig before him in the good case. At most, he can have inconclusive evidence for there being a pig. But there is an incorrigible judgment that Sid can make in both cases, a judgment about how things appear to Sid in his experience. (For Ayer, this a judgment about sense data, but even philosophers who deny that there are sense data can, and do, accept the more general claim.) And this incorrigible knowledge of appearances constitutes the evidence for the truth of material object sentences.

Austin regards this reasoning as simply confused. Ayer is supposing that there is a type of sentence, an observation sentence that represents how things appear in Sid's experience, that can be incorrigibly known to be true by Sid independently of the occasion of his expressing this knowledge.  Against the claim that, independent of an occasion of utterance, there is a sentence about how things appear in Sid's experience that can be incorrigibly known to be true, Austin insists that the truth of a claim is only determined by the standards in play on the occasion of utterance.% : 
% \begin{quote}
%     It seems to be generally realized nowadays that, if you take a bunch of sentences (or propositions, to use the term Ayer prefers) impeccably formulated in some language or other, there can be no question of sorting them out into those that are true and those that are false; for (leaving out of account the so-called `analytic’ sentences) the question of truth and falsehood does not turn only on what a sentence \emph{is}, nor yet on what it \emph{means}, but on, speaking very broadly, the circumstances in which it is uttered. Sentences are not \emph{as such} either true or false. But it is really equally clear, when one comes to think of it, that for much the same reasons there could be no question of picking out from one’s bunch of sentences those that are evidence for others, those that are `testable’, or those that are `incorrigible’. What kind of sentence is uttered as providing evidence for what depends, again, on the circumstance of the particular cases; there is no kind of sentence which \emph{as such} is evidence-providing, just as there is no kind of sentence which \emph{as such} is surprising, or doubtful, or certain, or incorrigible, or true. \citep[111]{Austin:1962lr}
% \end{quote}
If as Austin maintains, a sentence is only true when uttered on an occasion, there could be no sentence, independent of an occasion of utterance, that is true. And if there could be no sentence that is true independent of the occasion of utterance, then no such sentence could be incorrigibly known to be true.

While no sentence can be incorrigibly known to be true independent of an occasion of utterance, that's not to say that there are no occasions of utterance where Sid can speak with certainty. But recognizing that there are occasions where things can be incorrigibly known undermines the thought that what can be incorrigibly known is restricted to reports about how things appear in sense experience.% :
% \begin{quote}
%     \ldots\ it may be said, even if such cautious formulae are not \emph{intrinsically} incorrigible, surely there will be plenty of cases in which what we say by their utterance will \emph{in fact} be incorrigible \ldots\ Well, yes, no doubt this is true. But then exactly the same thing is true of utterances in which quite different forms of word are employed \ldots\ if I watch for some time an animal a few feet in front of me, in a good light, if I prod it perhaps, and sniff, and take note of the noises it makes, I may say, `That’s a pig’; and this too will be `incorrigible’, nothing could be produced that would show that I had made a mistake. \citep[114--5]{Austin:1962lr}
% \end{quote}
If circumstances are propitious, Sid can just know that there is a pig before him by seeing the pig. Seeing the pig and recognizing as a pig the animal that he sees is incompatible with the pig's absence and so tantamount to proof of the pig's presence. So Sid can know there's a pig and can express this knowledge by saying ``There's a pig''. This is not undermined by there being other circumstances and other occasions where the very same sentence could be used to say something false and so fail to express knowledge. That there are other possible circumstances where Sid would speak falsely and fail to express knowledge is consistent with Sid, in the present circumstances, speaking truly and expressing knowledge of a pig before him. (It is on these grounds as well that Austin rejects the accretion.)

There are two related aspects of Austin's emphasis on circumstances or occasions. Austin is drawing attention to facts about Sid's circumstance in seeing the pig and facts about the circumstance of saying that Sid sees the pig. Indeed, Austin is drawing attention to facts about the circumstances of saying that Sid sees the pig as a means of drawing attention to facts about Sid's circumstance in seeing the pig.

First, Austin in drawing attention to Sid's circumstance in seeing the pig is emphasizing the epistemological significance of specific relations among psychological states of a subject and between these and the environment confronted. In the good case, it is because Sid's experience presents him with the pig that he is in a position to know that there is a pig before him. That there are other occasions, perhaps indiscriminable upon reflection from the present occasion, where these relations do not obtain, is irrelevant. It is the presentation of the pig in Sid's perception that makes Sid knowledgeable of the pig. 

Second, the epistemological significance of Sid's encounter with the pig may depend on the specific relations that obtain among his psychological states and between these and the environment, but they depend, in another way, on potentially distinct circumstances, the circumstances of saying that Sid sees the pig. Specifically, what would count as the obtaining of these relations can vary with circumstance. Sid and the scene he confronts, being as they are, may sometimes count as Sid seeing and sometimes not, depending on the point of saying that Sid sees on the specific occasion of utterance. That Sid is knowledgeable of the pig is less a frame of mind than an epistemic status that he may enjoy. Whether he in fact enjoys it depends on the work needed to be authoritative about this subject, and what work would be needed depends on the circumstance of attributing this epistemic status to Sid.

Austin's emphasis on facts about Sid's circumstance in seeing the pig and his emphasis on facts about the circumstance of saying that Sid sees the pig do not pull in different directions. Far from being in tension, a focus on the latter is a means of focusing on the former. To get clearer on what would count as Sid's seeing the pig on a given occasion of saying is to get clearer about which objective aspects of Sid and the scene he confronts are epistemologically relevant.

Sid can know with certainty that there is a pig before him by seeing it in plain view. Relatedly, Sid in knowing that there is a pig before him does not know this on the basis of perceptual evidence.
% \begin{quote}
%     The situation in which I would properly be said to have \emph{evidence} for the statement that some animal is a pig is that, for example, in which the beast itself is not actually on view, but I can see plenty of pig-like marks on the ground outside its retreat. If I find a few buckets of pig-food, that’s a bit more evidence, and the noises and the smell may provide better evidence still. But if the animal then emerges and stands there plainly in view, there is no longer any question of collecting evidence; its coming into view doesn’t provide me with more \emph{evidence} that it’s a pig, I can now just \emph{see} that it is, the question is settled. \citep[115]{Austin:1962lr}
% \end{quote}
So Ayer's was wrong in maintaining that judgements about appearances are evidence for judgments about material objects like pigs. The pig appearing in Sid's perceptual experience is not evidence for there being a pig before him, the pig is merely evident in Sid's seeing it. 

Here we have an application of Austin's \citeyearpar{Austin:1961kl} idea in ``Other Minds'' that there is a contrast between believing and knowing. In the case of belief, one can ask ``Why?'' In the case of knowledge, one can merely inquire about the means by which one came to know by asking ``How?'' In suffering a perfect matching hallucination and mistakenly judging that there is a pig before him, Pia may ask ``Why does Sid believe that?'' And an adequate answer may be that it looked to Sid as if there was a pig before him. Looking as if there was a pig before him would be evidence for the perceptual belief. But if Sid just knows that there is a pig before him in the propitious circumstance of pig made manifest in his experience, then Pia cannot ask why Sid knows this. And, correlatively, Sid could not adequately answer her by citing as a evidence that it looked to him as if there was a pig before him. 

The contrast that Austin draws between believing and knowing supports, in this way, the Cook Wilsonian opposition to the Lockean conception of knowledge. Evidence comes in degrees and pertains to belief, not knowledge, and so knowledge could not be as much certainty as our frame can attain and as our condition needs. Importantly, Austin's contrast does this in a way that connects with anti-hybridism about knowledge. The fundamental difference between believing and knowing precludes the construction of knowledge out of belief that meets further external conditions. The Austinian contrast thus supports and articulates in a novel way Prichard's insistence that knowledge and belief differ in kind.% :
% \begin{quote}
%     Knowing is absolutely different from what is called indifferently believing or being convinced or being persuaded or having an opinion or thinking, in the sense in which we oppose thinking to knowing, as when we say `I think so but am not sure'. Knowing is not something which differs from being convinced by a difference of degree of something such as a feeling of confidence \ldots\ Knowing and believing differ in kind as do desiring and feeling, or as do a red colour and a blue colour. \citep[87]{Prichard:1950tg}
% \end{quote} 

We are now in a position to see how Austin's emphasis on facts about the perceiver's circumstances highlights the emerging need for an anti-hybridist conception of perception. Ayer postulates appearances that can obtain independently of the material objects they are taken to present. Perception couldn't be appearance in Ayer's sense that meets further external conditions, if perception can, on occasion, afford proof about our external environment. After all, according to Ayer, only judgments about appearances can be incorrigibly known. Judgments about the material environment can only be inconclusively verified on the basis of appearances. If explaining perception in terms of appearances that can obtain independently of the material object they are taken to present is committed to a Lockean epistemology, then so much the worse for hybridism \citep[see][for a contemporary development of this negative thought]{Putnam:1994kx}. 

There is, however, a more positive thought at work here. Nothing short of Sid's encounter with a pig in sight could make Sid knowledgeable of the pig if this is akin to the availability of proof. It is the presentation of the pig as an object of awareness in perceptual experience, an object whose existence is incompatible with there not being a pig, that makes Sid knowledgeable. The relation to the object of perception that makes a subject knowledgeable of that object simply couldn't be present in a case of hallucination. This is at the very least in tension with the idea that the subject could be so related in part by undergoing an appearance that can obtain independently of the material object that it is taken to present. The Cook Wilsonian conception of knowledge as proof requires an anti-hybridist conception of perception if perception is to make the subject knowledgeable of a mind-independent subject matter. 

Anti-hybridism or anti-conjunctivism about perception is a thesis about the nature of perception---that perception cannot be reductively explained in terms of a hybrid state consisting of an internal mental component and an external non-mental component. Experiential monism, in contrast, is a thesis about the nature of experience understood as the genus of which perception is a species. According to this doctrine, experience has a unitary nature. Despite being conceptually distinct in this way, the emerging debate reveals a tension between these doctrines, at least when set against a concern for realism. Oxford and Cambridge realists share a conception of knowledge where the objects of knowledge are independent of the act of knowing and a conception of perception where perception makes the subject knowledgeable of its object by affording sensory awareness of it. Cambridge realists, however, further held that the sensory mode of awareness was not distinctive of perception but characterized sense experience more generally. If the sensory mode of awareness characterizes experience generally, and if the arguments from illusion, hallucination, or conflicting appearances lead one to conclude that the objects of awareness are not ordinary material things like pigs, then it would be increasingly difficult to retain a common sense realism according to which Sid's seeing the pig puts him in a position to know that there is a pig before him. It is, perhaps, no accident that Russell's commitment to sense data led him to a representative realism that devolved into a form of phenomenalism. While Austin is not explicitly committed to the denial of experiential monism, he may be implicitly committed to its denial insofar as experiential monism is in tension with the common sense realism that he sought to defend with anti-hybridist conceptions of perception and knowledge. It will take the work of \citet{Hinton:1973js}, specifically his reflections on the semantics and epistemology of perception--illusion disjunctions, to make the denial explicit. Disjunctivists are experiential pluralists. Part of the point of such pluralism is to acknowledge what's distinctive about perception. And according to the present tradition, adequately conceiving of perception requires acknowledging what's distinctive about perceptual experience if it can make us knowledgeable of a world without the mind.

% Both central strands of thought in Cook Wilsonian epistemology and philosophy of language are intertwined in, and form the basis, of Austin's \citeyearpar{Austin:1962lr}, at times, exasperated, criticism of Ayer. The root of the debate is diagnosed as a misconceived concern for \emph{incorrigibility}, and an illusory need to find some sentences that are incorrigibly known to be true which could act as the foundations for all empirical knowledge.

% Austin is certainly right that the root of the debate with \emph{Ayer} is a concern for the incorrigibility of a certain class of sentences. However, it is less clear that this diagnosis applies more generally to other sense-data theorists. For example, \citet{Price:1932fk}, who Austin cites as keeping bad company with Ayer in this regard, simply does not have Ayer's epistemological motivations. Price's concerns are phenomenological---experience manifestly presents objects to us, and his commitment to sense data is a piece of substantive metaphysics that Ayer would reject \citep[see][for discussion]{Burnyeat:1979mv,Martin:2000nx}.

 % Second, in both the good and bad cases there is a judgment that Sid can make with certainty, a judgement about how things appear in his experience.

% section perception (end)

\nocite{Berkeley:1734fk}
\nocite{Berkeley:1744rm}

