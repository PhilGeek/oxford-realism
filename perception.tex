%!TEX TS-program = xelatex 
%!TEX TS-options = -synctex=1 -output-driver="xdv2pdf -q -E"
%!TEX encoding = UTF-8 Unicode
%
%  Oxford Realism
%
%  Created by Mark Eli Kalderon on 2009-08-08.
%

\documentclass[11pt]{article} 

% Definitions
\newcommand\myauthor{Mark Eli Kalderon and Charles Travis} 
\newcommand\mytitle{Oxford Realism}

% Packages
\usepackage{url}
\usepackage{txfonts}

% XeTeX
\usepackage[cm-default]{fontspec}
\usepackage{xltxtra,xunicode}
\defaultfontfeatures{Scale=MatchLowercase,Mapping=tex-text}
\setmainfont{Hoefler Text}
\setsansfont{Gill Sans}
\setmonofont{Inconsolata}

% Bibliography
\usepackage[round]{natbib}

% Title Information
\title{\mytitle}
\author{\myauthor} 
% \date{} % Leave blank for no date, comment out for most recent date

% PDF Stuff
\usepackage[plainpages=false, pdfpagelabels, bookmarksnumbered, backref, pdftitle={\mytitle}, pagebackref, pdfauthor={\myauthor}, xetex]{hyperref}

%%% BEGIN DOCUMENT
\begin{document}

% Title Page
\maketitle

% Layout Settings
\setlength{\parindent}{1em}

% Main Content

% %!TEX root = /Users/markelikalderon/Documents/oxford-realism/oxford.tex
\section{Introduction} % (fold)
\label{sec:introduction}

This is a story of roughly a century of Oxford philosophy told by two outsiders. Neither of us has ever either studied or taught there. Nor are we specially privy to some oral tradition. Our story is based on texts. It is, moreover, a very brief, and very highly selective, story.  We mean to trace the unfolding, across roughly the last century, of one particular line of thought---a sort of anti-idealism, and also a sort of anti-empiricism. By focussing in this way we will, inevitably, omit, or give short shrift to, more than one more than worthwhile Oxford philosopher. We will mention a few counter-currents to the main flow of 20th century Oxford thought. But much must be omitted entirely.

Our story begins with a turn away from idealism. Frege's case against idealism, so far as it exists in print, was made, for the most part, between \citeyear{Frege:1893fv} (in the preface to \emph{Grundgesetze} volume 1) and \citeyear{Frege:1918lq} (in ``Der Gedanke''). Within that same time span, at Oxford, John Cook Wilson, and his student, H.A. Prichard, developed, independently, their own case against idealism (and for what might plausibly be called---and they themselves regarded as---a form of ``realism''). Because of the way in which Cook Wilson left a written legacy it is difficult at best to give exact dates for the various components of this view. But the main ideas were probably in place by 1904, certainly before \citeyear{Prichard:1909yg}, which marked the publication of Prichard’s beautiful study, ``Kant’s Theory of Knowledge''. It is also quite probably seriously misleading to suggest that either Cook Wilson or Prichard produced a uniform corpus from the whole of their career---uniform either in content or in quality. (For Cook Wilson the issue is synchronic, while for Prichard it is diachronic, and, accordingly, somewhat puzzling.) But if we select the brightest spots, we find a view which overlaps with Frege’s at most key points, and which continued to be unfolded in the main lines of thought at Oxford for the rest of the century.

Frege's main brief against idealism (of the sort which was common currency in Frege's time) could be put this way: it placed the scope of experience (or awareness) outside of the scope of judgement. In doing that, it left us nothing to judge about. A central question about perception is: How can it make the world bear on what one is to think---how can it give me what are then my reasons for thinking things one way or another? The idealist answer to that, Frege showed, would have to be, ``It cannot''. What, in Frege's terms, ``belongs to the contents of my consciousness''---what, for its presence needs someone to be aware of it, where, further, that someone must be me---cannot, just in being as it is, be what might be held, truly, to be thus and so. (This is one point Prichard retained throughout his career, and which, late on, he directed against others who he termed ``sense-datum theorists''. It is also a point Cook Wilson directed, around 1904, against Stout. (See below.)) So, in particular, it was crucial to Frege that a thought could not be an idea (``Vorstellung''), in the sense of ``idea'' in which to be one is to belong to someone’s consciousness. The positive sides of these coins are: all there is for us to judge about---all there is which, in being as it is might be a way we could judge it to be---is that environment we all jointly inhabit; to be a thought is, intrinsically, to be sharable and communicable. All these are central points in Cook Wilson's, and Prichard's, Oxford realism. So, as they both held (early in the century), perception must afford awareness of, and relate us to, objects in our cohabited environment.

There is another point which Prichard, at least, shared with Frege. As Prichard put it:
\begin{quote}
	There seems to be no way of distinguishing perception and conception as the apprehension of different realities except as the apprehension of the individual and the universal respectively. Distinguished in this way, the faculty of perception is that in virtue of which we apprehend the individual, and the faculty of conception is that power of reflection in virtue of which a universal is made the explicit object of thought. \citep[44]{Prichard:1909yg}
\end{quote}
Compare Frege:
\begin{quote}
	\noindent A thought always contains something which reaches out beyond the particular case, by means of which it presents this to consciousness as falling under some given generality. (1882: Kernsatz 4)
	
	\noindent But don’t we see that the sun has set? And don’t we also thereby see that this is true? That the sun has set is no object which emits rays which arrive in our eyes, is no visible thing like the sun itself. That the sun has set is recognised as true on the basis of sensory input. (1918: 64)
\end{quote}
For the sun to have set is a way for things to be; that it has set is the way things are according to a certain thought. A way for things to be is a generality, instanced by things being as they are (where the sun has just set). Recognising its instancing is recognising the truth of a certain thought; an exercise of a faculty of thought. By contrast, what instances a way for things to be, what makes for that thought's truth, does not itself have that generality Frege points to in a thought---any more than, on a different level, which Frege calls ``Bedeutung'', what falls under a (first-level) concept might be the sort of thing things fall under. What perception affords is awareness of the sort of thing that instances a way for things to be. Perception's role is thus, for Frege, as for Prichard, to bring the particular, or individual, in view---so as, in a favourable case, to make recognisable its instancing (some of) the ways for things to be it does. The distinction Prichard points to here is as fundamental both to him and to Frege as is, for Frege, the distinction between objects and concepts.

For all this shared ground between Prichard, Cook Wilson, and Frege, there is still a difference in focus. For Frege, the central notion in his critique of idealism is \emph{truth}, or, correlatively, judging (a truth-evaluable stance towards things). The trouble with idealism, for him, is that it leaves no room for judgement. For Cook Wilson and Prichard, the central notion was \emph{knowledge}. The trouble with idealism (all idealism being, Prichard argued, subjective idealism) is that it leaves no room for knowledge. (It is just restating Frege's core point about ideas to say that ideas, or, in Prichard’s terms, appearances, are not things about which one can be knowledgeable: there is nothing to know about them.) And it is with this focus on knowledge that Cook Wilson’s and Prichard’s brief against idealism continued to shape Oxford philosophy throughout the last century.

Cook Wilson’s and Prichard’s rejection of idealism assumed its finished form in the first decade of the last century. It coincided roughly with several others. Frege's, notably, was in full flower in 1893, again in 1897, and then in his masterful case against idealism in \citeyear{Frege:1918lq}. At Cambridge, Moore’s and Russell's revolution began in \citeyear{Moore:1899sd} with Moore’s ``The Nature of Judgement'', and continued with his ``The Refutation of Idealism'' of \citeyear{Moore:1903uo}, and with various papers by Russell (see notably ``The Nature of Truth'' \citeyear{Russell:1906sm}). Russell's focus, as he himself points out, was a bit different from either Moore's or Cook Wilson's and Prichard's. As \citet[42]{Russell:1959fv} puts it, ``I think that Moore was most concerned with the rejection of idealism, while I was most interested in the rejection of monism.'' Specifically, Russell spent a good deal of time campaigning against a ``doctrine of internal relations'', held by Bradley and others. But, as \citet[42]{Russell:1959fv} also said, both he and Moore were concerned to insist on ``the doctrine that fact is in general independent of experience''. Moore's points coincided with Cook Wilson and Prichard at a number of crucial points. He insisted, for example, 
\begin{quote}
	[T]he existence of a table in space is related to my experience of it in precisely the same way as the existence of my own experience is related to my experience of that. \ldots\ if we are aware that the one exists, we are aware in precisely the same sense that the other exists; and if it is true that my experience can exist, even when I do not happen to be aware of its existence, we have exactly the same reason for supposing that the table can do so also. \ldots\ I am as directly aware of the existence of material things in space as of my own sensations; and what I am aware of with regard to each is exactly the same---namely that in one case the material thing, and in the other case my sensation does really exist. (1903: 453) (ref. of ideal. Mind NS v 12 n 48 (Oct 1903) 433-453)
\end{quote}
Though, for all that, one might reasonably find Cook Wilson and Prichard more relentlessly focussed on the structure of perceptual experience and of knowledge.

Russell reports finding it exhilarating to reject idealism:
\begin{quote}
	I felt it, in fact, as a great liberation, as if I had escaped from a hothouse on to a wind-swept headland. I hated the stuffiness involved in supposing that space and time were only in my mind. I liked the starry heavens even better than the moral law, and could not bear Kant’s view that the one I liked best was only a subjective figment. In the first exuberance of liberation, I became a naïve realist and rejoiced in the thought that grass is really green, in spite of the adverse opinions of all philosophers from Locke onwards. I have not been able to retain this pleasing faith in its pristine vigour, but I have never again shut myself up in a subjective prison. \citep[48]{Russell:1959fv}
\end{quote}
This last sentence is half-right. Neither Russell, nor Moore, nor Prichard (by the 1930s) was able to hang onto the anti-idealist insights with which they began. (If Cook Wilson did, then again, he died in 1915.) Indeed, by 1917, when he delivered the lectures \emph{The Philosophy of Logical Atomism}, Russell had again locked himself up in a thoroughly subjective prison, even insisting that, pace Frege, it was a positively good thing that thoughts could never be exactly communicated. If idealism is a doctrine (or set of them) about the cognitive role of ideas, in Frege’s sense of idea (\emph{Vorstellung})---something coeval with awareness of it, and which it took being so-and-so to be aware of---then nothing could be a more idealist view of the relation between thought and its objects, and of the objects of experience, than Russell’s logical atomism of around that year. By the ‘20s, Moore was himself drawn, reluctantly, into sense-datum theory. As for Prichard, though he remained always opposed to what he called ``sense data'', he did come, some time before \citeyear{Prichard:1938ve}, to believe that the objects of sight were things he called ``colours'', which, whatever else they were, were precisely ideas in Frege’s sense. We think there is a systematic reason why philosophers as insightful as these were uniformly unable to hold onto the realism with which, with the century, they began. It is, in brief, that (like Kant, as per the 4th paralogism) they did not have the tools really to resist a form of the argument from illusion. Those tools came only later, with Austin. We will elaborate this point in due course.

One more initial point. In addition to the realism just sketched, Cook Wilson also contributed to Oxford philosophy a new conception of philosophical good faith (certainly new relative to Hume, to Hegel, and to most of the post-Cartesian tradition). It is a conception perhaps better known as later championed by Moore. Cook Wilson expressed it thus:
\begin{quotation}
	\noindent The actual fact is that a philosophical distinction is prima facie more likely to be wrong than what is called a popular distinction, because it is based on a philosophic theory which may be wrong in its ultimate principles. \ldots\ There is a tendency to regard the linguistic distinction as the less trustworthy because it is popular and not due to reflective thought. The truth is the other way. Reflective thought tends to be too abstract, while the experience which has developed the popular distinctions recorded in language is always in contact with the particular facts.
	
	Now it is not uncommon in philosophic criticism that some popular term, when reflected on, presents great difficulties to the philosopher; difficulties which are often due to some false theory of his which is presupposed. The criticism sometimes ends \ldots\ so that \ldots\ any distinctive use of [the term] is supposed to be an illusion, or the meaning of the term may be pronounced to be altogether an illusion. When the philosopher arrives at such a conclusion it too often happens that he is satisfied with this negative result. \ldots\ We ought under such circumstances to inquire how it is, if the given term only means something else, that language ever developed it, and still so obstinately holds to it, and when we believe that we have explained a term away or shown that it is a mere unnecessary way of disguising some other meaning, we ought to put our result to the test by trying to do without the word criticized and seeing what would happen if we everywhere substituted for it what we suppose to be the truer expression. \citep[875]{Cook-Wilson:1926sf}
\end{quotation}
A philosopher's claims must be answerable to something. If they are, say, claims about seeing, there is nothing better to which they may be answerable than the way the verb ``see'' is actually used. This is one way of putting the foundations of what came to be known as ordinary language philosophy---some decades before there was any. This, though is a point about philosophic methodology. It does not yet identify the main focus of 20th century Oxonian interest in language. 

Despite that salient difference in focus between Frege, on the one hand, and Prichard and Cook Wilson on the other---despite the centrality of knowledge in 20th century Oxford’s concerns---we divide the following discussion into three sections in this order: language, knowledge, and perception.

\nocite{Russell:1985lk}

% section introduction (end)

% %!TEX root = /Users/markelikalderon/Documents/oxford-realism/oxford.tex
\section{Language} % (fold)
\label{sec:language}

Unlike Oxford views of knowledge, and of perception, the most significant Oxford views of language are not ones which persisted throughout the century. Rather, with their roots firmly in Cook Wilson, they flowered from the late \'40s until the early \'60s, largely thanks to J. L. Austin, and then more or less disappeared from the scene. It would be an interesting exercise---perhaps largely in sociology---to explain why this is so (though we will not attempt that here). For, we shall suggest, Oxford’s most distinctive views of language were borne mostly of necessity. More specifically, they were (or were seen as) what was necessary in order to keep afloat those very views of knowledge and perception which not only bear the Oxford mark, but, moreover, did persist at Oxford into this millennium. A question which is very hard to answer is how, after the early \'60s, proponents of these last views thought they could get along without those insights Austin found essential. Again, we shall do at best only a little towards answering it.

There were, in the last century, two distinctive Oxford views of language. One is a particular conception of the relation of language to thought (or thoughts); thus, also, a particular conception of truth. The other is, in effect, a methodological strategy. One, one might say, concerns the relation between mind and language, the other is a strategy of minding one’s language. We do not normally attend to the ways our words work, but rather to what we hope to work with them. But, the idea is, in philosophy words can all too easily work to block our view of the phenomena we mean to speak of; clarity as to their workings—how and when they actually apply---often is the best way to see through them to those objects of our study. Both these views are rooted in Cook Wilson, though in somewhat different ways. We begin here with the first.

There is a line of thought in Cook Wilson’s treatment of a notion of a proposition, and its role in logic, which adumbrates a main line in Austin’s view of language, and which, going on texts, may well have influenced it. Cook Wilson was, roughly, a contemporary of Frege. So it is fair to compare the two. Both wrote on logic. On first reading, Cook Wilson---precisely in his concern for the ordinary use of words---may seem to be missing all Frege’s best insights. No doubt he did miss some, though on closer reading perhaps not \emph{quite} so many as first appears. In any event, both agreed in finding a \emph{grammatical} distinction between subject and predicate---a distinction as generated by English or German syntax---of little or no relevance to logic. Frege writes:
\begin{quote}
	Our logic books still drag in much—for example, subject and predicate—that really does not belong to logic. (1897: 60))
\end{quote}
Rejecting that distinction, he gives fundamental importance to another, that between \emph{object} and \emph{concept}. Cook Wilson writes:
\begin{quote}
	The above analysis [of a statement, or proposition] would make the distinction of subject and predicate, one not of words but of what is meant by the verbal expression. We may call this the strict logical analysis, and the distinction of the words of the sentence into `subject words' and `predicative words' may be called the grammatical analysis. (1926: 124)
\end{quote}
Thus, for example, in `That building is the Bodleian', `that building' is the grammatical subject; in `Glass is elastic', `glass' is the grammatical subject. But in the first either `that building' or `the Bodleian' may identify the \emph{logical} subject, depending on just what is being said. In the second, either `glass' or `elastic' may identify the logical subject. \emph{Mutatis mutandis} for logical predicates. In other words, the \emph{sentence} `Glass is elastic' (or any other) may, while meaning just what it does, having precisely the syntax and semantics it does, so while having the same grammatical subject and predicate might have either of two pairs of strict (or true) logical subject and predicate; which is to say (given the relational nature of the notions \emph{subject} and \emph{predicate}) that it may express either of two different propositions. Like Frege, Cook Wilson dismisses the grammatical notion as irrelevant to logic. But he does find some, though it is unclear just what, relevance in the logical notion (since he thinks that, in some sense, ``logic \ldots\ is some study of the nature of our thinking'' (1926: 150).

To a Fregean, two or three things may seem to have gone wrong already. One of these lies in something Cook Wilson stresses about the just-mentioned `logical' distinction. As he puts it:
\begin{quote}
	Subject and predicate mean not the idea or conception of an object, but the object which is said to be an object of the idea or conception. But, while the things called subject and predicate are objects without anything that belongs to our apprehension of them or our mode of conceiving them, the distinction of them as subject and predicate is entirely founded on our subjective apprehension of them, or our opinion about them, and on nothing in their own nature as apart from the fact that they are apprehended or conceived. It may be said that the distinction is not in them, but in their relation to our knowledge or opinion of them, and so not a relation between what they are in themselves apart from their being sometimes apprehended. (1926: 139)
\end{quote}
This, for a start, may seem to portray \emph{logical subject and predicate} as mere psychological notions of a sort which, for Frege, could also have no bearing on logic. (A related issue: Frege’s distinction between \emph{concept} and \emph{object} precisely \emph{is} a distinction between the sorts of things we designate in expressing the thoughts we do.) But the notions of logical subject and predicate need not be read as psychological in any such tendentious sense. They need not be psychological notions in any sense in which Frege’s notion of a thought would not also be psychological. A thought, for Frege, is the content of a certain sort of stance for a thinker to take towards the world. In taking such a stance a thinker would expose himself to risk of error, of a sort succumbed to or avoided merely in the world being as it is (thus an objective stance). The thought which is the content of that stance is what fixes precisely what risk a thinker would thus run; just \emph{when} he would succumb, just how the world may matter to whether he has. Stances towards the world are part of a thinker’s psychology, on a perfectly good use of that term. Being psychological in this sense need not mean that it is a psychological matter what such stances there are to take, and certainly does not mean that it is a psychological matter how such and such stances relate to one another (e.g., which ones stand farther down or up on truth-preserving paths).

Cook Wilson’s logical subjects and predicates need not be psychological in any other way than a Fregean thought would be. For Frege, a thought marks a commitment there is for one to make in exposing himself to risk of error; accordingly (given the sort of error one risks), a way to represent things to be. Differences between thoughts are thus differences in ways there are to represent things; in the commitments there are for one to make. For Cook Wilson, propositions which differ in whether such-and-such constituent is their logical subject are ones which differ in being answers to different sorts of questions; just where answers which so differ thereby differ in what one is committed to in giving them. Such need be no more psychologistic than any other thesis as to how Fregean thoughts are to be counted; as to how one commitment, or exposure to risk of error, is liable to differ from another. Of course, so read, it is a \emph{substantial} thesis. It needs to be made out that the differences in commitment which Cook Wilson finds correspond to different risks to run of error; and, perhaps, that such difference in risk can make the difference to whether one has represented \emph{truly}. But then, this just adumbrates the really important issue to come.

In his very dismissal of the grammatical subject-predicate distinction, as well as in many other contexts, Frege insists:
\begin{quote}
	Thus we will never forget that two different sentences can express the same thought, that as to the content of a sentence, what concerns us is only what can be true or false. (1897: 60)
\end{quote}
One sentence, perhaps, can express many thoughts (each on some occasion). But what concerns Frege here is that many sentences can express \emph{one} thought. As he often stresses, the same thought can be articulated, now this way, now that, so that now this, now that, appears as predicative in it. The same thought can be structured in many different ways out of many different sets of concepts and objects. Intuitively, we can see how we would, in some sense, understand `That building is the Bodleian' differently depending on whether it was an answer to the question what that building is, or an answer to the question which building is the Bodleian. But what we have not seen---and what, it seems, Cook Wilson has done nothing towards showing us---is that \emph{that} difference in understanding makes for different thoughts expressed---or, again, exploiting Frege’s above framework, that such a difference could make any difference to when the thought thus expressed would be true.

Frege’s object--concept distinction falls on one side of another distinction, equally fundamental for him, between sense and `Bedeutung'. One might think of this \emph{Bedeutung}, on Cook Wilson’s lines, as what we speak of, on some understanding of speaking of. But it is not the sort of object of discussion that Cook Wilson has in mind. Rather, it is, so to speak, a distillate from things at the level of sense, notably thoughts, of what matters for the sorts of calculations, or relations, of concern to logic, most notably truth-preservation. Frege begins a discussion of his main essay on the sense-reference distinction by remarking,
\begin{quote}
	The fundamental logical relation is that of an object falling under a concept; all relations between concepts reduce to this. (18920-1895: 25)
\end{quote}
He goes on to observe that, waiving some grammatical niceties, there is considerable justice in the view of extensionalist logicians. Having first explained how attempts to name concepts, or what they name, with expressions like, `the concept \( A \)', or `What the concept \( A \) names' generally misfire, so that, e.g., in saying `The concept \( A \) is (identical with) the concept \( B \)', we end up speaking of a relation between objects when we really mean to be speaking of one between concepts, he goes on to remark:
\begin{quote}
	If we keep all this in mind, we are indeed in a position to say, `What two concept-words denote is the same just in case the associated extensions of the concepts coincide'. And with this, I think, an important concession is made to the extensionalist logicians. (1892-1895: 31)
\end{quote}
If logic is concerned with, as Frege puts it, the laws of being true (\emph{Wahrsein}), then logic is concerned with thoughts, since, as Frege also insists, thoughts just are the things which, in the first instance, are eligible to be true or false (the things which make questions of truth arise). (See 1918: 59-60.) But the business of logic reduces, for most purposes, at least, to operations on the level of \emph{Bedeutung}. The first sentence here is all that is needed, and really all that Cook Wilson demands, to honour his insistence that logic is, in some sense, about thought. The second seems entirely consistent with his views on the role of relations between things as opposed to our manners, on occasion, of apprehending them.

So though, for several reasons, Frege is not prepared to say (or admit to having said) just what a concept is (here see 1904), one can think of what is at the level of \emph{Bedeutung} as including such things as mappings from some range of things to others; as the taking on of such-and-such range of values for such-and-such range of arguments. (Again, we may, with Frege, keep grammatical obstacles in mind.) What corresponds to objects and concepts at the level of sense is, to use one of Frege’s terms for this, modes of presentation of them: ways of thinking of something which bring some Fregean object, or concept, into play. For example, in speaking of fauns as being gambollers, I bring into play, for purposes of calculating truth preservation, among other things, a function from objects to truth-values which takes on the value true for just those objects which, as it happens, gambol. So speaking of being a gamboller is a way of presenting things which brings that concept into play; accordingly, for Frege, a way of presenting it. What there is not at the level of sense, on Frege’s conception of things, is anything corresponding to logical subjects and predicates, or more pertinently, since something would be a logical subject, or predicate, within some given proposition, or something of that form, there is, for Frege, nothing at the level of sense which has logical subjects and predicates. Certainly thoughts do not. Thoughts, for Frege, articulate into elements—being about certain objects, or was for them to be—only relative to an analysis. If we were to decompose a thought so that its elements were being about the Bodleian, and being about being in the Broad, what we would thus have would be, in effect, a mode of presentation of that thought—a way, one among others, of thinking about \emph{it}. We would have a mode of presentation of a mode of presentation of whatever it is, at the level of Bedeutung, that thoughts present (for Frege, a truth-value). If what is to be found at the level of sense always presents something at the level of reference, there is no room for a distinction between logical subject and logical predicate at either of Frege’s levels.

Cook Wilson also has a second level corresponding, in some way, to Frege’s level of \emph{Bedeutung}. It is inhabited by the things we talk about, on an ordinary understanding on which this includes, for example, the Bodleian, glass, being in the Broad, and being elastic, and by `real relations' between them. So it is not quite inhabited by the same things which belong to Frege’s \emph{Bedeutung}. But it might be seen as inhabited by Cook Wilson’s candidates for the things which really matter to the concerns of logic---notably truth-preservation. For he insists that when we say, `That building is the Bodleian', no matter what the grammatical, or even logical, subject may be, what we \emph{speak of} is just that building being the Bodleian. Which, one might well think---and Cook Wilson seems sometimes to think---leaves nothing for truth to turn on but whether that building \emph{is} the Bodleian. Perhaps it is this which leads Cook Wilson to say, of the subject-predicate distinction, no matter how drawn:
\begin{quote}
	It remains to say that the choice of one or other method of formulating the distinction of subject and predicate, in accordance with what seems to be the only rationale of the traditional definition, is a matter of no great moment, for the distinction is of no importance in logic proper, and indeed of no use whatever for the solution of the usual problems of logic. (1926: 124)
\end{quote}
But then, why is there \emph{any} interest in the notions of (strict) logical subject and predicate, at least if one’s concern is, like Frege’s, only with that in the understandings take words to bear to which laws of logic might apply? How can whether such-and-such is the logical subject of one’s statement matter to the error one risks in stating it (or in judging what is thus stated), at least where such error is error as to how things are (or are correctly viewed as being)?

One approach to answering these questions would be as follows. Frege, while admitting that there are all sorts of aspects to the ways in which one would understand the words we in fact speak, allows into sense, in his sense, only what bears on questions of truth. That is why notion corresponding to logical subjects and predicates shows up, for Frege, at the level of sense. The most obvious way to place those notions there would be to show that they \emph{do} bear on truth; that two truth-bearers (proposition, thoughts, statements) which differed \emph{only} in that the logical subject in one was the logical predicate in the other, and vice-versa, might, for all that, differ in when they would be true. Such would require logical subjects and predicates at \emph{Frege’s} level of sense. Such an idea seems to have inspired Austin. His essay, “How To Talk (Some Simple Ways)” (1952) is, in effect, a more refined elaboration of Cook Wilson’s idea; its object (or one of them) is to show that distinctions of this kind do bear on questions of truth; or on whether one is correct as to how things are.

In “How to Talk”, Austin marks two distinctions---two pairs of distinctive features---where Cook Wilson has only one. He distinguishes, first, between `directions of fit', and second, between what he calls `onuses'. The first distinction is illustrated by cases like this: there is a flower, and a battery of kinds of flower it may be. Looking through the chooses, one commits to it being a dahlia, and not, say, an iris; by contrast, one is asked, of an array of flowers, which one is the \emph{dahlia}, and answers, `This one'. In the first case, one fits the flower to a rubric (in Austin’s terms, `cap-fitting'. In the second, one fits a rubric to the flower. Austin also calls the first thing `placing', and the second, casting. (In this presumably exploratory work he is neither parsimonious, nor elegant, with technical vocabulary.) The contrast in onus is made with examples like the following. There is a color sample---a piece of cloth, say. It is perfectly clear how it is colored. The question is whether being so colored is being crimson. (`Can you really call it \emph{crimson} when there is so much blue in it?') Or it is perfectly clear what it would be for something to be (when it would be) crimson; what is in question is whether \emph{this} sample qualifies. (`Doesn’t it have too much blue in it?')

Austin’s two contrasts yield four possible pairs of distinguishing features---of an onus and a direction---and, correspondingly, four different things to be done in saying such-and-such (that flower) to be such-and-such way or kind (a dahlia, say). Complicating his initial model slightly, these four things to be done become what he calls `calling', `exemplifying', `describing' and `classing'. At which point he points to the different considerations that would come into play in holding one or another of these performances to be \emph{mistaken}, or \emph{incorrect}:
\begin{quote}
	If we are accused of wrongly calling 1228 a polygon \ldots\ then we are accused of \emph{abusing language}. \ldots\ In calling 1228 a polygon \ldots\ we modify or stretch the use of our name \ldots\ If on the other hand we are accused of wrongly describing, or of \emph{mis}describing, 1228 as a polygon, we are accused of doing violence to the \emph{facts}. In describing 1228 as a polygon \ldots\ we are simplifying or neglecting the specificity of the item 1228, and we are committing ourselves thereby to a certain view of it. (1952: 147-148)
\end{quote}
Different ways of going wrong, for different combinations of fit and onus, raise the possibility of going wrong in some such combination, in speaking of, say, \emph{this} flower as a \emph{dahlia}, where one would not go wrong in another combination in speaking of precisely that. Depending on the sort of wrongness involved here, this might be the very sort of contrasting pair that Cook Wilson would need in order to bring his logical subjects and predicates into the realm of sense---aspects of the understandings we bestow on words which do bear on questions of truth. So, for example, if it is France, or a piece of iridescent fabric with the red appearing as behind the blue, or a genetically modified `dahlia' which is neon orange, ten feet tall, glows in the dark, eats birds and sometimes small children, etc., then it may not be true to how the thing is---may mislead, or even misinform---to call it, respectively, a polygon, or crimson, or a dahlia. If what you are doing is saying how the \emph{thing} is, then you have chosen at the least very bad terms in which to do it. Whereas if the question is what you \emph{could} call a polygon, or crimson, or a dahlia---what being this these things really \emph{is}---then you can call France a polygon if you ignore enough irregularity, the sample crimson if you ignore the blue sheen the crimson would then be seen through, the flower a dahlia if you do not mind what dahlias might get up to, so long as the DNA is close enough. And, perhaps, there is nothing in the notions \emph{polygon}, \emph{crimson}, \emph{dahlia}, which rules out, absolutely, so viewing things. If to call France a polygon is to take a certain view of France, it being given what France is like, then that may be, at the least, a very bad view to take. Whereas if to allow that polygonal is the sort of thing France just \emph{might} be allowed to be is to take a certain view of being polygonal, that just might not be such a bad view to take of being that.

We are still some distance from making the case that would need making to install logical subjects and predicates (or Austin’s more refined successors to them) within the realm of sense. One would need to make out that very bad views, such as one of that monster as a dahlia, may correspond to representing \emph{falsely}, or at least not truly. That would take some work. But we need not pursue this issue further. For lines of thought such as this one suggest a certain generalisation, which can be shown on independent grounds: whether one speaks truth in saying things to be a certain way, or \emph{a} thing to be a certain way (or of a certain sort) depends on the standards to which one is thus to be held; where these standards depend, not only on what the words you use speak of—just what, simply in and by using them, you are saying to be what—but also on the circumstances in which you speak (on such things as what questions you are to be held responsible for answering in \emph{so} speaking). You spoke of that flower as a dahlia, or of France as a polygon. To what standards of correctness are you thus to be held? What would be required for you to be correct in speaking of that as a dahlia? That question is \emph{not} answered by all said so far as to what you did. This is the generalisation Austin expresses in \emph{Sense and Sensibilia}, in saying:
\begin{quote}
	It seems to be fairly generally realized nowadays that if you just take a bunch of sentences (or propositions, to use the term Ayer prefers) impeccably formulated in some language or other, there can be no question of sorting them out into those that are true and those that are false; for \ldots\ the question of truth and falsehood does not turn only on what a sentence \emph{is}, nor yet on what it \emph{means}, but on, speaking very broadly, the circumstances in which it is uttered. Sentences are not as such either true or false. (1962: 110-11)
\end{quote}
Whether one speaks truth or falsehood in saying that cloth to be crimson, or that fossil a dahlia, depends on the circumstances of one’s so speaking, and on the standards for things being the way in question---the conditions on truth---that then and there apply. So one may, on one occasion, speak truly, and on another falsely, in and by saying the very same thing, in the very same condition, to be crimson, or a dahlia, or and so on \emph{ad infinitum}. Otherwise put, there are various things being crimson, or being a dahlia, might be understood to be; where one speaks either truly or falsely in speaking of something as a dahlia, there is something this is to be understood to be, where that is just one of an indefinite variety of things this might be.

So the idea of logical subjects and predicates had, by mid-century, in Austin’s hands, turned into the idea that there are many things that might be understood by something being some given way---by being a dahlia, or crimson, for example---where, on different such understandings, different ranges of things would count as those of which it was true that they were dahlias, or crimson, or whatever; that a given way (or sort of thing) for things to be, specified no matter how, does not as such pick out any unique range of things as its instances, full stop; but that what counts as instancing it on one way of viewing this is liable not so to count on others. Such is one way in which at least some of Oxford, by mid-century, had built on the foundations Cook Wilson laid in the first decade of the century (or perhaps before). But it would certainly be wrong to suggest that this view of language was ubiquitous in Oxford at mid-century. And, as noted already, it is a curious fact that, by some time in the \'70s, it had more or less died out.

The most significant dissenter, around mid-century, at least, was H. P. Grice. He first broached his counter-view in (Grice, 1961), and then, more fully, in his William James lectures of 1968(?). First we need to note one small corollary of the view just set out. Suppose that, as per that view, words (e.g., `Fauns gambol') \emph{underdetermine} what would be said in using them as meaning what they do (since, as per above, that might be any of indefinitely many distinguishable things). Then there is substantial work for circumstances of a speaking to do---again, as Austin insists in \emph{Sense and Sensibilia}. In those circumstances, there must be something which would be to be understood by, e.g., \emph{gambolling}; and this should be substantial enough to make what was said in that speaking truth-evaluable---gambolling, on the required understanding, must be something fauns either do or fail to. It is always possible in principle, and, Austin thinks, it sometimes occurs in practice, that circumstances are just not up to the job. So you cannot expect to say, `Fauns gambol' just any time you please and thereby say something either true or false. Or if, through kindness of the world, you might expect this with `Fauns gambol', perhaps you will have poorer luck with that strategy for a sentence like `Sid tried to lift his pen', or `Pia did it of her own free will'. It is this corollary of Austin’s view on which Grice focuses.

Grice’s case against Austin is centred on the thought that while, in speaking, we may say things that are either true or false, we may also suggest, or imply, or etc., other things which are either true or false. If I say, ‘Pia became pregnant and married’, I may certainly at least suggest that the first-mentioned preceded the second---though (importantly for Grice) it is possible to arrange my saying this so that I would not. But the fact that I am likely at least to suggest this is compatible with my not actually having \emph{said} it, with my at \emph{most} suggesting it; and certainly compatible with there being nothing in the meaning of `and', or any other feature of the sentence uttered, which concerns temporal order. Grice introduces the technical term `implicate' for all those ways I, or my words, may have related to propositions about temporal order other than stating them.

Now the core idea to be used against Austin is to be: where Austin sees the possibility of \emph{saying} a variety of things in given (unambiguous) words (while meaning what they do), Grice will argue that this variety of things is only implicated, while, in fact, there is some one thing (to be specified) which is what was said. Or rather, this is what Grice needs to argue. He tends, instead, as mentioned, to focus on the corollary, arguing instead that if, in certain circumstances, one would not say, e.g., `Sid tried to lift his pen', this may be, not because what one thus said would not be true, but rather because one would implicate something unwanted. It is not clear that Grice really understood what Austin’s point was. If not, this may be because of what proved to be an unfortunate choice of vocabulary by Austin and Austinians. We will come to that issue shortly. In any case, the idea of implicature is arguable ill-suited for the application it would need for it to touch Austin’s view. The idea to be countered is: a sentence, say, `That painting is crimson', may be used of a given painting, in a given condition, to say different things, some true, some false, where there is no limit, in principle, to the new things new occasions may make available thus to say. The counter would be: these different things are merely implicated. But then, what is implicated, on any such occasion is, on some possible understanding of being crimson, that the painting is crimson. Now what, in addition to \emph{that}, is to be the thing which is \emph{said} throughout all those cases? Surely something to the effect that the painting is crimson. So it is `crimson', whatever that comes to (on some understanding so being), and, moreover, for a given occasion, it is what being crimson is to be understood to be on that occasion. But what is this additional thing which being crimson always comes to throughout? And how is \emph{that} compatible with the different things it would be taken to come to on different occasions?

We just mentioned an (as it proved) unfortunate choice of vocabulary---one of which Grice certainly makes much. This choice is most evident in the second methodological point, begun by Cook Wilson, developed by Austin. The idea was: in philosophy, we need to mind our language. This idea is put most clearly and elegantly by Austin:
\begin{quotation}
	First, words are our tools, and as a minimum, we should use clean tools: we should know what we mean and what we do not. and we must forearm ourselves against the traps that language sets us. Secondly \ldots\ we need to prise them off the world \ldots\ so that we can realize their inadequacies and arbitrariness, and can re-look at the world without blinkers. Thirdly \ldots\ our common stock of words embodies all the distinctions men have found worth drawing, and the connexions they have found worth making, in the lifetimes of many generations: these surely are likely to be more numerous, more sound, since they have stood up to the long test of the survival of the fittest, and more subtle, at least in all ordinary and reasonably practical matters, than any that you or I are likely to think up in our armchairs of an afternoon \ldots
	\ldots\ When we examine what we should say when, what words we should use in what situations, we are looking again not merely at words \ldots\ but also at the realities we use the words to talk about: we are using a sharpened awareness of words to sharpen our perception of, though not as the final arbiter of, the phenomena. (1956-57: 182)
\end{quotation}
We should, we are told, mind our language for several reasons. For one thing, philosophical problems often depend on taking some word in its usual (ordinary, English) sense. Has anyone ever seen a tomato? If that is not in question when it is asked whether what we see are things in our environment, or if it is in question only in some technical sense of ‘see’ (to be specified), then the question is not obviously as interesting as it initially seems to be, and much more work needs to be done to show it to be interesting at all. That one does not see `material objects' was (we thought) meant to be an amazing discovery. Conversely, for another, if philosophers are not to fly off into the empyrean, only to lose their way there, then they need to be held accountable for what they say. Causal relations hold only between mere appearances. Oh, really? So you did not just now fill my glass. Oh, you didn’t mean that by `appearances'? Well, then, what \emph{did} you mean? (This is all too likely to prove to be nothing at all.) Finally, philosophers too often find introducing technical vocabulary, so as for it to make \emph{sense}, an all too easy matter. Seeing the complexities of ordinary vocabulary, and of the task of getting \emph{it} to apply to the world may be sobering. Moreover, it may show us how our \emph{thinking} falls into confusion by failure to note the complexities involved in isolating a phenomenon.

Austin’s advice should, perhaps, have been old saws, but in fact reconceives philosophical good faith, changes what a philosopher could say with a straight face from what this would be taken to be by Hume, or Bradley, or the subject at large in the 18th and 19th centuries, and in some quarters (cf., e.g., Sartre) in the 20th. But here the vexatious vocabulary intrudes. Austin speaks of what we should (would) say when. A natural way of speaking if you want to respect the idea that it is intrinsic to words to equip us to say, or do, different things with them in different circumstances, on different occasions for the doing. But `what we would say' can be read so as to encompass such things as not saying, `What’s the vigorish?' when your neighbour asks to borrow a cup of milk (but perhaps saying it if it is a cup of Scotch), or not saying `That’s just autobiography' to your small niece when she says she wants another biscuit. And this is how Grice is inclined to read it. On the other hand, asking what one would say when can be a way of asking what the words one uses in fact apply to, or for doing what they are in fact applicable---what one would say (as what one would describe a thing, for what one would ask, what sort of greeting or condolence one would convey) in using them (for what they are for in the language). If one is moved primarily by that main view of language, as developed by Austin from Cook Wilson’s seminal idea---that it is not, e.g., English words, but rather their use on an occasion, which determines how they may, or must, be articulated in understanding what they said---then one certainly will read those words `what we should say when' in this last way.

% section language (end)

% %!TEX root = /Users/markelikalderon/Documents/oxford-realism/oxford.tex
\section{Knowledge} % (fold)
\label{sec:knowledge}

The last section traced (somewhat speculatively) the most pregnant part of Austin's view of language, and of thought, to an idea of Cook Wilson's. But if Austin's view was so inspired, it was, plausibly, also inspired by need. Austin's view of language did not long survive him in Oxford itself. It was soon to be supplanted by what is commonly known as ``the Davidsonic boom''. But another idea, central in Cook Wilson, held a central place at Oxford until roughly the end of the century. It is an idea about knowledge which found applications to perception as well. As Austin saw things, that idea requires his view of language and thought in order to be viable. From some time in the '70s on, the dominant view in Oxford seems to have been that the idea about knowledge is perfectly fine with no help from that Austinian view. This section will set out the idea about knowledge and raise the question whether it is really true that no such Austinian help is needed.

The idea about knowledge can be stated simply. To know that \( P \) is no less than to have proof that \( P \) (or, perhaps, for \( P \) to be simply self-evident). A proof that \( P \) is something whose existence is \emph{absolutely} incompatible with things being otherwise than that \( P \). Having proof that \( P \) is, first, having (being entitled to) \emph{complete} certainty as to whether \( P \); and, second, appreciating adequately the proof one has available as the proof it is. So it is appreciating adequately the incompatibility of that which one sees as to how things are with it being otherwise than P. Cook Wilson expresses this idea as follows:
\begin{quote}
	In knowing, we can have nothing to do with the so-called `greater strength' of the evidence on which the opinion is grounded; simply because we know that this `greater strength' of evidence of \( A \)'s being \( B \) is compatible with \( A \)'s not being \( B \) after all. \ldots\ Belief is not knowledge and the man who knows does not believe at all what he knows; he knows it. \citep[100]{Cook-Wilson:1926sf}
\end{quote}
Prichard insists that knowledge is ``certainty'', and vice-versa. Certainty, for Prichard, is not a feeling. (As he insists, one might have any feeling, whether he knew or not.) Rather, it involves standing in a particular way towards the (mind-independent) world. He describes that way as follows:
\begin{quote}
	We should consider what has now become of the objection that our certainty that an \( A \) is \( B \) cannot be knowledge because an \( A \) need not in the real world conform to our certainty by being \( B \). The fact is that it has simply vanished. For now admittedly it is a condition of our being certain that an \( A \) is \( B \), that we know a certain fact in nature, \emph{viz}. that the possession by an \( A \) of a certain characteristic, a, necessitates its having the characteristic of being \( B \), and, knowing this, we cannot even raise the question `Need an \( A \) in nature have the characteristic \( B \)?', because we know that a certain definite characteristic which it has requires it to have that characteristic. \citep[103--104]{Prichard:1950tg}
\end{quote}
So to insist that knowledge is certainty in Prichard's sense is to endorse Cook Wilson's idea. To know that \( A \)s are \( B \) is to have proof: knowledge of some fact of nature which, one appreciates, is incompatible with things being otherwise. In Prichard's terms, one cannot even raise the question whether \( A \)s must be \( B \). At least one cannot intelligibly wonder this. There is no room for one to raise an intelligible doubt. So it is with sapience on Cook Wilson’s and Prichard's views.

The bite in this view begins to show in Cook Wilson's reference, above, to evidence. Can there be knowledge by, or on, evidence? One might think so. Has Sid been drinking? That loopy expression on his face is some evidence that he has. His slurred speech is a bit more. Now he comes close, and we smell his breath. Now we \emph{know} it. What has happened? One story might be: Sid's breath is (perhaps quite a bit) more evidence. Now the evidence has mounted so high, become so strong, that we may correctly take ourselves to \emph{know} that he has been drinking. So, in general, good enough evidence amounts to knowledge. Cook Wilson and Prichard reject this story. On their view, if all we have is evidence, even very strong evidence (but still, something evaluable in terms of strength or weakness), then, for all we have to show that Sid has been drinking, it is at least possible that he has not. It can make sense to ask whether he really has been.  So for all we know, perhaps not. But if one knows that Sid has been drinking, then not: for all he knows, perhaps not. So this is not knowledge. Which is not to say that one cannot come to know that Sid has been drinking by smelling his breath. But where one does this, one is aware of, as Prichard puts it, some fact of nature: Sid could have breath like that only if he had been drinking (his breath is that of one who has imbibed). In which case, that his breath smells thus does not stand to his having been drinking as evidence for this, but rather as proof.

Cook Wilson refers to knowing as a ``frame of mind''. Prichard concurs. One could use the term ``mental state'' here if one allows that whether one is in it depends, \emph{inter alia}, on how he stands towards the world. This last proviso points to something the two take great pains to stress: to see whether you know that P, do not try to examine your mental state (if that is some kind of pyschological, perhaps introspective, enterprise). Rather, turn your attention towards the things to be known---that Sid has been drinking, say, or that there is no largest prime---and see whether you have a proof of that in hand (whether you can see things being as they are to be incompatible with their being otherwise in that respect). If I know that Sid has been drinking, that is because, as I can appreciate, breath like that (at least in this case) can only mean that he has been drinking. To see by any other means whether I know that P would be, as Prichard points out (see 1950: 92-93), self-defeating (the start of an infinite regress). For if knowing were a mental state distinguished by some mark (which I might detect, say, by introspection), this would help me see whether I know that P only if I knew my current state to have, or to lack, that mark. But, on this plan for detecting knowledge, I could see myself to do that only if I could know my current state to have the distinctive mark of knowing that my state has the mark of knowing that P. And so on \emph{ad infinitum}. So if we were to call knowing a mental state, the way to see whether one was in it \emph{in re} P could only be by directing attention to its object, P. This idea, in more general form, has enjoyed a long life at Oxford. (See, e.g., Gareth Evans, ??)

Cook Wilson and Prichard also stress the further point that knowledge is not a particular variety of belief. In Prichard’s words:
\begin{quote}
	Knowing is not something which differs from being convinced by a difference of degree of something such as a feeling of confidence, as being more convinced differs from being less convinced \ldots\ Knowing and believing differ in kind as do desiring and feeling, or as do a red colour and a blue colour. \ldots\ To know is not to have a belief of a special kind, differing from beliefs of other kinds; and no improvement in a belief and no increase in the feeling of conviction which it implies will convert it into knowledge. \ldots\ It is not that there is a general kind of activity, for which the name would have to be thinking, which admits of two kinds, the better of which is knowing and the worse believing. (1932/1950: 87-88)
\end{quote}
Part of the point here is that knowledge is not \emph{analysable} in terms of belief (or, for both thinkers, in terms of anything). It is not as if knowing is believing with such-and-such further features added---thus, some special variety of believing, or of any other (non-factive) way of standing towards it being so (or its being so) that P. Such is now a widely held view, still at Oxford, and well beyond. But Cook Wilson also holds what is now generally seen as a stronger thesis: when you know that P, you do not believe it. (See above.) This is, to say the least, less widely held. It may \emph{seem} to be controverted by obvious facts---e.g., if Sid stands as he does towards Pia being the new dean, then it can be (depending on how he thus stands) that I, knowing that she is, may say, truly, ``Sid knows that Pia is dean'', while you, doubting that Pia could have been chosen, may say, also truly, ``Well, Sid \emph{thinks} that Pia is dean''. Each of us, it seems, states a truth about Sid's condition; truths which hold simultaneously, and, it seems, may hold of the same frame of mind, or mental state. Austin's view of language should make this seem a less convincing case against Cook Wilson's thesis. The thesis may then come to seem more plausible if we first recognise it as one version of disjunctivism---a denial of a certain sort of common factor in standing towards a thought that P as one might stand whether or not that thought is true, and standing towards a fact of its being so that P---and then apply J.M. Hinton's conception of what such a common factor---what would relevantly hold wherever the disjunction ``Sid believes, or he knows, that P'' would need to be. (See Hinton, 1967.) However, for reasons of space, we leave this here as mere suggestion.

There is a further feature of Cook Wilson’s and Prichard’s view. It is one they are at considerable pains to stress. Given their conception of a frame of mind, it seems to them simply to follow from the above conception of knowledge as proof---though \emph{perhaps} there is room to resist the inference. Cook Wilson sets up the inference by considering the possibility that there are two frames of mind---one knowing, the other merely being under the impression of knowing---which were such that if you were in the one, you might be unable to tell that you were in it rather than the other, so that, as he puts it:
\begin{quote}
	\ldots\ the two states of mind in which the man conducts his arguments, the correct and the erroneous one, are quite indistinguishable to the man himself. But if this is so, as the man does not know in the erroneous state of mind, neither can he know in the other state. (1926: 107)
\end{quote}
So a state of knowing,of actually having proof---if there is such a thing at all---cannot be indistinguishable to someone in it from an ``erroneous'' state---one of merely seeming to have proof; nor vice-versa. Prichard puts the conclusion here this way:
\begin{quote}
	We must recognize that whenever we know something we either do, or at least can, by reflecting, directly know that we are knowing it, and that whenever we believe something, we similarly either do or can directly know that we are believing it and not knowing it. (1950: 86)
\end{quote}
He insists on this point far more than just the once. For convenience, we will refer to this point as \emph{the accretion}.

There is no general thought here that if one is in a frame of mind, he can, by reflection, come to see that he is. Nor are Cook Wilson and Prichard endorsing some form of what has come to be known as ``semantic internalism''. The point is, or is meant to be, a quite special one about what knowledge, or what proof, is. The thought would go something like this. Suppose I am in a frame of mind in which I cannot, by reflection come to see (if I do not see already) whether this is one of having proof that P, or whether it is not. (One might plausibly think of this as my being unable to distinguish this from some other conceivable conditions I might be in in which I would not have proof---plausible, but optional for the present argument.) Then that frame of mind cannot be one of my actually having proof in the requisite sense of having proof. For, whatever grounds I may have for taking it that P, even if these are grounds which might, in fact, be incompatible with things being otherwise than P, they cannot be grounds which I appreciate as proving that P---as being incompatible with things being otherwise. For if I did so appreciate them, then I would see that my state could not be one of merely being under the impression that I had proof that P. So the imagined frame of mind is not one of knowing. Now contrapose. The core thought: I am either in the frame of mind, or I am not---I either have proof or I do not; having proof is the sort of thing such that if you do it, then you should be able to see yourself to do so.

At this point, the whole conception of knowledge as proof begins to appear on shaky ground. On this conception, for one thing, could I ever know such a thing as that a pig is in the sty? Conceivably, Dr. Zarco (call him) might build a ringer-pig which one, or I, could not tell, at least by sight, from the real thing. As I now stare at the pig in the pen, can I tell by \emph{mere reflection} that I am not in such a situation? What, in fact, from my present vantage point on the world, allows me to \emph{tell} that I am not in such a situation, but rather in one in which the thing before me is \emph{really} a \emph{genuine} pig? So knowledge by perception---knowing because, e.g., you see it---seems ruled out absolutely. Which is unlikely to leave a viable ``intellectualist'' conception of knowledge, on which mathematics is the paradigm (and more or less exhausts the field). One can be fooled by a bogus proof. I now take myself to have a genuine proof, say, of some proposition of number theory. Perhaps it is genuine. But can I tell whether it is by mere reflection? If there were a flaw in the proof, could I detect that by mere reflection? (And just what counterfactual is this?) If one conceives reflection as Cook Wilson and Prichard appear to, then, perhaps, the answer is ``Yes''. It is within the reach of human reason to detect such flaws. It is, perhaps, within my reach if one neglects limitations of memory, attention, patience, and enough other things which might block my seeing the flaw. But if one so conceives reflection, then, plausibly, mathematical knowledge, at least in a broad enough domain to take in my theorem, just is knowledge by reflection. There remains for all that what I do know and what I do not about number theory. Being able to detect flaws on an overly idealized conception of this will not draw the distinction. What, then, might?

It is time for Austin's entrance. Austin takes over several points from Cook Wilson. (But, we shall see, \emph{not} the accretion.) There is, first, the idea that there is no knowledge by evidence. This comes out in Austin as follows:
\begin{quote}
	The situation in which I would properly be said to have evidence for the statement that some animal is a pig is that, for example, in which the beast itself is not actually on view, but I can see plenty of pig-like marks on the ground outside its retreat. If I find a few buckets of pig-food, that’s a bit more evidence, and the noises and the smell may provide better evidence still. But if the animal then emerges and stands there plainly in view, there is no longer an question of collecting evidence; its coming into view doesn’t provide me with more \emph{evidence} that it’s a pig, I can now just \emph{see} that it is, the question is settled. (1962: 115)
\end{quote}
Evidence is distinguished from proof. Even very good evidence, like the noises and the smell (in the situation Austin envisions) is compatible with it not being so (in Austin’s case) that the animal is a pig. By contrast, there is another sort of thing which (to speak archly) might speak in favor of taking the animal to be a pig; another way in which the world might come to bear for you on the question whether there is a pig before you. It is illustrated, in Austin’s case, by the pig standing there in plain view. In this case, when I see the pig, I can, thereby, see there to be a pig before me; with which ``the question is settled'': the pig’s presence (which, in this case, I can see to be the presence of a pig) is absolutely incompatible with \emph{that} animal failing to be a pig. There is as little room for that as there is for a largest prime, given the proof that there is none. Where, as in this case, I can see a pig to be present, I appreciate what I thus have in hand as proof. So the situation is this: vision affords me awareness of something, the obtaining of which proves that I confront a pig (leaves no doubt as to this); my recognising what I am thus aware of as a pig being before me (my seeing it to be a pig before me) is my appreciating what I have (am thus aware of) as proof. The proof here is short: the pig’s presence proves that a pig is present. (Note the step here from what does not have the form of a proposition to what does.) Anyway, such is a model of knowledge gained through perception. It is one on which such knowledge is not based on evidence (as, on the present conception, no knowledge could be).

Pursuant to this point, Austin echoes Prichard in insisting that knowledge is distinct from belief, or being (even justifiably) very sure. He writes,
\begin{quote}
	Saying `I know \ldots\ is \emph{not} saying, `I have performed a specially striking feat of cognition, superior, in the same scale as believing and being sure, even to being merely quite sure': for there is nothing in that scale superior to being quite sure. (1946/1970: 99)
\end{quote}
What, then, \emph{is} the difference between knowing and believing or being sure? For Cook Wilson and Prichard, these are different ``frames of mind''; where one can tell which he is in by ``reflection''. Austin puts things in somewhat different terms:
\begin{quotation}
	\noindent When I say `I promise', a new plunge is taken: I have not merely announced my intention, but, by using this formula \ldots\ I have bound myself to others \ldots\ Similarly, saying `I know' is taking a new plunge. \ldots\ When I say `I know', I \emph{give others my word}; I \emph{give others my authority for saying} that `S is P'.
	When I have said only that I am sure \ldots\ I am not liable to be rounded on in the same way as when I have said `I know'. I am sure \emph{for my part}, you can take it or leave it \ldots\ that’s your responsibility. But I don’t know `for my part', and when I say `I know' I don't mean you can take it or leave it (though of course you \emph{can} take it or leave it). (1946: 99-100)
\end{quotation}
This particular point in Austin has attracted a large amount of criticism. There seem to be two main complaints. First, the verb `know' seems to have other uses in the first person than that Austin has in mind---e.g., ``It's hard to park near the beach in August''; ``I know, I know''. Second, waiving that point, even if ``I know'' does typically mark a special force attaching to words, ``I know that P'', still, to describe that force, even correctly and in detail, is not yet to tell us what knowledge is---what it is for someone to know something, for a given such thing, under just what conditions of the world it would be true that he did.

For all that, though, Austin may have made a good start on saying what knowledge is, insofar as there is such a thing as saying that. Suppose that there is the use of ``I know'' that Austin has in mind, and someone, Sid, makes that use of it on an occasion. He may have done so correctly or incorrectly. As usual in such matters, one needs to choose his notion of correctness. It might be that one in Sid's position, grasping what he would say speaking as Sid did, might be able so to speak with complete sincerity and honesty. He might thus be perfectly justified in so speaking. That is one notion of correctness. But there may also be a certain position which one must be in in order to use the words (``I know'') for what they are to be used for (on this use); and Sid uses the words while in this position. Then that is another notion of correctness. (Compare: I may say, ``Pigs grunt'', being perfectly justified in taking it that pigs grunt, thus correctly on one notion of correctness, and, further, I may so speak while, in fact, pigs grunt, and thus be correct on that other notion.) What Austin suggests is that to say ``I know that P'', on the use he has in mind, is to claim authority as to whether P; to offer oneself as authoritative on that point. The position one must be in to do this correctly, on our second notion of correctness, is to be, in fact, authoritative as to whether P. Suppose this is right. Now suppose I tell you, ``Sid knows that Pia is at the Dew Drop Inn''. From the account so far, we can extrapolate something I am thus committed to: Sid is in a position to offer himself as an authority on that point (should he care to). We now have in hand what begin to look like materials for a general account of what one says in saying N to know that P (with a bit of feeling for the different uses ``I know'', ``You know'', etc., in fact have)---perhaps not the most elaborate account one might wish for, but anyway an account of the right shape.

What is not yet in view is any particular point in putting things in these terms. For that we need to see Austin's most significant contribution to making Cook Wilson's conception viable. It is most neatly captured here:
\begin{quote}
	It seems to be fairly generally realised nowadays that if you just take a bunch of sentences \ldots\ impeccably formulated in some language or other, there can be no question of sorting them out into those that are true and those that are false; for \ldots\ the question of truth and falsehood does not turn only on what a sentence \emph{is}, nor yet on what it \emph{means}, but on, speaking very broadly, the circumstances in which it is uttered. Sentences are not as such either true or false. But it is really equally clear \ldots\ that for much the same reasons there could be no question of picking out from one’s bunch of sentences those that are evidence for others, those that are `testable', or those that are `incorrigible'. (1962: 110-111)
\end{quote}
So whether A is evidence for B (or it is true to say so), as opposed to being \emph{no} evidence, or as opposed to being proof, depends not just on what A and B are, but on the circumstances of, or for, so saying (or so counting things). So, correspondingly, whether N has proof, or merely has evidence, that P thus depends on circumstance. So, accepting the Cook-Wilsonian conception of knowledge as proof, whether N \emph{knows} that P (or it is true to say so) depends equally on circumstance. The model here should come from Austin's view of language, as per the last section. Is the sky blue? There are various things to be said in speaking of it, and saying, it to be blue, some true, some false. What one would say depends on the circumstances in which he so spoke. So, apart from an occasion for so speaking, the question has no answer, is ill-formed. Where there  is an answer, what it is depends on the occasion for giving it. Now the idea is: the same goes for knowledge. Suppose, with Austin, that we think of knowing that P as a matter of being authoritative on the subject, or, even more Austinianly, as being in a position to offer oneself as an authority. So to say that N knows that P is, at least in the central use, to say that N is in such a position. Now here is the idea, applied to Pia as she watches the (free range) pig emerge from its straw shelter and approach the barbed wire between them. Such are \emph{her} circumstances. Now, does she know that a pig approaches? There are many (possible) occasions on which one might say her to, or not to. For each of these, it would be true to say, on it, that she knows this if (but only if) it is correct to acknowledge her position as authoritative on that subject. For each of these, there is what it then would take to be thus authoritative. For some of these, Pia has what it would then take, so it would be true to say that she knows a pig approaches. For others it would not, so it would not be true to say this. Independent of these truths, and falsehoods, to be stated on occasions, there is no well-formed question as to whether she knows or not, no fact that, occasion-independently (on this use of ``occasion'') she ``really'' knows, or ``really'' does not. Such is knowledge on the view Austin proposes.

Let us apply (sketchily) the central idea here to questions of evidence. We can begin with Sid’s breath. Is this proof that he has been drinking, or merely (some more) evidence? The question, asked just like that, seems embarrassing. There is, after all, some gap in conceptual space between having breath like that and having been drinking. So, when the question is asked like that, there seems \emph{some} possibility that, for all of his breath being as it is, he has not been drinking; which suggests that his breath cannot be \emph{proof}. What might some of the ways be for the inference here to fail? Perhaps you can get breath like that from near-beer, or by kissing a drunk (or enough drunks), or from tasting and spitting, or from strong whisky-flavored gum (a good wheeze, or good for undercover). Sid’s breath is as it is. Those are his circumstances. But there are many occasions for taking it (or not) as evidence (or more) of drinking. Suppose that Sid is a well-known wine taster, and it is reasonable to suppose that he has been practising his profession. Then, perhaps, his breath is no evidence at all that he has been drinking (in the meaning of the act). Or, again, suppose it is unlikely, but not entirely ruled out, that Sid has been chewing that special gum. Then his breath may be evidence, but hardly proof. But suppose there is simply no question of Sid having come by his breath in any such unusual way. Then to smell his breath is to know what he has been up to; his breath is proof. What varies here is our (or one's) circumstances on an occasion for making something of Sid's breath; for taking it as proof, or mere evidence, or as not even that. What varies with that is whether, in those circumstances, it is true to say that Sid's breath is evidence (or etc.), whether it so counts.

Consider now Pia, across the fence from the approaching pig. Does she know that a pig approaches? Hers is not much of an occasion for her to say, either that she does know, or that she does not. That is, she is not, most likely, in circumstances which determine any answer to that question (as, in general, circumstances are always liable to do wherever Austin’s core point applies). She has, as one says, the evidence of her eyes, for whatever that is worth. But does this amount to knowledge? If what she sees is the pig approaching, \emph{if} she can recognise this---if, in those circumstances, she can recognise a pig by sight---so that she can see that a pig approaches, then, trivially, what she sees, in seeing what she does, is proof, and not mere evidence, for her that a pig approaches. So she knows this. If not, not. At best, the ``evidence of her eyes'' is merely evidence. \emph{Are} these conditions satisfied? Whether they \emph{count} as such depends on the occasion for the counting, as per Austin’s core idea. There might, e.g., on some such occasion, be reason to doubt whether Pia can really tell \emph{pigs} from certain other animals, or whether the beast in question might be some sort of monster, porcine on its visible side only, and so on. (A comparison. Suppose the question were whether Pia knows that the approaching pig is a \emph{bísaro}---a particular kind of pig, marked by long rear legs, and large, floppy ears. Can Pia really tell \emph{bísaros} by sight? Did she really get a good view of the hind legs? Or did she only see the front part of the pig? Etc.) In such cases, it might be true to say Pia \emph{not} to know that a \emph{pig} approaches. But Pia has seen pigs before, and on many occasions counts as being able to tell a pig by sight. On some of these occasions, she may be said, truly, to know that a pig approaches. On such occasions, the evidence of her eyes is, for her, not merely evidence; it is proof. Such is a sketchy illustration of Austin’s idea applied to knowledge by perception.

How does Austin's idea apply to the accretion? The idea which moves the accretion is very briefly this. Suppose I cannot tell, on reflection, that I have proof that P. Then, for all I know, I do not have proof. But then I do not know. The response to that idea now takes this form: the question which, on it, I am supposed to answer on reflection, if I know that P is ill-formed; not a question with an answer at all. Again consider Pia and the approaching pig. She stands as she does towards the pig. Pia stands as she does towards the pig. On some occasions for considering her position, this would count as her having proof that a pig approaches. On others it would not. (Compare, again, Pia and Sid's breath.) There is no further fact as to Pia ``really'' having, or ``really'' lacking proof. So what should Pia be able to tell on reflection? Presumably not that, on the occasion of Sid and Zoë discussing her situation, it would be true for \emph{them} to say that she had proof. Why pick that situation, or impose on her the burden of seeing how \emph{their} circumstances would matter to whether she then counted as having proof? Nor, presumably, whether it would be true for her to say, at the moment of her gazing, that she had proof. First, there is likely to be no such thing to be said either truly or falsely at that moment. Second, why pick on that moment, when knowledge is, grammatically, a state---something which persists whether or not you are \emph{talking} about the matter in question. Third, whether or not she were able to say, truly, on her occasion, that she had proof, this would not settle those questions to be settled on other occasions in then so asking. But there is no further question besides such special questions as these. So there is no question as to having proof such that whether one knows that P turns on whether, on reflection, he can answer it. It is not as if, on this account, one may know that P while being ignorant as to whether he has proof. It is that no sense is to be made of what it is that on reflection one is supposed to see (on that idea which motivates the accretion).

Where Pia, as she stares across the fence, counts as knowing that a pig approaches, she counts as appreciating adequately as proof the proof at her disposal. But this may just come to her counting as seeing the pig, and as able, in this situation, to tell a pig, or this one as a pig, at sight. Perhaps there is also some requirement as to her actual convictions \emph{in re} it being a \emph{pig}, which is \emph{approaching}, though, as evidenced in the literature, there is room for dispute as to just what this requirement might be. Here we bracket that discussion.

So Austin's core idea leaves us with Cook Wilson's conception minus the accretion. It also leaves us, in the domain of knowledge, with a very significant form of disjunctivism. To find it, we can begin from an argument not unlike that for the accretion. This argument takes a case where all is well---say, where that pig is approaching Pia, and pairs it with a ringer for it (in fact, some one of indefinitely many different ringers). What makes for a ringer is this: if Pia were in the ringer situation, she would not be able to tell that she was in it rather than in her actual situation (or, more exactly, one in which a pig was approaching). Everything would be, so far as she could tell, without changing that situation, just as it in fact is with the pig approaching. But no pig would be approaching. Though we skip further details, ringers are always conceivable. In the ringer situation Pia would not have proof that a pig was approaching, since none would be. At best (according to the argument) she would have whatever reasons--\emph{nota bene} inconclusive ones---for supposing there to be a pig approaching. Such-and-such (according to the argument) would be her evidence for that, but no more than evidence. Now we shift to the non-ringer (the actual) situation. Here, according to the argument, she would have just the evidence she had in the ringer situation. But if, in this case, she knew that a pig approached, she would have to have something more as well; something which ruled out her being in the ringer situation---that is, which allowed her to distinguish her actual situation from the ringer. By hypothesis, though, there is no such thing. If there were, then the ringer would not be a ringer. So in the actual case she does not know that a pig approaches. So, ringers always being conceivable, knowing such things about the world around one is impossible.

If Austin's central point is correct, then there is more than a little wrong with this argument. First, one cannot suppose that there is such a thing as ``the evidence Pia has'' in the ringer situation. If she \emph{were} in some situation which was a ringer for a pig approaching her, then there would be indefinitely many occasions for discussing her status there. On different of these, there would be different things to be said truly as to what her evidence was. What counted as her evidence on one such occasion for discussing her predicament might not do so on some other. Similarly, in the real situation there is no such thing as ``her evidence'' \emph{tout court}, but only what, on some particular occasion, might be said truly as to what evidence she had. Which leads to a more important second point. In the ringer situation (of course) she could never count as having any more than evidence on any occasion for discussing her plight, since there cannot be proof of what is not the case. But it does not follow that she can never have any more than evidence in the actual situation, nor that what she can have in the actual situation is restricted to what she would have in the ringer situation plus some addition. In the ringer situation, perhaps (depending on how the situation is set up), something uncannily porcine-looking approaches. That a porcine-looking thing approaches can sometimes be evidence, and normally no more, that a pig approaches. In the ringer situation it gives Pia some reason to suppose that a pig approaches, but, of course, no more. In the actual situation a pig does approach. If Pia can see it approaching, and if she can recognise what she thus sees as that (that is, as a pig approaching), then may have as her reason for supposing that a pig approaches (if one chooses to put things so archly) that she \emph{sees} this. And, on some occasions for considering her actual plight, this would count as her reason. On such occasions, it is not as though she still counts as having, as evidence for so supposing, that something porcine-looking approaches. There is, on such an occasion, no room for this to figure in her reasons at all. Nor is there any clear way of evaluating it as evidence, in the terms of evaluation to which evidence is subject. Exactly how strong or weak is it, in the circumstances? Given that she sees the pig, and can see herself to do so, how can it matter that, moreover, the pig actually looks like a pig? Such is part of the point of insisting, with Austin, that what sometimes may be evidence is, other times, not so much as any evidence at all.

Summing up, then, in the actual situation, and on a favourable (but possible) occasion on which to consider Pia’s plight in that situation, Pia’s reasons for supposing that a pig approaches do not consist in the evidence she would have in the ringer case plus something else to rule out her being in that ringer case. Her reasons do not include such ringer-case evidence at all; and, they being what they are, no reason in addition to them is needed to rule out her actual case being the ringer. At which point, the argument as set out collapses. There are, then, two kinds of case: a case in which knowledge is in reach, and, on some occasion for considering it, counts as possessed; and a ringer case in which knowledge is not in reach. The reasons one has for supposing, falsely, in the ringer case something which is so in the first case are not a factor in common to both sorts of case. The first sort of case is thus not a ringer-case plus some addition. Such is one form of the disjunctivism for which Oxford later came to be well-known.

At Oxford, as noted, Austin's view of language and of thought did not long outlive his death. But Cook Wilson's conception of knowledge (most often with the accretion suppressed) continued to have its champions, most notably John McDowell. (See his two landmark essays, (McDowell: 1982, 1995).) McDowell’s main concern in those essays was to resist a picture in which knowledge is a sort of construct out of belief (or some other non-factive condition) plus some additional factors which might obtain or not without the knower’s awareness of this (a view which he refers to as ``the hybrid conception''). His position is thus far very much Cook Wilson's. Further, he resists the argument just canvassed by denying its conception of a common factor between cases of knowing and ringers for them, in line with the conclusion just suggested (though not quite for the same reasons). McDowell, though, does not accept that view of thought and its expression which Austin (mistakenly) characterised as ``fairly generally realised nowadays''---the central point in the above story. It remains a good question how Cook Wilson’s view can be viable without this. McDowell writes,
\begin{quote}
	Whether we like it or not, we have to rely on favours from the world \ldots\ that on occasion it actually is the way it appears to be. But that the world does someone the necessary favour, on a given occasion, of being the way it appears to be is not extra to the person’s standing in the space of reasons. \ldots\ once she has achieved such a standing, she needs no extra help from the world to count as knowing. (1995: 406)
\end{quote}
So if a pig actually is approaching Pia, then the needed favour has been done. For McDowell, no further favours are needed for her to count as \emph{knowing}. She must, of course, but need only, be able to appreciate her situation for, in this respect, what it is. Where a pig approaches, there is something \emph{for} her to appreciate as to what her situation is, which is not there to be appreciated in any situation in which no pig is on the way. So she may stand towards her situation in a way in which she could not in a deceptive case. She may, if all is well, have as her reason for taking a pig to approach that she sees one to. Such is the core of a disjunctivism on the model of Austin's (but without that central idea which, to Austin’s eye, makes the disjunctivism viable. The question now for McDowell is how such disjunctivism can be viable.

Suppose a pig is approaching Pia. So the world has done its favour. Still, on the conception of knowledge which McDowell endorses---one on which ``the unconnected obtaining of [that] fact'' cannot ``have any intelligible bearing on an epistemic postion'' (vide 1995: 403)---Pia must relate to that fact in the right way---one which draws on her cognitive capacities---if she is to know it. So there remain two sorts of case. Pia may, or may not, have proof at her disposal; and, if so, may, or may not, be able to appreciate what is at her disposal as the proof it is. In the case at hand, she may or may not be able to see the pig approaching; and, if she does, then she may, or may not, be able to recognize what she thus sees as what it thus is. Without Austin's idea in place, it is fair to ask: In what cases of viewing does Pia see the pig, and in what does she not---e.g., in what does she merely see a porcine front half of an animal which is approaching? Thompson Clarke (1965) offers principled reasons for finding that question more than just difficult to answer. Then, if she does see the pig, there is the question whether she is able to recognize what she sees as that. Consider all the ways in which she might be viewing a scene (not necessarily this one) where no pig approaches---\emph{inter alia}, all the ways for the world to have failed to do its present favour. For example, there are those situations in which a shaved goat, or in which a tapir, would be approaching. Without Austin's means, we must say: for some of these, if Pia could not distinguish her actual situation from them---perhaps, e.g., if she could not tell pigs from tapirs---then she would not count as able to tell, in her situation, that a pig approaches. For other---perhaps, say, for Dr. Zarco’s miracle mechanical pig---no such conditional holds. What is needed now is some kind of principled, or at least recognizably correct, way of drawing the distinction---of saying when Pia would have done her bit to earn the relevant status in ``the space of reasons''. Austin offers a principled way to reject the questions, and to be unsurprised when the pursuit of an answer leads only to bafflement. McDowell has no such means. Thus, though we may commend him for the conception of knowledge which he offers, and for his demonstrating the unviability of the alternatives, the question remains how he can make Cook Wilson’s conception viable.

% section knowledge (end)


\section{Perception} % (fold)
\label{sec:perception}

A concern for realism motivates a fundamental strand of Oxford reflection on perception. Begin with the realist conception of knowledge.  The question then will be: What must perception be like if we can know something about an object without the mind by seeing it? What must perception be if it can, on occasion, afford us with \emph{proof} concerning a subject matter independent of the mind? The resulting conception of perception is not unlike the conception of perception shared by Cambridge realists such as Moore and Russell. Roughly speaking, perception is conceived to be a fundamental and irreducible sensory mode of awareness of mind-independent objects, a non-propositional mode of awareness that enables those with the appropriate recognitional capacities to have propositional knowledge concerning that subject matter. 

The difference between Oxford and Cambridge realism concerns the extent of this fundamental sensory mode of awareness. Whereas Oxford realists maintained that perception affords us this sensory mode of awareness, Cambridge realists maintained that this mode of awareness has a broader domain. Let experience be the genus of which perception is a species. Cambridge realists maintained that \emph{all} experience, and not just perception, involves this non-propositional sensory mode of awareness. Cambridge realists are thus committed to a kind of \emph{experiential monism} (in Snowdon's \citeyear{Snowdon:2008oz} terminology)---the thesis that experience has a unitary nature. Specifically, all experience, perceptual and non-perceptual alike, involves, as part of its nature, a non-propositional sensory mode of awareness. Even subject to illusion or hallucination, there is something of which one is aware. And with that, they were an application of the argument from illusion, or hallucination, or conflicting appearances away from immaterial sense data and a representative realism that tended, over time, to devolve into a form of phenomenalism.

Framing the discussion is the fundamental realist (or anti-idealist) commitment common to Cook Wilson and Moore---that the objects of knowledge are independent of the act of knowing. Suppose that in seeing the pig Sid is in a position to know various things about it. The pig is the object of Sid's knowledge in the sense that Sid knows something about \emph{it}---that the pig is before Sid, or that the pig is black, say. According to the fundamental realist commitment, the pig is the object of Sid's knowledge only insofar as it exists independently of Sid's knowing. 

This is a thesis about knowledge, not perception. What connects the fundamental realist commitment to perception is a doctrine whose slogan might be---\emph{perception is a form of knowing}. Perception, conceived as a form of knowing, is a sensory mode of awareness that makes the subject \emph{knowledgeable} of its object. In being so aware of an object, the subject is in a position to know certain things about it, depending, of course, on the subject's possession and exercise of the appropriate recognitional capacities in the circumstances of perception. The subject is knowledgeable of the object of perception in the sense that knowledge is \emph{available} to the subject in perceiving the object. Whether such knowledge is in fact ``activated'' (in Williamson's \citeyear{Williamson:1990uq} terminology) depends on the possession and exercise of recognitional capacities appropriate to the occasion.

Suppose, then, that perception is a form of knowing in the sense that it makes the subject knowledgeable of its object. The objects of perception are then at least potential objects of knowledge. If knowledge is always knowledge of a mind-independent subject matter, and the objects of perception are at least potential objects of knowledge, then it follows that the objects of perception are themselves mind-independent and so independent of the act of perceiving. In this way the doctrine that perception is a form of knowing allows the realist conception of knowledge to have implications for how perception is properly conceived in light of it.

Working out the demands of the realist conception of knowledge on the nature of perception was subject to internal and external pressures. 

Internally, the core features of the realist conception of knowledge get differently conceived by different authors, in a process of refinement and extension, and so the demands that conception of knowledge places on the nature of perception are themselves reconceived. Importantly, an independent aspect of Cook Wilson's conception of knowledge, \emph{the accretion}, an aspect endorsed by Prichard and rejected by Austin, turns out to be inconsistent with the idea that perception makes the subject knowledgeable of a mind-independent subject matter. So the development of the realist conception of knowledge involved not merely refinement and extension, but elimination as well.

Externally, Oxford reflection on perception is subject to alien influences, in particular, Cantibrigian and Viennese influences. Thus Price comes to Oxford from Cambridge where he was Moore's student. Paul comes to Oxford from Cambridge as well but studied with Wittgenstein. And Ayer, given Ryle's encouragement, studied for a time with the logical positivists in Vienna. Incorporating the insights and resisting the challenges posed by these alien influences play an important part in the development of philosophy of perception in Oxford.

Cook Wilson never published on perception. The main source of Cook Wilson's \citeyearpar[764--800]{Cook-Wilson:1926sf} views on perception is a letter of July 1904 criticizing Stout's \citeyearpar{Stout:1903zl} ``Primary and Secondary Qualities''. To highlight the connections between his realist conception of knowledge and his views about perception, it is useful to begin, however, with Cook Wilson's \citeyearpar[801--808]{Cook-Wilson:1926sf} earlier letter of January 1904 to Prichard. There Cook Wilson discusses two variants of a fundamental fallacy concerning knowledge or apprehension.

The first variant is the idealist attempt to understand knowledge as an activity. If knowledge is an activity, then in knowing something a subject must \emph{do} something to the object known. But this, Cook Wilson claims, is absurd. The object of knowledge must be independent of the subject's knowing it, if coming to know is to be a discovery: 
\begin{quote}
	You can no more act upon the object by knowing it than you can `please the Dean and Chapter by stroking to dome of St. Paul's'. The man who first discovered the equable curvature meant equidistance from a point didn't supposed that he `produced' the truth---that absolutely contradicts the idea of truth---nor that he changed the nature of the circle or curvature, or of the straight line, or of anything spatial. \citep[802]{Cook-Wilson:1926sf}
\end{quote}

The second variant is the representative realist's attempt to understand knowledge and apprehension in terms representation. Whereas the idealist attempts to explain apprehension in terms of apprehending, the representative realist attempts to explain apprehension in terms of the object apprehended, in the present instance, an idea or some other representation. The problem is that this merely pushes the problem back a level:
\begin{quote}
	The chief fallacy of this is not so much the impossibility of knowing such image is like the object, or that there is any object at all, but that it assumes the very thing it is intended to explain. The image itself has still to be \emph{apprehended} and the difficulty is only repeated. \citep[803]{Cook-Wilson:1926sf}
\end{quote}

In what sense are the fallacies of explaining apprehension in terms of apprehending and in terms of the object of apprehension variants of the same fallacy? The are variants of the same fallacy in that both attempt to \emph{explain} knowledge or apprehension:
\begin{quote}
	Perhaps most fallacies in the theory of knowledge are reduced to the primary one of trying to \emph{explain} the nature of knowledge or apprehending. We cannot \emph{construct knowing}---the act of apprehending---out of any elements. I remember quite early in my philosophic reflection having an instinctive aversion to the very expression `\emph{theory} of knowledge'. I felt the words themselves suggested a fallacy---an utterly fallacious inquiry, though I was not anxious to proclaim <it>. \citep[803]{Cook-Wilson:1926sf}
\end{quote}
This is a clear statement of the anti-hybridism or anti-conjunctivism about knowledge that \citet{McDowell:1982kx} and \citet{Williamson:2000lr} will later defend. So conceived, knowledge is not a hybrid state consisting of an internal, mental state and the satisfaction of some external conditions. Cook Wilson's aversion to the ``theory of knowledge'' is just an aversion to explaining knowledge by constructing it out of elements, and this skepticism will be echoed by Prichard, Ryle, and Austin and in precisely these terms.

Suppose that perception makes the subject knowledgeable of a mind-independent subject matter. Suppose further that the knowledge the subject is in a position to acquire cannot be explained or constructed out of elements. What must perception be like to make us knowledgeable of the environment in that sense? Must perception itself be non-conjunctive? Does Cook Wilson himself endorse anti-hybridism about perception? In his letter to Stout he does defend a conception of perception as the direct apprehension of objects spatially external to the perceiving subject. And in the letter to Prichard he does at one point speak indifferently of knowledge, apprehension, and perception. If the main conclusions of that letter are meant to apply to all three, then Cook Wilson endorses anti-hybridism about knowledge, apprehension, \emph{and perception}. Neither consideration is decisive. More telling, however, is that the variant fallacies of explaining apprehension in terms of apprehending and the object apprehended are echoed in the letter written later that year to Stout on perception and, indeed, form the core of its content. In particular, both idealist and representative realist accounts of perception are criticized in line with the two variant fallacies concerning knowledge or apprehension. Let's consider these in turn.

First, like \citet{Moore:1903uo}, Cook Wilson emphasizes the distinction between the object of perception and the act of perceiving. In perceiving an object, the object appears to the subject, and so the subjective act of perceiving is sometimes described as an \emph{appearance}. Given the distinction between the object perceived and the act of perceiving, an appearance, so understood, is necessarily distinguished from the object. However, Cook Wilson warns against a misleading ``objectification'' of appearing:
\begin{quote}
	But next the \emph{appearance}, though properly the appear\emph{ing} of the object, gets to be looked on as itself an object and the immediate object of consciousness, and being already, as we have seen, distinguished from the object and related to our subjectivity, becomes, so to say, a mere subjective `object'---`appearance' in that sense. And so, as \emph{appearance} of the object, it has now to be represented not as the object but as the phenomenon caused in our consciousness by the object. Thus for the true appearance (=appearing) to us of the \emph{object} is substituted, through the `objectification' of the appearing as appearance, the appearing to us of an \emph{appearance}, the appearing of a phenomenon caused in us by the object.  \citep[796]{Cook-Wilson:1926sf}
\end{quote}

If perceptual appearances are ``the appearing of a phenomenon caused in us by the object'', then it would be impossible for a subject to come to know about the mind-independent object on the basis of its perceptual appearance and hence impossible to discover how things stand with a mind-independent subject matter by perceiving:
\begin{quote}
	It must be observed that the result of this is that there could be no direct perception or consciousness of Reality under any circumstances or any condition of knowing or perceiving: for the whole view is developed entirely from the fact that the object is distinct from our act of knowing it or recognizing it, which distinction must exist in any kind of knowing it or perceiving it. From this error would necessarily result a mere subjective idealism. Reality would become an absolutely unknowable `Thing in Itself', and finally disappear altogether (as with Berkeley) as an hypothesis that we could not possibly justify. \citep[797]{Cook-Wilson:1926sf}
\end{quote}
This straightforwardly parallels the fallacy of explaining apprehension in terms of apprehending. 

Second, Cook Wilson singles out for criticism Stout’s \citeyearpar[144]{Stout:1903zl} representative realism, in particular his claim that the sensations which mediate knowledge of external qualities such as extension do so only in so far as ``they represent, express, or stand for something other than themselves''. The basis of of his criticism involves negative and positive claims about the nature of representation. The negative claim is that nothing is intrinsically representational: ``Nothing has \emph{meaning} in itself'' \citep[770]{Cook-Wilson:1926sf}. The positive claim is put as follows: ``Representation is our subjective act. ... It is \emph{we} who mean'' \citep[770]{Cook-Wilson:1926sf}. According to Cook Wilson, then, representation is personal. It is we who mean. So conceived, representation is something that the subject does. 

How, according to Stout, might the sensation of extension ``represent, express, or stand for'' extension? Plausibly it might in two ways: by resembling extension or by necessarily covarying with the presence of extension. However, the natural relations of mimesis and necessary covariation are \emph{impersonal}---they obtain independently of anything that the subject does. And since they are \emph{symmetric}, this has the surprising consequence that external qualities represent sensations. However, if it is \emph{we} who mean, if representation is something that a subject does, then the natural relations of mimesis and necessary covariation could not be what makes a sensation represent an external quality (let alone what makes an external quality represent a sensation, for plausibly nothing does). These are not two analyses of different notions of representation; at most, mimesis and necessary covariation are merely natural relations that \emph{incline} us to represent things by means of them---they are merely relations that can be exploited by a subject's representational ends:
\begin{quote}
	It is we who make the weeping willow a symbol of sorrow. There may of course be something in the object which prompts us to give it a meaning, e.g., the resemblance of the weeping willow to a human figure bowed over in the attitude of grief. But the willow in itself can neither `mean' grief, nor `represent' nor `stand for' nor `express' grief. \emph{We} do all that.  \citep[770]{Cook-Wilson:1926sf}
\end{quote}
The weeping willow resembles a human figure bowed over in the attitude of grief. This presents a subject with an opportunity to exploit that resemblance for their own representational ends, at least if they are apprised of that resemblance. In using the willow to represent grief, the subject apprehends the content of that representation. And that, according to Cook Wilson, is precisely what prevents representation from figuring in an explanation of perceptual apprehension. Any such explanation would be circular and, hence, no explanation at all. This straightforwardly parallels the fallacy of explaining apprehension in terms of the object of apprehension, an idea or representation more generally.

Thus Cook Wilson's discussion of perception in his letter to Stout, parallels his discussion of knowledge in his letter to Prichard. In particular the two fallacies of explaining apprehension in terms of apprehending and in terms of the object apprehended (a representation) arise in the perceptual case as well. This raises the question whether in the perceptual case these fallacies are variants of the fundamental fallacy of trying to \emph{explain} perception in more fundamental terms. Just as knowledge cannot be explained in terms of belief that meets further  external conditions, perhaps perception cannot be explained in terms of, say, experience or appearance that meets further external conditions. Cook Wilson expresses his skepticism about such explanations in the case of knowledge by denying that there is any such thing as a theory of knowledge. Farquharson in the postscript to \emph{Statement and Inference} reports a similar attitude in the perceptual case: ``He came to think of a theory of Perception as philosophically preposterous'' \citep[882]{Cook-Wilson:1926sf}. 

The evidence is not decisive. However, even if we were convinced that Cook Wilson accepted an anti-hybridist conception of perception, we would remain unclear why the realist conception of knowledge requires this. A reason begins to emerge with Prichard's case \emph{against} the idea that perception is a form of knowing. While Prichard opposes the doctrine that links the realist conception of knowledge with the nature of perception, his discussion reveals some of what is required if one were to retain the doctrine distinctive of twentieth century realists that perception makes us knowledgeable of a mind-independent subject matter.

Cook Wilson provides neither a theory of perception nor of the nature of appearances. However, Prichard's \citeyearpar{Prichard:1906gf,Prichard:1909yg} theory of appearing builds on some of Cook Wilson's insights. Following Cook Wilson, Prichard holds that the object of perception, like the object of knowledge, must be independent of the act of perceiving, and that an appearance is properly understood as an appearing of a mind-independent object to the perceiving subject. \citet{Prichard:1909yg} thus opposes any conception of appearance, such as Kant's \citeyearpar{Kant1781Critique-of-Pur}, where appearances are states of a subject produced by external objects. (For criticism see Price \citeyear{Price:1932fk}; the theory of appearing is subsequently defended by Alston \citeyear{Alston:1993zl}, Chisholm \citeyear{Chisholm:1950rj}, and Langsam \citeyear{Langsam:1997md}) However, from at least since ``Seeing Movement'' written in 1921, Prichard abandons the theory of appearing. Specifically, he comes to deny that the objects of perception are mind-independent objects located in space, coming to favor, instead, a Berkelean conception of perception where the objects of perception depend on our perceptual experience of them. At the heart of this change of mind is a doubt about whether perception could be a form of knowing.

The central argument occurs in Prichard's \citeyearpar{Prichard:1938ve} ``Sense Datum Fallacy''. His main target is the sense data theory of Cambridge realists such as Moore and Russell. Like their Oxford counterparts, the Cambridge realists held that the object of knowledge is independent of the act of knowing, and that perception is a form of knowing. Cambridge realism departs from Oxford realism in its adherence to a further thesis. Cambridge realists held, in addition, that there is something of which a subject is aware in undergoing sense experience whether perceiving or no. According to the theories of \citet{Moore:1953nx}, \citet{Russell:1912uq}, and \citet{Price:1932fk}, sense data are whatever we are aware of in sense experience. So understood, sense data just are whatever entities that play this epistemic role. This characterization of sense data is \emph{neutral} in the sense that it assumes nothing about the substantive nature of objects that play this epistemic role. Further argument is required to establish substantive claims about the nature of sense data. We have already noted how the sense data theory is committed to an experiential monism---all experience, perceptual and non-perceptual alike, involves, as part of its nature, a non-propositional sensory mode of awareness. A further commitment is presently important. For so conceived, sense data are objects whose substantive nature is open to investigation independent of our acts of awareness of them. It is this consequence of the conjunction of the realist conception of knowledge, the conception of perception as a form of knowing, and the sense data theory that is Prichard's primary target. And Prichard's central thought is that perception could not make one knowledgeable of its object, since the object of perception depends on the subject's experience of it in a way that the object of knowledge could not.

Much of Prichard's case is a variant of Berkeley's \citeyearpar{Berkeley:1734fk,Berkeley:1734zp} critique of \citet{Locke1690An-Essay-Concer}. However, two arguments go beyond the familiar Berkelean critique. The first derives from a peculiar feature of the Cook Wilsonian conception of knowledge, the accretion, and the second is explicitly derived from \citet{Paul:1936kd}. Both present important morals for Oxford realism. The moral of the first argument is that the accretion must be abandoned if Oxford realism is to be sustained. The moral of the second argument is that the realist conception of knowledge and the conception of perception as a form of knowing requires abandoning the Cambridge realist's commitment to experiential monism (though it will take the work of \citet{Austin:1962lr} and \citet{Hinton:1973js} to begin to vindicate this).

The first argument can seem like a variant of the argument from illusion (though it really has a very different character): 
\begin{quote}
	\ldots\ if perceiving were a kind of knowing, mistakes about what we perceive would be impossible, and yet they are constantly being made, since at any rate in the cases of seeing and feeling or touching we are almost always in a state of thinking that what we are perceiving are various bodies, although we need only to reflect to discover that in this we are mistaken. \citep[11]{Prichard:1938ve}
\end{quote}
The passage is frustrating in its lack of explicitness. Indeed in the last line Prichard seems to echo Hume’s \citeyearpar[§XII]{Hume:1740lr} contention that it takes the slightest philosophy to show naïve realism to be false. 

Suppose a pig is in plain view of Sid, and Sid can recognize as a pig the animal that he sees. It might seem that what Sid is thus aware is incompatible with there not being a pig before him. In which case, perception affords Sid something akin to proof of a porcine presence. In this way, perception can seem to make the subject knowledgeable of a mind-independent subject matter. Prichard's insight is that this picture is incompatible with a further feature of Cook Wilson's conception of knowledge, \emph{the accretion}. Prichard understands the accretion as follows:
\begin{quote}
	We must recognize that whenever we know something we either do, or at least can, by reflecting, directly know that we are knowing it, and that whenever we believe something, we similarly either do or can directly know that we are believing it and not knowing it. (1950: 86)
\end{quote}
If Sid knows that P, Sid can know upon reflection that he knows that P. And if Sid has some attitude other than knowledge to that proposition, then Sid can know upon reflection that his attitude is something other than knowledge. Knowledge admits of no ringers---a state indiscriminable upon reflection from knowledge just is knowledge. What would it take for perception to make us knowledgeable of a mind-independent subject matter if there are no ringers for knowledge? If Sid's seeing the pig makes him knowledgeable of the pig's presence, then Sid must recognize that what he is aware of in seeing the pig is incompatible with the pig's absence. But is Sid in seeing the pig in a position to recognize that? After all, there are situations indiscriminable upon reflection from seeing a pig that do not involve the pig's presence. Sid's hallucination of the scene would be indiscirminable upon reflection from his perceiving it. If what Sid is aware of in seeing the pig is not discriminable upon reflection from what, if anything, he is aware of in hallucinating the pig, then it could seem that he is not in a position to recognize that what is aware of in seeing the pig is incompatible with the pig's absence. He would lack proof of a pig before him. Since perception admits of ringers, it could not be a source or form of ringerless knowledge.

This argument reveals a tension within the Oxford realism of Cook Wilson and early Prichard. If Cook Wilson and early Prichard were right in claiming that the objects of knowledge are mind-independent objects, and the objects of perception are at least potential objects of knowledge, then these claims can only be sustained by abandoning the accretion. Indeed, it is telling that Austin jettison's just this feature of Cook Wilson's epistemology.

Prichard's second argument derives from \citet{Paul:1936kd}. Arguably it has ancient roots as well. At the very least, it is a variant of Berkeley's interpretation of the \emph{Theatetus} (\emph{Siris} §§ 253, 304-5). On the Berkelean interpretation, the objects of perception are in a perpetual flux of becoming. In perception, every subject is incorrigibly aware of the sensible qualities whose coming and going constitute the flux since every subject is the ``measure'' of what they perceive. Though perception affords us with an incorrigible awareness of its objects, this mode of awareness could not constitute knowledge since knowledge pertains to \emph{being}, not \emph{becoming}. More prosaically, the objects of perception could not have a continuing identity through time, if every feature they manifest is relativized to a perceiver at a time. Nor could the objects of perception be publicly accessible to different perceivers. But this would preclude the objects of perception from being objects of knowledge if knowledge is to have a mind-independent subject matter \citep[see][for further discussion of the Berkelean interpretation]{Burnyeat:1990dp}. Paul's discussion of sense data is of a piece. Paul, and Prichard following him, emphasize our inability to decide key questions about the persistence and publicity of sense data. If sense data are meant to be objects open to investigation independent of our awareness of them, then such questions should be settled by looking to the sense data themselves. But our inability to decide such questions belies this thought. At best, sense data are shadows cast by experiences that can be elicited by suitably affecting the mind. So conceived, open questions about the nature of sense data are resolved not by investigation but by linguistic decision. In this last regard, Paul is clearly influenced by Wittgenstein's discussion of sense data in \emph{The Blue Book}:
\begin{quote}
    Queerly enough, the introduction of this new phraseology has deluded people into thinking that they had discovered new entities, new elements of the structure of the world, as though to say “I believe that there are sense data” were similar to saying “I believe that matter consists of electrons”. \citep[70]{Wittgenstein:1958rr}
\end{quote}

Suppose the central claim here is right---that sense data do not have a substantive nature open to investigation independent of our awareness of them in sense experience. There are at least three potential morals:

\begin{enumerate}
	\item One might claim that sense data constitutively depend on our awareness of them in sense experience. Sense data would be in this regard like Berkelean ideas. Sense data would lack a substantive nature independent of our awareness of them. Though, Ayer, at least, would regard this Berkelean alternative as piece of substantive metaphysics on a par with Moorean sense data. (Though neither deploy the sense-data vocabulary, Berkeley, later Prichard)
	\item One might deny that there are any substantive facts about the nature of sense data that are open to investigation independent of our awareness of them in sense experience. (Wittgenstein, Paul, Ayer)
	\item One might retain the conception of perception, common to Oxford and Cambridge realists, as a sensory mode of awareness that makes one knowledgeable of a mind-independent subject matter by abandoning the fundamental claim of the sense-datum theory---that there is an object of which we are aware whenever we undergo sense experience---and the experiential monism that came in its wake. (Austin, Hinton)
\end{enumerate}

There have been relatively few takers for the Berkelean moral (though see Foster \citeyear{Foster:00ny} for a recent defense). We will set it aside and focus, instead, on the second and third morals, as represented by the work of Ayer and Austin respectively.

In the \emph{Foundations of Empirical Knowledge}, \citet{Ayer:1958kx} takes over from Carnap and the other logical positivists the general idea that there is no substantive metaphysics and that metaphysical disagreements are better understood as practical disagreements about what language or conceptual scheme to adopt. Ayer applies this general idea to sense data and suggests that talk of sense data is just an alternative way of talking about facts that all of us can agree about, namely, facts about appearances. Ayer cites \citet{Paul:1936kd} as an antecedent. However, as previously noted, the most likely proximate influence on Paul is the middle period Wittgenstein and not the logical positivists. Moreover, it is clear that Paul's attitude toward this claim is more ironic than Ayer's:
\begin{quote}
    The important point is whatever we do is not demanded by the nature of objects which we are calling `sense-data', but that we have a choice of different notations for describing observations, the choice being determined only by the greater convenience of one notation, or our personal inclination, or by tossing a coin. \citep[74]{Paul:1936kd}
\end{quote}

Ayer understands the argument from illusion to establish not that there are sense data, distinct from material objects, that are the objects of sensory awareness, if this is to be understood as a substantive metaphysical claim; rather, the argument from illusion highlights the practical need to regiment our perceptual vocabulary. According to Ayer, ``see'', ``perceive'', and their cognates have readings that implicate the existence of the object seen or perceived \emph{and} readings that fail to so implicate. Sense data theorists, as Ayer understands them, simply regiment in favor of the existential reading. The practical need for talk of immaterial sense data arises in the context of an epistemological project:
\begin{quote}
    For since in philosophizing about perception our main object is to analyse the relationship of our sense-experience to the propositions we put forward concerning material things, it is useful for us to have a terminology that enables us to refer to the contents of our experiences independently of the material things they are taken to present. \citep[]{Ayer:1958kx}
\end{quote}

That project involved two central claims:
\begin{enumerate}
	\item (non-analytic) sentences about material objects are empirically testable but do not admit of conclusive verification while 
	\item (non-analytic) sentences about sense data are \emph{observation} sentences---\-they furnish evidence for other sentences and are themselves incorrigible. 
\end{enumerate}
Each of these claims are instances of more fundamental commitments that are independent of Ayer's positivism. Moreover, each stands opposed to fundamental claims in Cook Wilsonian epistemology and philosophy of language, at least as extended and refined by Austin.

The first claim involves a commitment to a \emph{Lockean conception of knowledge}:
\begin{quote}
    I believe that, in practice, most people agree with John Locke that ``the certainty of things existing \emph{in rerum natura}, when we have the testimony of our sense for it, is not only as great as our frame can attain to, but as our condition needs.'' \citep[1]{Ayer:1958kx}
\end{quote}
The Lockean conception of knowledge is opposed to the Cook Wilsonian conception of knowledge as proof. According to Cook Wilson, knowing that P is akin to having a proof that P since a subject only knows that P when he is in a state that is absolutely incompatible with not-P. However, if knowledge only requires as much certainty as our frame can attain to and as our condition needs, then such certainty can, and most certainly will, fall short of proof (as Ayer acknowledges in conceding that material sentences do not admit of conclusive verification.) In this way, this dispute replays key elements of the early modern dispute between Hobbes and Boyle on the epistemic status of experimental philosophy \citep[see][for discussion]{Shapin:1985ad}.

The second claim involves a commitment to \emph{a form of foundationalism} according to which there are a subclass of sentences (observation sentences, in the present instance, sentences about sense data) that can be incorrigibly known to be true. Moreover, these sentences can serve as the basis of an inferential transition to less certain sentences (sentences about material objects) that can nevertheless be known to be true on the basis of the evidence they provide. However, foundationalism, so conceived, conflicts with a fundamental claim in Cook Wilsonian philosophy of language, at least as extended and refined by Austin. 

Suppose that Sid sees a pig in plain view. The pig that Sid sees is a material object, and for Ayer statements about material objects do not admit of conclusive verification. His thought seems to be this. Contrast Sid seeing a pig in plain view with a perfect matching hallucination---Sid seeming to see a pig but where there is no pig to be seen and where the Sid's seeming to see a pig is, at least in this instance, indiscriminable upon reflection from seeing a pig. While the statement ``There's a pig'' is true in the good case, it is false in the bad case. Since from Sid's perspective the bad case is a ringer for the good case, Ayer concludes that the possibility of Sid's mistakenly judging that a pig is before him in the bad case means that he cannot be certain that there is a pig before him in the good case. At most, he can have inconclusive evidence for there being a pig. But there is an incorrigible judgment that Sid can make in both cases, a judgment about how things appear to Sid in his experience. (For Ayer, this a judgment about sense data, but even philosophers who deny that there are sense data can, and do, accept the more general claim.) And this incorrigible knowledge of appearances constitutes the evidence for the truth of material object sentences.

Austin regards this reasoning as simply confused. Ayer is supposing that there is a type of sentence, an observation sentence that represents how things appear in Sid's experience, that can be incorrigibly known to be true by Sid independently of the occasion of his expressing this knowledge.  Against the claim that, independent of an occasion of utterance, there is a sentence about how things appear in Sid's experience that can be incorrigibly known to be true, Austin insists that the truth of a claim is only determined by the standards in play on the occasion of utterance: 
\begin{quote}
    It seems to be generally realized nowadays that, if you take a bunch of sentences (or propositions, to use the term Ayer prefers) impeccably formulated in some language or other, there can be no question of sorting them out into those that are true and those that are false; for (leaving out of account the so-called `analytic’ sentences) the question of truth and falsehood does not turn only on what a sentence \emph{is}, nor yet on what it \emph{means}, but on, speaking very broadly, the circumstances in which it is uttered. Sentences are not \emph{as such} either true or false. But it is really equally clear, when one comes to think of it, that for much the same reasons there could be no question of picking out from one’s bunch of sentences those that are evidence for others, those that are `testable’, or those that are `incorrigible’. What kind of sentence is uttered as providing evidence for what depends, again, on the circumstance of the particular cases; there is no kind of sentence which \emph{as such} is evidence-providing, just as there is no kind of sentence which \emph{as such} is surprising, or doubtful, or certain, or incorrigible, or true. \citep[111]{Austin:1962lr}
\end{quote}
If as Austin maintains, a sentence is only true when uttered on an occasion, there could be no sentence, independent of an occasion of utterance, that is true. And if there could be no sentence that is true independent of the occasion of utterance, then no such sentence could be incorrigibly known to be true.

While no sentence can be incorrigibly known to be true independent of an occasion of utterance, that's not to say that there are no occasions of utterance where Sid can speak with certainty. But recognizing that there are occasions where things can be incorrigibly known undermines the thought that what can be incorrigibly known is restricted to reports about how things appear in sense experience:
\begin{quote}
    \ldots\ it may be said, even if such cautious formulae are not \emph{intrinsically} incorrigible, surely there will be plenty of cases in which what we say by their utterance will \emph{in fact} be incorrigible \ldots\ Well, yes, no doubt this is true. But then exactly the same thing is true of utterances in which quite different forms of word are employed \ldots\ if I watch for some time an animal a few feet in front of me, in a good light, if I prod it perhaps, and sniff, and take note of the noises it makes, I may say, `That’s a pig’; and this too will be `incorrigible’, nothing could be produced that would show that I had made a mistake. \citep[114--5]{Austin:1962lr}
\end{quote}
If circumstances are propitious, Sid can just know that there is a pig before him by seeing the pig. Seeing the pig and recognizing as a pig the animal that he sees is incompatible with the pig's absence and so tantamount to proof of the pig's presence. So Sid can know there's a pig and can express this knowledge by saying ``There's a pig''. This is not undermined by there being other circumstances and other occasions where the very same sentence could be used to say something false and so fail to express knowledge. That there are other possible circumstances where Sid would speak falsely and fail to express knowledge is consistent with Sid, in the present circumstances, speaking truly and expressing knowledge of a pig before him. (It is on these grounds as well that Austin rejects the accretion.)

There are two related aspects of Austin's emphasis on circumstances or occasions. Austin is drawing attention to facts about Sid's circumstance in seeing the pig and facts about the circumstance of saying that Sid sees the pig. Indeed, Austin is drawing attention to facts about the circumstances of saying that Sid sees the pig as a means of drawing attention to facts about Sid's circumstance in seeing the pig.

First, Austin in drawing attention to Sid's circumstance in seeing the pig is emphasizing the epistemological significance of specific relations among psychological states of a subject and between these and the environment confronted. In the good case, it is because Sid's experience presents him with the pig that he is in a position to know that there is a pig before him. That there are other occasions, perhaps indiscriminable upon reflection from the present occasion, where these relations do not obtain, is irrelevant. It is the presentation of the pig in Sid's perception that makes Sid knowledgeable of the pig. 

Second, the epistemological significance of Sid's encounter with the pig may depend on the specific relations that obtain among his psychological states and between these and the environment, but they depend, in another way, on potentially distinct circumstances, the circumstances of saying that Sid sees the pig. Specifically, what would count as the obtaining of these relations can vary with circumstance. Sid and the scene he confronts, being as they are, may sometimes count as Sid seeing and sometimes not, depending on the point of saying that Sid sees on the specific occasion of utterance. That Sid is knowledgeable of the pig is less a frame of mind than an epistemic status that he may enjoy. Whether he in fact enjoys it depends on the work needed to be authoritative about this subject, and what work would be needed depends on the circumstance of attributing this epistemic status to Sid.

Austin's emphasis on facts about Sid's circumstance in seeing the pig and his emphasis on facts about the circumstance of saying that Sid sees the pig do not pull in different directions. Far from being in tension, a focus on the latter is a means of focusing on the former. To get clearer on what would count as Sid's seeing the pig on a given occasion of saying is to get clearer about which objective aspects of Sid and the scene he confronts are epistemologically relevant.

Sid can know with certainty that there is a pig before him by seeing it in plain view. Relatedly, Sid in knowing that there is a pig before him does not know this on the basis of perceptual evidence.
\begin{quote}
    The situation in which I would properly be said to have \emph{evidence} for the statement that some animal is a pig is that, for example, in which the beast itself is not actually on view, but I can see plenty of pig-like marks on the ground outside its retreat. If I find a few buckets of pig-food, that’s a bit more evidence, and the noises and the smell may provide better evidence still. But if the animal then emerges and stands there plainly in view, there is no longer any question of collecting evidence; its coming into view doesn’t provide me with more \emph{evidence} that it’s a pig, I can now just \emph{see} that it is, the question is settled. \citep[115]{Austin:1962lr}
\end{quote}
So Ayer's was wrong in maintaining that judgements about appearances are evidence for judgments about material objects like pigs. The pig appearing in Sid's perceptual experience is not evidence for there being a pig before him, the pig is merely evident in Sid's seeing it. 

Here we have an application of Austin's \citeyearpar{Austin:1961kl} idea in ``Other Minds'' that there is a contrast between believing and knowing. In the case of belief, one can ask ``Why?'' In the case of knowledge, one can merely inquire about the means by which one came to know by asking ``How?'' In suffering a perfect matching hallucination and mistakenly judging that there is a pig before him, Pia may ask ``Why does Sid believe that?'' And an adequate answer may be that it looked to Sid as if there was a pig before him. Looking as if there was a pig before him would be evidence for the perceptual belief. But if Sid just knows that there is a pig before him in the propitious circumstance of pig made manifest in his experience, then Pia cannot ask why Sid knows this. And, correlatively, Sid could not adequately answer her by citing as a evidence that it looked to him as if there was a pig before him. 

The contrast that Austin draws between believing and knowing supports, in this way, the Cook Wilsonian opposition to the Lockean conception of knowledge. Evidence comes in degrees and pertains to belief, not knowledge, and so knowledge could not be as much certainty as our frame can attain and as our condition needs. Importantly, Austin's contrast does this in a way that connects with anti-hybridism about knowledge. The fundamental difference between believing and knowing precludes the construction of knowledge out of belief that meets further external conditions. The Austinian contrast thus supports and articulates in a novel way Prichard's insistence that:
\begin{quote}
    Knowing is absolutely different from what is called indifferently believing or being convinced or being persuaded or having an opinion or thinking, in the sense in which we oppose thinking to knowing, as when we say `I think so but am not sure'. Knowing is not something which differs from being convinced by a difference of degree of something such as a feeling of confidence \ldots\ Knowing and believing differ in kind as do desiring and feeling, or as do a red colour and a blue colour. \citep[87]{Prichard:1950tg}
\end{quote} 

We are now in a position to see how Austin's emphasis on facts about the perceiver's circumstances highlights the emerging need for an anti-hybridist conception of perception. Ayer postulates appearances that can obtain independently of the material objects they are taken to present. Perception couldn't be appearance in Ayer's sense that meets further external conditions, if perception can, on occasion, afford proof about our external environment. After all, according to Ayer, only judgments about appearances can be incorrigibly known. Judgments about the material environment can only be inconclusively verified on the basis of appearances. If explaining perception in terms of appearances that can obtain independently of the material object they are taken to present is committed to a Lockean epistemology, then so much the worse for hybridism \citep[see][for a contemporary development of this negative thought]{Putnam:1994kx}. 

There is, however, a more positive thought at work here. The positive thought is that nothing short of Sid's encounter with a pig in sight could make Sid knowledgeable of the pig if this is akin to the availability of proof. It is the presentation of the pig as an object of awareness in perceptual experience, an object whose existence is incompatible with there not being a pig, that makes Sid knowledgeable. The relation to the object of perception that makes a subject knowledgeable of that object simply couldn't be present in a case of hallucination. This is at the very least in tension with the idea that the subject could be so related in part by undergoing an appearance that can obtain independently of the material object that it is taken to present. The Cook Wilsonian conception of knowledge as proof requires an anti-hybridist conception of perception if perception is to make the subject knowledgeable of a mind-independent subject matter. 

Anti-hybridism or anti-conjunctivism about perception is a thesis about the nature of perception---that perception cannot be reductively explained in terms of a hybrid state consisting of an internal mental component and an external non-mental component. Experiential monism, in contrast, is a thesis about the nature of experience understood as the genus of which perception is a species. According to this doctrine, experience has a unitary nature. Despite being conceptually distinct in this way, the emerging debate reveals a tension between these doctrines, at least when set against a concern for realism. Oxford and Cambridge realists share a conception of knowledge where the objects of knowledge are independent of the act of knowing and a conception of perception where perception makes the subject knowledgeable of its object by affording sensory awareness of it. Cambridge realists, however, further held that the sensory mode of awareness was not distinctive of perception but characterized sense experience more generally. If the sensory mode of awareness characterizes experience generally, and if the arguments from illusion, hallucination, or conflicting appearances lead one to conclude that the objects of awareness are not ordinary material things like pigs, then it would be increasingly difficult to retain a common sense realism according to which Sid's seeing the pig puts him in a position to know that there is a pig before him. It is, perhaps, no accident that Russell's commitment to sense data led him to a representative realism that devolved into a form of phenomenalism. While Austin is not explicitly committed to the denial of experiential monism, he may be implicitly committed to its denial insofar as experiential monism is in tension with the common sense realism that he sought to defend with anti-hybridist conceptions of perception and knowledge. It will take the work of \citet{Hinton:1973js}, specifically his reflections on the semantics and epistemology of perception--illusion disjunctions, to make the denial explicit. Disjunctivists are experiential pluralists. Part of the point of such pluralism is to acknowledge what's distinctive about perception. And according to the present tradition, adequately conceiving of perception requires acknowledging what's distinctive about perceptual experience if it can make us knowledgeable of a world without the mind.

% Both central strands of thought in Cook Wilsonian epistemology and philosophy of language are intertwined in, and form the basis, of Austin's \citeyearpar{Austin:1962lr}, at times, exasperated, criticism of Ayer. The root of the debate is diagnosed as a misconceived concern for \emph{incorrigibility}, and an illusory need to find some sentences that are incorrigibly known to be true which could act as the foundations for all empirical knowledge.

% Austin is certainly right that the root of the debate with \emph{Ayer} is a concern for the incorrigibility of a certain class of sentences. However, it is less clear that this diagnosis applies more generally to other sense-data theorists. For example, \citet{Price:1932fk}, who Austin cites as keeping bad company with Ayer in this regard, simply does not have Ayer's epistemological motivations. Price's concerns are phenomenological---experience manifestly presents objects to us, and his commitment to sense data is a piece of substantive metaphysics that Ayer would reject \citep[see][for discussion]{Burnyeat:1979mv,Martin:2000nx}.

 % Second, in both the good and bad cases there is a judgment that Sid can make with certainty, a judgement about how things appear in his experience.

% section perception (end)

\nocite{Berkeley:1734fk}
\nocite{Berkeley:1744rm}

% (end)

\bibliographystyle{plainnat} 
\bibliography{Philosophy}

\end{document}
