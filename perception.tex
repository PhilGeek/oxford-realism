%!TEX TS-program = xelatex 
%!TEX TS-options = -synctex=1 -output-driver="xdv2pdf -q -E"
%!TEX encoding = UTF-8 Unicode
%
%  Oxford Realism
%
%  Created by Mark Eli Kalderon on 2009-08-08.
%

\documentclass[11pt]{article} 

% Definitions
\newcommand\myauthor{Mark Eli Kalderon and Charles Travis} 
\newcommand\mytitle{Oxford Realism}

% Packages
\usepackage{url}
\usepackage{txfonts}

% XeTeX
\usepackage[cm-default]{fontspec}
\usepackage{xltxtra,xunicode}
\defaultfontfeatures{Scale=MatchLowercase,Mapping=tex-text}
\setmainfont{Hoefler Text}
\setsansfont{Gill Sans}
\setmonofont{Inconsolata}

% Bibliography
\usepackage[round]{natbib}

% Title Information
\title{\mytitle}
\author{\myauthor} 
% \date{} % Leave blank for no date, comment out for most recent date

% PDF Stuff
\usepackage[plainpages=false, pdfpagelabels, bookmarksnumbered, backref, pdftitle={\mytitle}, pagebackref, pdfauthor={\myauthor}, xetex]{hyperref}

%%% BEGIN DOCUMENT
\begin{document}

% Title Page
\maketitle

% Layout Settings
\setlength{\parindent}{1em}

% Main Content

% \section{Introduction} % (fold)
\label{sec:introduction}

This is a story of roughly a century of Oxford philosophy told by two outsiders. Neither of us has ever either studied or taught there. Nor are we specially privy to some oral tradition. Our story is based on texts. It is, moreover, a very brief, and very highly selective, story.  We mean to trace the unfolding, across roughly the last century, of one particular line of thought---a sort of anti-idealism, and also a sort of anti-empiricism. By focussing in this way we will, inevitably, omit, or give short shrift to, more than one more than worthwhile Oxford philosopher. We will mention a few counter-currents to the main flow of 20th century Oxford thought. But much must be omitted entirely.

Our story begins with a turn away from idealism. Frege's case against idealism, so far as it exists in print, was made, for the most part, between 1893 (in the preface to \emph{Grundgesetze} volume 1) and 1918 (in ``Der Gedanke''). Within that same time span, at Oxford, John Cook Wilson, and his student, H. A. Prichard, developed, independently, their own case against idealism (and for what might plausibly be called---and they themselves regarded as---a form of ``realism''). Because of the way in which Cook Wilson left a written legacy it is difficult at best to give exact dates for the various components of this view. But the main ideas were probably in place by 1904, certainly before 1909, which marked the publication of Prichard’s beautiful study, ``Kant’s Theory of Knowledge''. It is also quite probably seriously misleading to suggest that either Cook Wilson or Prichard produced a uniform corpus from the whole of their career---uniform either in content or in quality. (For Cook Wilson the issue is synchronic, while for Prichard it is diachronic, and, accordingly, somewhat puzzling.) But if we select the brightest spots, we find a view which overlaps with Frege’s at most key points, and which continued to be unfolded in the main lines of thought at Oxford for the rest of the century.

Frege's main brief against idealism (of the sort which was common currency in Frege's time) could be put this way: it placed the scope of experience (or awareness) outside of the scope of judgement. In doing that, it left us nothing to judge about. A central question about perception is: How can it make the world bear on what one is to think---how can it give me what are then my reasons for thinking things one way or another? The idealist answer to that, Frege showed, would have to be, ``It cannot''. What, in Frege's terms, ``belongs to the contents of my consciousness''---what, for its presence needs someone to be aware of it, where, further, that someone must be me---cannot, just in being as it is, be what might be held, truly, to be thus and so. (This is one point Prichard retained throughout his career, and which, late on, he directed against others who he termed ``sense-datum theorists''. It is also a point Cook Wilson directed, around 1904, against Stout. (See below.)) So, in particular, it was crucial to Frege that a thought could not be an idea (``Vorstellung''), in the sense of ``idea'' in which to be one is to belong to someone’s consciousness. The positive sides of these coins are: all there is for us to judge about---all there is which, in being as it is might be a way we could judge it to be---is that environment we all jointly inhabit; to be a thought is, intrinsically, to be sharable and communicable. All these are central points in Cook Wilson's, and Prichard's, Oxford realism. So, as they both held (early in the century), perception must afford awareness of, and relate us to, objects in our cohabited environment.

There is another point which Prichard, at least, shared with Frege. As Prichard put it:
\begin{quote}
	There seems to be no way of distinguishing perception and conception as the apprehension of different realities except as the apprehension of the individual and the universal respectively. Distinguished in this way, the faculty of perception is that in virtue of which we apprehend the individual, and the faculty of conception is that power of reflection in virtue of which a universal is made the explicit object of thought. (1909: 44)
\end{quote}
Compare Frege:
\begin{quote}
	\noindent A thought always contains something which reaches out beyond the particular case, by means of which it presents this to consciousness as falling under some given generality. (1882: Kernsatz 4)
	
	\noindent But don’t we see that the sun has set? And don’t we also thereby see that this is true? That the sun has set is no object which emits rays which arrive in our eyes, is no visible thing like the sun itself. That the sun has set is recognised as true on the basis of sensory input. (1918: 64)
\end{quote}
For the sun to have set is a way for things to be; that it has set is the way things are according to a certain thought. A way for things to be is a generality, instanced by things being as they are (where the sun has just set). Recognising its instancing is recognising the truth of a certain thought; an exercise of a faculty of thought. By contrast, what instances a way for things to be, what makes for that thought's truth, does not itself have that generality Frege points to in a thought---any more than, on a different level, which Frege calls ``Bedeutung'', what falls under a (first-level) concept might be the sort of thing things fall under. What perception affords is awareness of the sort of thing that instances a way for things to be. Perception's role is thus, for Frege, as for Prichard, to bring the particular, or individual, in view---so as, in a favourable case, to make recognisable its instancing (some of) the ways for things to be it does. The distinction Prichard points to here is as fundamental both to him and to Frege as is, for Frege, the distinction between objects and concepts.

For all this shared ground between Prichard, Cook Wilson, and Frege, there is still a difference in focus. For Frege, the central notion in his critique of idealism is \emph{truth}, or, correlatively, judging (a truth-evaluable stance towards things). The trouble with idealism, for him, is that it leaves no room for judgement. For Cook Wilson and Prichard, the central notion was \emph{knowledge}. The trouble with idealism (all idealism being, Prichard argued, subjective idealism) is that it leaves no room for knowledge. (It is just restating Frege's core point about ideas to say that ideas, or, in Prichard’s terms, appearances, are not things about which one can be knowledgeable: there is nothing to know about them.) And it is with this focus on knowledge that Cook Wilson’s and Prichard’s brief against idealism continued to shape Oxford philosophy throughout the last century.

Cook Wilson’s and Prichard’s rejection of idealism assumed its finished form in the first decade of the last century. It coincided roughly with several others. Frege's, notably, was in full flower in 1893, again in 1897, and then in his masterful case against idealism in 1918. At Cambridge, Moore’s and Russell's revolution began in 1899 with Moore’s ``The Nature of Judgement'', and continued with his ``The Refutation of Idealism'' of 1903, and with various papers by Russell (See notably ``Joachim On Truth'', Mind, Oct. 1906). Russell's focus, as he himself points out, was a bit different from either Moore's or Cook Wilson's and Prichard's. As Russell puts it, ``I think that Moore was most concerned with the rejection of idealism, while I was most interested in the rejection of monism.'' (1959: 42) Specifically, Russell spent a good deal of time campaigning against a ``doctrine of internal relations'', held by Bradley and others. But, as Russell also said, both he and Moore were concerned to insist on ``the doctrine that fact is in general independent of experience''. (Ibid) Moore's points coincided with Cook Wilson and Prichard at a number of crucial points. He insisted, for example, 
\begin{quote}
	[T]he existence of a table in space is related to my experience of it in precisely the same way as the existence of my own experience is related to my experience of that. \ldots\ if we are aware that the one exists, we are aware in precisely the same sense that the other exists; and if it is true that my experience can exist, even when I do not happen to be aware of its existence, we have exactly the same reason for supposing that the table can do so also. \ldots\ I am as directly aware of the existence of material things in space as of my own sensations; and what I am aware of with regard to each is exactly the same---namely that in one case the material thing, and in the other case my sensation does really exist. (1903: 453) (ref. of ideal. Mind NS v 12 n 48 (Oct 1903) 433-453)
\end{quote}
Though, for all that, one might reasonably find Cook Wilson and Prichard more relentlessly focussed on the structure of perceptual experience and of knowledge.

Russell reports finding it exhilarating to reject idealism:
\begin{quote}
	I felt it, in fact, as a great liberation, as if I had escaped from a hothouse on to a wind-swept headland. I hated the stuffiness involved in supposing that space and time were only in my mind. I liked the starry heavens even better than the moral law, and could not bear Kant’s view that the one I liked best was only a subjective figment. In the first exuberance of liberation, I became a naïve realist and rejoiced in the thought that grass is really green, in spite of the adverse opinions of all philosophers from Locke onwards. I have not been able to retain this pleasing faith in its pristine vigour, but I have never again shut myself up in a subjective prison. (1959: 48)
\end{quote}
This last sentence is half-right. Neither Russell, nor Moore, nor Prichard (by the 1930s) was able to hang onto the anti-idealist insights with which they began. (If Cook Wilson did, then again, he died in 1915.) Indeed, by 1917 Russell had again locked himself up in a thoroughly subjective prison, even insisting that, pace Frege, it was a positively good thing that thoughts could never be exactly communicated. If idealism is a doctrine (or set of them) about the cognitive role of ideas, in Frege’s sense of idea (\emph{Vorstellung})---something coeval with awareness of it, and which it took being so-and-so to be aware of---then nothing could be a more idealist view of the relation between thought and its objects, and of the objects of experience, than Russell’s logical atomism of around that year. By the ‘20s, Moore was himself drawn, reluctantly, into sense-datum theory. As for Prichard, though he remained always opposed to what he called ``sense data'', he did come, some time before 1938, to believe that the objects of sight were things he called ``colours'', which, whatever else they were, were precisely ideas in Frege’s sense. We think there is a systematic reason why philosophers as insightful as these were uniformly unable to hold onto the realism with which, with the century, they began. It is, in brief, that (like Kant, as per the 4th paralogism) they did not have the tools really to resist a form of the argument from illusion. Those tools came only later, with Austin. We will elaborate this point in due course.

One more initial point. In addition to the realism just sketched, Cook Wilson also contributed to Oxford philosophy a new conception of philosophical good faith (certainly new relative to Hume, to Hegel, and to most of the post-Cartesian tradition). It is a conception perhaps better known as later championed by Moore. Cook Wilson expressed it thus:
\begin{quotation}
	\noindent The actual fact is that a philosophical distinction is prima facie more likely to be wrong than what is called a popular distinction, because it is based on a philosophic theory which may be wrong in its ultimate principles. \ldots\ There is a tendency to regard the linguistic distinction as the less trustworthy because it is popular and not due to reflective thought. The truth is the other way. Reflective thought tends to be too abstract, while the experience which has developed the popular distinctions recorded in language is always in contact with the particular facts.
	
	Now it is not uncommon in philosophic criticism that some popular term, when reflected on, presents great difficulties to the philosopher; difficulties which are often due to some false theory of his which is presupposed. The criticism sometimes ends \ldots\ so that \ldots\ any distinctive use of [the term] is supposed to be an illusion, or the meaning of the term may be pronounced to be altogether an illusion. When the philosopher arrives at such a conclusion it too often happens that he is satisfied with this negative result. \ldots\ We ought under such circumstances to inquire how it is, if the given term only means something else, that language ever developed it, and still so obstinately holds to it, and when we believe that we have explained a term away or shown that it is a mere unnecessary way of disguising some other meaning, we ought to put our result to the test by trying to do without the word criticized and seeing what would happen if we everywhere substituted for it what we suppose to be the truer expression. (1926: 875)
\end{quotation}
A philosopher's claims must be answerable to something. If they are, say, claims about seeing, there is nothing better to which they may be answerable than the way the verb ``see'' is actually used. This is one way of putting the foundations of what came to be known as ordinary language philosophy---some decades before there was any. This, though is a point about philosophic methodology. It does not yet identify the main focus of 20th century Oxonian interest in language. 

Despite that salient difference in focus between Frege, on the one hand, and Prichard and Cook Wilson on the other---despite the centrality of knowledge in 20th century Oxford’s concerns---we divide the following discussion into three sections in this order: language, knowledge, and perception.

% section introduction (end)
% \section{language} % (fold)
\label{sec:language}

Unlike Oxford views of knowledge, and of perception, the most significant Oxford views of language are not ones which persisted throughout the century. Rather, with their roots firmly in Cook Wilson, they flowered from the late \'40s until the early \'60s, largely thanks to J. L. Austin, and then more or less disappeared from the scene. It would be an interesting exercise---perhaps largely in sociology---to explain why this is so (though we will not attempt that here). For, we shall suggest, Oxford’s most distinctive views of language were borne mostly of necessity. More specifically, they were (or were seen as) what was necessary in order to keep afloat those very views of knowledge and perception which not only bear the Oxford mark, but, moreover, did persist at Oxford into this millennium. A question which is very hard to answer is how, after the early \'60s, proponents of these last views thought they could get along without those insights Austin found essential. Again, we shall do at best only a little towards answering it.

There were, in the last century, two distinctive Oxford views of language. One is a particular conception of the relation of language to thought (or thoughts); thus, also, a particular conception of truth. The other is, in effect, a methodological strategy. One, one might say, concerns the relation between mind and language, the other is a strategy of minding one’s language. We do not normally attend to the ways our words work, but rather to what we hope to work with them. But, the idea is, in philosophy words can all too easily work to block our view of the phenomena we mean to speak of; clarity as to their workings—how and when they actually apply---often is the best way to see through them to those objects of our study. Both these views are rooted in Cook Wilson, though in somewhat different ways. We begin here with the first.

There is a line of thought in Cook Wilson’s treatment of a notion of a proposition, and its role in logic, which adumbrates a main line in Austin’s view of language, and which, going on texts, may well have influenced it. Cook Wilson was, roughly, a contemporary of Frege. So it is fair to compare the two. Both wrote on logic. On first reading, Cook Wilson---precisely in his concern for the ordinary use of words---may seem to be missing all Frege’s best insights. No doubt he did miss some, though on closer reading perhaps not \emph{quite} so many as first appears. In any event, both agreed in finding a \emph{grammatical} distinction between subject and predicate---a distinction as generated by English or German syntax---of little or no relevance to logic. Frege writes:
\begin{quote}
	Our logic books still drag in much—for example, subject and predicate—that really does not belong to logic. (1897: 60))
\end{quote}
Rejecting that distinction, he gives fundamental importance to another, that between \emph{object} and \emph{concept}. Cook Wilson writes:
\begin{quote}
	The above analysis [of a statement, or proposition] would make the distinction of subject and predicate, one not of words but of what is meant by the verbal expression. We may call this the strict logical analysis, and the distinction of the words of the sentence into `subject words' and `predicative words' may be called the grammatical analysis. (1926: 124)
\end{quote}
Thus, for example, in `That building is the Bodleian', `that building' is the grammatical subject; in `Glass is elastic', `glass' is the grammatical subject. But in the first either `that building' or `the Bodleian' may identify the \emph{logical} subject, depending on just what is being said. In the second, either `glass' or `elastic' may identify the logical subject. \emph{Mutatis mutandis} for logical predicates. In other words, the \emph{sentence} `Glass is elastic' (or any other) may, while meaning just what it does, having precisely the syntax and semantics it does, so while having the same grammatical subject and predicate might have either of two pairs of strict (or true) logical subject and predicate; which is to say (given the relational nature of the notions \emph{subject} and \emph{predicate}) that it may express either of two different propositions. Like Frege, Cook Wilson dismisses the grammatical notion as irrelevant to logic. But he does find some, though it is unclear just what, relevance in the logical notion (since he thinks that, in some sense, ``logic \ldots\ is some study of the nature of our thinking'' (1926: 150).

To a Fregean, two or three things may seem to have gone wrong already. One of these lies in something Cook Wilson stresses about the just-mentioned `logical' distinction. As he puts it:
\begin{quote}
	Subject and predicate mean not the idea or conception of an object, but the object which is said to be an object of the idea or conception. But, while the things called subject and predicate are objects without anything that belongs to our apprehension of them or our mode of conceiving them, the distinction of them as subject and predicate is entirely founded on our subjective apprehension of them, or our opinion about them, and on nothing in their own nature as apart from the fact that they are apprehended or conceived. It may be said that the distinction is not in them, but in their relation to our knowledge or opinion of them, and so not a relation between what they are in themselves apart from their being sometimes apprehended. (1926: 139)
\end{quote}
This, for a start, may seem to portray \emph{logical subject and predicate} as mere psychological notions of a sort which, for Frege, could also have no bearing on logic. (A related issue: Frege’s distinction between \emph{concept} and \emph{object} precisely \emph{is} a distinction between the sorts of things we designate in expressing the thoughts we do.) But the notions of logical subject and predicate need not be read as psychological in any such tendentious sense. They need not be psychological notions in any sense in which Frege’s notion of a thought would not also be psychological. A thought, for Frege, is the content of a certain sort of stance for a thinker to take towards the world. In taking such a stance a thinker would expose himself to risk of error, of a sort succumbed to or avoided merely in the world being as it is (thus an objective stance). The thought which is the content of that stance is what fixes precisely what risk a thinker would thus run; just \emph{when} he would succumb, just how the world may matter to whether he has. Stances towards the world are part of a thinker’s psychology, on a perfectly good use of that term. Being psychological in this sense need not mean that it is a psychological matter what such stances there are to take, and certainly does not mean that it is a psychological matter how such and such stances relate to one another (e.g., which ones stand farther down or up on truth-preserving paths).

Cook Wilson’s logical subjects and predicates need not be psychological in any other way than a Fregean thought would be. For Frege, a thought marks a commitment there is for one to make in exposing himself to risk of error; accordingly (given the sort of error one risks), a way to represent things to be. Differences between thoughts are thus differences in ways there are to represent things; in the commitments there are for one to make. For Cook Wilson, propositions which differ in whether such-and-such constituent is their logical subject are ones which differ in being answers to different sorts of questions; just where answers which so differ thereby differ in what one is committed to in giving them. Such need be no more psychologistic than any other thesis as to how Fregean thoughts are to be counted; as to how one commitment, or exposure to risk of error, is liable to differ from another. Of course, so read, it is a \emph{substantial} thesis. It needs to be made out that the differences in commitment which Cook Wilson finds correspond to different risks to run of error; and, perhaps, that such difference in risk can make the difference to whether one has represented \emph{truly}. But then, this just adumbrates the really important issue to come.

In his very dismissal of the grammatical subject-predicate distinction, as well as in many other contexts, Frege insists:
\begin{quote}
	Thus we will never forget that two different sentences can express the same thought, that as to the content of a sentence, what concerns us is only what can be true or false. (1897: 60)
\end{quote}
One sentence, perhaps, can express many thoughts (each on some occasion). But what concerns Frege here is that many sentences can express \emph{one} thought. As he often stresses, the same thought can be articulated, now this way, now that, so that now this, now that, appears as predicative in it. The same thought can be structured in many different ways out of many different sets of concepts and objects. Intuitively, we can see how we would, in some sense, understand `That building is the Bodleian' differently depending on whether it was an answer to the question what that building is, or an answer to the question which building is the Bodleian. But what we have not seen---and what, it seems, Cook Wilson has done nothing towards showing us---is that \emph{that} difference in understanding makes for different thoughts expressed---or, again, exploiting Frege’s above framework, that such a difference could make any difference to when the thought thus expressed would be true.

Frege’s object--concept distinction falls on one side of another distinction, equally fundamental for him, between sense and `Bedeutung'. One might think of this \emph{Bedeutung}, on Cook Wilson’s lines, as what we speak of, on some understanding of speaking of. But it is not the sort of object of discussion that Cook Wilson has in mind. Rather, it is, so to speak, a distillate from things at the level of sense, notably thoughts, of what matters for the sorts of calculations, or relations, of concern to logic, most notably truth-preservation. Frege begins a discussion of his main essay on the sense-reference distinction by remarking,
\begin{quote}
	The fundamental logical relation is that of an object falling under a concept; all relations between concepts reduce to this. (18920-1895: 25)
\end{quote}
He goes on to observe that, waiving some grammatical niceties, there is considerable justice in the view of extensionalist logicians. Having first explained how attempts to name concepts, or what they name, with expressions like, `the concept \( A \)', or `What the concept \( A \) names' generally misfire, so that, e.g., in saying `The concept \( A \) is (identical with) the concept \( B \)', we end up speaking of a relation between objects when we really mean to be speaking of one between concepts, he goes on to remark:
\begin{quote}
	If we keep all this in mind, we are indeed in a position to say, `What two concept-words denote is the same just in case the associated extensions of the concepts coincide'. And with this, I think, an important concession is made to the extensionalist logicians. (1892-1895: 31)
\end{quote}
If logic is concerned with, as Frege puts it, the laws of being true (\emph{Wahrsein}), then logic is concerned with thoughts, since, as Frege also insists, thoughts just are the things which, in the first instance, are eligible to be true or false (the things which make questions of truth arise). (See 1918: 59-60.) But the business of logic reduces, for most purposes, at least, to operations on the level of \emph{Bedeutung}. The first sentence here is all that is needed, and really all that Cook Wilson demands, to honour his insistence that logic is, in some sense, about thought. The second seems entirely consistent with his views on the role of relations between things as opposed to our manners, on occasion, of apprehending them.

So though, for several reasons, Frege is not prepared to say (or admit to having said) just what a concept is (here see 1904), one can think of what is at the level of \emph{Bedeutung} as including such things as mappings from some range of things to others; as the taking on of such-and-such range of values for such-and-such range of arguments. (Again, we may, with Frege, keep grammatical obstacles in mind.) What corresponds to objects and concepts at the level of sense is, to use one of Frege’s terms for this, modes of presentation of them: ways of thinking of something which bring some Fregean object, or concept, into play. For example, in speaking of fauns as being gambollers, I bring into play, for purposes of calculating truth preservation, among other things, a function from objects to truth-values which takes on the value true for just those objects which, as it happens, gambol. So speaking of being a gamboller is a way of presenting things which brings that concept into play; accordingly, for Frege, a way of presenting it. What there is not at the level of sense, on Frege’s conception of things, is anything corresponding to logical subjects and predicates, or more pertinently, since something would be a logical subject, or predicate, within some given proposition, or something of that form, there is, for Frege, nothing at the level of sense which has logical subjects and predicates. Certainly thoughts do not. Thoughts, for Frege, articulate into elements—being about certain objects, or was for them to be—only relative to an analysis. If we were to decompose a thought so that its elements were being about the Bodleian, and being about being in the Broad, what we would thus have would be, in effect, a mode of presentation of that thought—a way, one among others, of thinking about \emph{it}. We would have a mode of presentation of a mode of presentation of whatever it is, at the level of Bedeutung, that thoughts present (for Frege, a truth-value). If what is to be found at the level of sense always presents something at the level of reference, there is no room for a distinction between logical subject and logical predicate at either of Frege’s levels.

Cook Wilson also has a second level corresponding, in some way, to Frege’s level of \emph{Bedeutung}. It is inhabited by the things we talk about, on an ordinary understanding on which this includes, for example, the Bodleian, glass, being in the Broad, and being elastic, and by `real relations' between them. So it is not quite inhabited by the same things which belong to Frege’s \emph{Bedeutung}. But it might be seen as inhabited by Cook Wilson’s candidates for the things which really matter to the concerns of logic---notably truth-preservation. For he insists that when we say, `That building is the Bodleian', no matter what the grammatical, or even logical, subject may be, what we \emph{speak of} is just that building being the Bodleian. Which, one might well think---and Cook Wilson seems sometimes to think---leaves nothing for truth to turn on but whether that building \emph{is} the Bodleian. Perhaps it is this which leads Cook Wilson to say, of the subject-predicate distinction, no matter how drawn:
\begin{quote}
	It remains to say that the choice of one or other method of formulating the distinction of subject and predicate, in accordance with what seems to be the only rationale of the traditional definition, is a matter of no great moment, for the distinction is of no importance in logic proper, and indeed of no use whatever for the solution of the usual problems of logic. (1926: 124)
\end{quote}
But then, why is there \emph{any} interest in the notions of (strict) logical subject and predicate, at least if one’s concern is, like Frege’s, only with that in the understandings take words to bear to which laws of logic might apply? How can whether such-and-such is the logical subject of one’s statement matter to the error one risks in stating it (or in judging what is thus stated), at least where such error is error as to how things are (or are correctly viewed as being)?

One approach to answering these questions would be as follows. Frege, while admitting that there are all sorts of aspects to the ways in which one would understand the words we in fact speak, allows into sense, in his sense, only what bears on questions of truth. That is why notion corresponding to logical subjects and predicates shows up, for Frege, at the level of sense. The most obvious way to place those notions there would be to show that they \emph{do} bear on truth; that two truth-bearers (proposition, thoughts, statements) which differed \emph{only} in that the logical subject in one was the logical predicate in the other, and vice-versa, might, for all that, differ in when they would be true. Such would require logical subjects and predicates at \emph{Frege’s} level of sense. Such an idea seems to have inspired Austin. His essay, “How To Talk (Some Simple Ways)” (1952) is, in effect, a more refined elaboration of Cook Wilson’s idea; its object (or one of them) is to show that distinctions of this kind do bear on questions of truth; or on whether one is correct as to how things are.

In “How to Talk”, Austin marks two distinctions---two pairs of distinctive features---where Cook Wilson has only one. He distinguishes, first, between `directions of fit', and second, between what he calls `onuses'. The first distinction is illustrated by cases like this: there is a flower, and a battery of kinds of flower it may be. Looking through the chooses, one commits to it being a dahlia, and not, say, an iris; by contrast, one is asked, of an array of flowers, which one is the \emph{dahlia}, and answers, `This one'. In the first case, one fits the flower to a rubric (in Austin’s terms, `cap-fitting'. In the second, one fits a rubric to the flower. Austin also calls the first thing `placing', and the second, casting. (In this presumably exploratory work he is neither parsimonious, nor elegant, with technical vocabulary.) The contrast in onus is made with examples like the following. There is a color sample---a piece of cloth, say. It is perfectly clear how it is colored. The question is whether being so colored is being crimson. (`Can you really call it \emph{crimson} when there is so much blue in it?') Or it is perfectly clear what it would be for something to be (when it would be) crimson; what is in question is whether \emph{this} sample qualifies. (`Doesn’t it have too much blue in it?')

Austin’s two contrasts yield four possible pairs of distinguishing features---of an onus and a direction---and, correspondingly, four different things to be done in saying such-and-such (that flower) to be such-and-such way or kind (a dahlia, say). Complicating his initial model slightly, these four things to be done become what he calls `calling', `exemplifying', `describing' and `classing'. At which point he points to the different considerations that would come into play in holding one or another of these performances to be \emph{mistaken}, or \emph{incorrect}:
\begin{quote}
	If we are accused of wrongly calling 1228 a polygon \ldots\ then we are accused of \emph{abusing language}. \ldots\ In calling 1228 a polygon \ldots\ we modify or stretch the use of our name \ldots\ If on the other hand we are accused of wrongly describing, or of \emph{mis}describing, 1228 as a polygon, we are accused of doing violence to the \emph{facts}. In describing 1228 as a polygon \ldots\ we are simplifying or neglecting the specificity of the item 1228, and we are committing ourselves thereby to a certain view of it. (1952: 147-148)
\end{quote}
Different ways of going wrong, for different combinations of fit and onus, raise the possibility of going wrong in some such combination, in speaking of, say, \emph{this} flower as a \emph{dahlia}, where one would not go wrong in another combination in speaking of precisely that. Depending on the sort of wrongness involved here, this might be the very sort of contrasting pair that Cook Wilson would need in order to bring his logical subjects and predicates into the realm of sense---aspects of the understandings we bestow on words which do bear on questions of truth. So, for example, if it is France, or a piece of iridescent fabric with the red appearing as behind the blue, or a genetically modified `dahlia' which is neon orange, ten feet tall, glows in the dark, eats birds and sometimes small children, etc., then it may not be true to how the thing is---may mislead, or even misinform---to call it, respectively, a polygon, or crimson, or a dahlia. If what you are doing is saying how the \emph{thing} is, then you have chosen at the least very bad terms in which to do it. Whereas if the question is what you \emph{could} call a polygon, or crimson, or a dahlia---what being this these things really \emph{is}---then you can call France a polygon if you ignore enough irregularity, the sample crimson if you ignore the blue sheen the crimson would then be seen through, the flower a dahlia if you do not mind what dahlias might get up to, so long as the DNA is close enough. And, perhaps, there is nothing in the notions \emph{polygon}, \emph{crimson}, \emph{dahlia}, which rules out, absolutely, so viewing things. If to call France a polygon is to take a certain view of France, it being given what France is like, then that may be, at the least, a very bad view to take. Whereas if to allow that polygonal is the sort of thing France just \emph{might} be allowed to be is to take a certain view of being polygonal, that just might not be such a bad view to take of being that.

We are still some distance from making the case that would need making to install logical subjects and predicates (or Austin’s more refined successors to them) within the realm of sense. One would need to make out that very bad views, such as one of that monster as a dahlia, may correspond to representing \emph{falsely}, or at least not truly. That would take some work. But we need not pursue this issue further. For lines of thought such as this one suggest a certain generalisation, which can be shown on independent grounds: whether one speaks truth in saying things to be a certain way, or \emph{a} thing to be a certain way (or of a certain sort) depends on the standards to which one is thus to be held; where these standards depend, not only on what the words you use speak of—just what, simply in and by using them, you are saying to be what—but also on the circumstances in which you speak (on such things as what questions you are to be held responsible for answering in \emph{so} speaking). You spoke of that flower as a dahlia, or of France as a polygon. To what standards of correctness are you thus to be held? What would be required for you to be correct in speaking of that as a dahlia? That question is \emph{not} answered by all said so far as to what you did. This is the generalisation Austin expresses in \emph{Sense and Sensibilia}, in saying:
\begin{quote}
	It seems to be fairly generally realized nowadays that if you just take a bunch of sentences (or propositions, to use the term Ayer prefers) impeccably formulated in some language or other, there can be no question of sorting them out into those that are true and those that are false; for \ldots\ the question of truth and falsehood does not turn only on what a sentence \emph{is}, nor yet on what it \emph{means}, but on, speaking very broadly, the circumstances in which it is uttered. Sentences are not as such either true or false. (1962: 110-11)
\end{quote}
Whether one speaks truth or falsehood in saying that cloth to be crimson, or that fossil a dahlia, depends on the circumstances of one’s so speaking, and on the standards for things being the way in question---the conditions on truth---that then and there apply. So one may, on one occasion, speak truly, and on another falsely, in and by saying the very same thing, in the very same condition, to be crimson, or a dahlia, or and so on \emph{ad infinitum}. Otherwise put, there are various things being crimson, or being a dahlia, might be understood to be; where one speaks either truly or falsely in speaking of something as a dahlia, there is something this is to be understood to be, where that is just one of an indefinite variety of things this might be.

So the idea of logical subjects and predicates had, by mid-century, in Austin’s hands, turned into the idea that there are many things that might be understood by something being some given way---by being a dahlia, or crimson, for example---where, on different such understandings, different ranges of things would count as those of which it was true that they were dahlias, or crimson, or whatever; that a given way (or sort of thing) for things to be, specified no matter how, does not as such pick out any unique range of things as its instances, full stop; but that what counts as instancing it on one way of viewing this is liable not so to count on others. Such is one way in which at least some of Oxford, by mid-century, had built on the foundations Cook Wilson laid in the first decade of the century (or perhaps before). But it would certainly be wrong to suggest that this view of language was ubiquitous in Oxford at mid-century. And, as noted already, it is a curious fact that, by some time in the \'70s, it had more or less died out.

The most significant dissenter, around mid-century, at least, was H. P. Grice. He first broached his counter-view in (Grice, 1961), and then, more fully, in his William James lectures of 1968(?). First we need to note one small corollary of the view just set out. Suppose that, as per that view, words (e.g., `Fauns gambol') \emph{underdetermine} what would be said in using them as meaning what they do (since, as per above, that might be any of indefinitely many distinguishable things). Then there is substantial work for circumstances of a speaking to do---again, as Austin insists in \emph{Sense and Sensibilia}. In those circumstances, there must be something which would be to be understood by, e.g., \emph{gambolling}; and this should be substantial enough to make what was said in that speaking truth-evaluable---gambolling, on the required understanding, must be something fauns either do or fail to. It is always possible in principle, and, Austin thinks, it sometimes occurs in practice, that circumstances are just not up to the job. So you cannot expect to say, `Fauns gambol' just any time you please and thereby say something either true or false. Or if, through kindness of the world, you might expect this with `Fauns gambol', perhaps you will have poorer luck with that strategy for a sentence like `Sid tried to lift his pen', or `Pia did it of her own free will'. It is this corollary of Austin’s view on which Grice focuses.

Grice’s case against Austin is centred on the thought that while, in speaking, we may say things that are either true or false, we may also suggest, or imply, or etc., other things which are either true or false. If I say, ‘Pia became pregnant and married’, I may certainly at least suggest that the first-mentioned preceded the second---though (importantly for Grice) it is possible to arrange my saying this so that I would not. But the fact that I am likely at least to suggest this is compatible with my not actually having \emph{said} it, with my at \emph{most} suggesting it; and certainly compatible with there being nothing in the meaning of `and', or any other feature of the sentence uttered, which concerns temporal order. Grice introduces the technical term `implicate' for all those ways I, or my words, may have related to propositions about temporal order other than stating them.

Now the core idea to be used against Austin is to be: where Austin sees the possibility of \emph{saying} a variety of things in given (unambiguous) words (while meaning what they do), Grice will argue that this variety of things is only implicated, while, in fact, there is some one thing (to be specified) which is what was said. Or rather, this is what Grice needs to argue. He tends, instead, as mentioned, to focus on the corollary, arguing instead that if, in certain circumstances, one would not say, e.g., `Sid tried to lift his pen', this may be, not because what one thus said would not be true, but rather because one would implicate something unwanted. It is not clear that Grice really understood what Austin’s point was. If not, this may be because of what proved to be an unfortunate choice of vocabulary by Austin and Austinians. We will come to that issue shortly. In any case, the idea of implicature is arguable ill-suited for the application it would need for it to touch Austin’s view. The idea to be countered is: a sentence, say, `That painting is crimson', may be used of a given painting, in a given condition, to say different things, some true, some false, where there is no limit, in principle, to the new things new occasions may make available thus to say. The counter would be: these different things are merely implicated. But then, what is implicated, on any such occasion is, on some possible understanding of being crimson, that the painting is crimson. Now what, in addition to \emph{that}, is to be the thing which is \emph{said} throughout all those cases? Surely something to the effect that the painting is crimson. So it is `crimson', whatever that comes to (on some understanding so being), and, moreover, for a given occasion, it is what being crimson is to be understood to be on that occasion. But what is this additional thing which being crimson always comes to throughout? And how is \emph{that} compatible with the different things it would be taken to come to on different occasions?

We just mentioned an (as it proved) unfortunate choice of vocabulary---one of which Grice certainly makes much. This choice is most evident in the second methodological point, begun by Cook Wilson, developed by Austin. The idea was: in philosophy, we need to mind our language. This idea is put most clearly and elegantly by Austin:
\begin{quotation}
	First, words are our tools, and as a minimum, we should use clean tools: we should know what we mean and what we do not. and we must forearm ourselves against the traps that language sets us. Secondly \ldots\ we need to prise them off the world \ldots\ so that we can realize their inadequacies and arbitrariness, and can re-look at the world without blinkers. Thirdly \ldots\ our common stock of words embodies all the distinctions men have found worth drawing, and the connexions they have found worth making, in the lifetimes of many generations: these surely are likely to be more numerous, more sound, since they have stood up to the long test of the survival of the fittest, and more subtle, at least in all ordinary and reasonably practical matters, than any that you or I are likely to think up in our armchairs of an afternoon \ldots
	\ldots\ When we examine what we should say when, what words we should use in what situations, we are looking again not merely at words \ldots\ but also at the realities we use the words to talk about: we are using a sharpened awareness of words to sharpen our perception of, though not as the final arbiter of, the phenomena. (1956-57: 182)
\end{quotation}
We should, we are told, mind our language for several reasons. For one thing, philosophical problems often depend on taking some word in its usual (ordinary, English) sense. Has anyone ever seen a tomato? If that is not in question when it is asked whether what we see are things in our environment, or if it is in question only in some technical sense of ‘see’ (to be specified), then the question is not obviously as interesting as it initially seems to be, and much more work needs to be done to show it to be interesting at all. That one does not see `material objects' was (we thought) meant to be an amazing discovery. Conversely, for another, if philosophers are not to fly off into the empyrean, only to lose their way there, then they need to be held accountable for what they say. Causal relations hold only between mere appearances. Oh, really? So you did not just now fill my glass. Oh, you didn’t mean that by `appearances'? Well, then, what \emph{did} you mean? (This is all too likely to prove to be nothing at all.) Finally, philosophers too often find introducing technical vocabulary, so as for it to make \emph{sense}, an all too easy matter. Seeing the complexities of ordinary vocabulary, and of the task of getting \emph{it} to apply to the world may be sobering. Moreover, it may show us how our \emph{thinking} falls into confusion by failure to note the complexities involved in isolating a phenomenon.

Austin’s advice should, perhaps, have been old saws, but in fact reconceives philosophical good faith, changes what a philosopher could say with a straight face from what this would be taken to be by Hume, or Bradley, or the subject at large in the 18th and 19th centuries, and in some quarters (cf., e.g., Sartre) in the 20th. But here the vexatious vocabulary intrudes. Austin speaks of what we should (would) say when. A natural way of speaking if you want to respect the idea that it is intrinsic to words to equip us to say, or do, different things with them in different circumstances, on different occasions for the doing. But `what we would say' can be read so as to encompass such things as not saying, `What’s the vigorish?' when your neighbour asks to borrow a cup of milk (but perhaps saying it if it is a cup of Scotch), or not saying `That’s just autobiography' to your small niece when she says she wants another biscuit. And this is how Grice is inclined to read it. On the other hand, asking what one would say when can be a way of asking what the words one uses in fact apply to, or for doing what they are in fact applicable---what one would say (as what one would describe a thing, for what one would ask, what sort of greeting or condolence one would convey) in using them (for what they are for in the language). If one is moved primarily by that main view of language, as developed by Austin from Cook Wilson’s seminal idea---that it is not, e.g., English words, but rather their use on an occasion, which determines how they may, or must, be articulated in understanding what they said---then one certainly will read those words `what we should say when' in this last way.

% section language (end)
% \section{Knowledge} % (fold)
\label{sec:knowledge}

% section knowledge (end)

The last section identified a distinctive core in J. L. Austin’s view of language and thought, and traced this view, intellectually, at least, to some at first sight peculiar ideas in Cook Wilson. Though Austin’s ideas on language were the most distinctive, and even, in some sense, the most distinctively Oxonian, in 20th century Oxford, it cannot be said that they survived there long, at least as a major presence. They were soon to be supplanted there by what has come to be known as the ‘Davidsonic boom’. No doubt Austin’s interest in language was intrinsic. But, as already suggested, it was motivated by commitment to another distinctively Oxonian view---perhaps the most distinctively Oxonian of all, and this time one which did flourish throughout the century. This is, in the first instance, a view of knowledge---though it has become almost a tradition (as evidenced, e.g., in very different forms, in Prichard, Austin and, later, John McDowell) to apply it to perception as well. There is something very compelling about the idea from which this idea begins. Yet it can immediately seem (as good as) impossible to reconcile it with (what seem) undeniable facts. What Austin was the first to see (and saw clearly) is that there is only one satisfactory resolution to this dilemma. It requires invoking the view of language presented in the last section. (As recent work has shown, it also requires some delicacy in the application.)

The core idea can be brought into proper focus by asking whether there is such a thing as knowledge based on evidence. The answer can seem clearly to be, ‘Yes’. That loopy expression on Sid’s face is some evidence that he has been drinking. His slurred speech is a bit more. His errant gait yet a bit more. Then he comes close; we smell his breath. Finally the evidence has mounted to the point where we can say we know he has imbibed. Or again, zinc stops colds. It worked for Sid and Pia. That is a bit of evidence. We broaden our study, perhaps introducing a control group. The evidence holds up. We broaden some more. The evidence continues to mount. If it does, then at a certain point we can say: we know zinc stops colds. But is this right? Suppose that all I have is evidence that Sid is drunk. Evidence may make this very probable. The idea here seems to be: if it is probable enough (say, very, very probable) then we know. If this is the idea, then there is the following to say: for it to be merely very, very probable that Sid is drunk is for there to be some chance that he is not. So if we reflect on our position in having very strong evidence that he is drunk, we come to see that, for all the evidence at our disposal, Sid might not be drunk. But to admit that Sid might not be drunk is to admit that we do not know he is. I cannot know he is if, for all I know, it is possible for him not to be. This, in one form, is the core idea. That idea can also be put this way: merely to have evidence, no matter how high mounted up, is not to have proof. But to know something is to have proof. So there is no such thing as knowledge by evidence. And, admittedly, that claim can seem absolutely incredible (and/or simply the assertion of scepticism).

Still, though, it is a compelling thought that to know is to have proof, to see how things stand in the relevant respect; and that one who merely has evidence does not see how things stand---or, conversely, if he does see, then he does not need evidence (and even, perhaps, that, in that position, nothing could be evidence for him). One way to know that Sid is drunk, e.g., is to witness his drunkenness (and to recognise what it is that one thus witnesses). One cannot do that just be having evidence. For, whatever the evidence is, it is something distinct from the fact of Sid’s drunkenness, so that awareness of it cannot be awareness of (witnessing) that drunkenness. And, conversely, if I am in that happy position just described, then evidence for his drunkenness, whatever it might be, is otiose. It cannot provide my reason for thinking he is drunk. True, I may know that Sid is drunk without witnessing his drunkenness. But this is not to say that I can know that Sid is drunk while enjoying any lesser epistemic status in re the question whether he is drunk than I enjoy where I am witness—that is, where I do any less than being acquainted with the fact of his drunkenness, so as to have that as the grounds on which I know. Such, expanded, is the idea that to know is to have proof.

Cook Wilson puts the point thus far as follows:
\begin{quote}
	In knowing, we can have nothing to do with the so-called `greater strength' of the evidence on which the opinion is grounded; simply because we know that this `greater strength' of evidence of \( A \)’s being \( B \) is compatible with \( A \)’s not being \( B \) after all. \ldots\ Belief is not knowledge and the man who knows does not believe at all what he knows; he knows it. \cite[100]{Cook-Wilson:1926sf}
\end{quote}

Indeed, he holds that knowledge is not analysable in any terms at all. It is not a form of belief, or of anything else, distinguished from other such forms by the present of such-and-such feature(s). Nor could one test to see whether the condition he was in was one of knowing that P by seeing whether it was a mental state with some feature, \( F \). (Indeed, though he speaks of knowledge as a frame of mind, or, sometimes, as a mental condition, Cook Wilson begins to point towards a sense in which knowledge is not a mental state at all.) To see whether you know that \( P \), direct your attention, not at yourself, but rather at the question whether \( P \), and attend to the answer to this that you have in hand. If you know that \( P \), then that which convinces you is, recognisably, absolutely incompatible with P not being so (in the way that a peccary’s presence on your path is absolutely incompatible with it failing to be so that there is a peccary there—though this is not a very Cook-Wilsonian case).

Somewhat later, Cook Wilson’s student, H. A. Prichard, expressed the view thus far as follows:
\begin{quote}
	Knowing is not something which differs from being convinced by a difference of degree of something such as a feeling of confidence, as being more convinced differs from being less convinced \ldots\ Knowing and believing differ in kind as do desiring and feeling, or as do a red colour and a blue colour. \ldots\ To know is not to have a belief of a special kind, differing from beliefs of other kinds; and no improvement in a belief and no increase in the feeling of conviction which it implies will convert it into knowledge. \ldots\ It is not that there is a general kind of activity, for which the name would have to be thinking, which admits of two kinds, the better of which is knowing and the worse believing. (1932/1950: 87-88)
\end{quote}

Here Prichard also stresses the idea that knowledge, or more precisely, its object, is neither true nor false. To think, say, that that peccary is bristly is to take a stand on (or posture towards) a certain question; to relate to a thought---a particular way of being right or wrong as to how things are. Whereas to know that that peccary is bristly is, intrinsically and irreducibly, to relate to the fact of that peccary’s being bristly---if there is no such fact, then \emph{ipso facto} there is no knowledge.

Both Cook Wilson and Prichard adjoin to the above a further view. Cook Wilson points towards this addition here:

A correct way to put the case before us seems to be that the two processes, the two states of mind in which the man conducts his arguments, the correct and the erroneous one, are quite indistinguishable to the man himself. But if this is so, as the man does not know in the erroneous state of mind, neither can he know in the other state. (1926: 107)

Prichard works this worry into the following idea:
\begin{quote}
	We must recognize that whenever we know something we either do, or at least can, by reflecting, directly know that we are knowing it, and that whenever we believe something, we similarly either do or can directly know that we are believing it and not knowing it. (1950: 86)
\end{quote}
I need only reflect, the idea is, on what makes me feel convinced that \( P \)---what convinces \emph{me}---and I can see immediately either that the way I stand towards \( P \) is knowing it, or, again, that it is not. So that, on reflection, one cannot mistake one’s own belief for knowledge, nor vice-versa.

We turn now to Austin. Austin sees something importantly right in Cook Wilson’s and Prichard’s view of knowledge. But it faces an obvious problem. It is Austin’s response to that problem which is the centre of interest here. For a start, Prichard stresses, above, that the difference between knowing something and merely being convinced of it is not a difference of degree. Austin concurs. As he puts it:
\begin{quote}
	Saying `I know' \ldots is not saying, `I have performed a specially striking feat of cognition, superior, in the same scale as believing and being sure, even to being merely quite sure': for there is nothing in that scale superior to being quite sure. (1946/1970: 99)
\end{quote}
Austin’s treatment of this point represents a crucial modification in the Cook-Wilsonian conception of knowledge. We will return to this presently. But first to introduce the problem.

Just how does knowledge differ from a maximal point on a scale of being sure? The obvious answer is: being sure, even \emph{very} sure, of something is compatible with it not being so. Knowing is not. As Cook Wilson and Prichard conceive that incompatibility, where one knows something, that \( P \), he is aware of what is (and what he could recognise to be) absolutely incompatible with it not being so that \( P \), which rules this out absolutely; he is thus in a situation which is absolutely incompatible with P not being so. In short, to know is (nothing less than) to have proof; which differs from having even very strong evidence in the way Cook Wilson points to. On this view, where one \emph{knows} that \( P \), there is a kind of relation between what answers the question \emph{how} you know this---what it is that reveals to you that \( P \) is so---and what it is that you know, that \( P \) is so. It is the kind of relation which holds between what is cited in a proof that there is no largest prime and there being no largest prime: what is cited in the proof rules out \emph{absolutely} there being such a prime. Suppose you see a peccary before you. That peccary’s presence before you, and, also, the fact of your seeing it before you, are as incompatible with it not being so that there is a peccary before you as the proof that there is no largest prime is incompatible with there being a largest prime. There is no room in conceptual space for the one thing---the peccary’s presence, e.g.,---without the other---its being so that a peccary is present. The peccary’s presence, and your seeing it, do not have the form of a fact that such-and-such (though that you see the peccary does). But the incompatibility here is one we are as well equipped to recognise as that between the facts cited in a mathematical proof and the failure of its conclusion to hold.

So if I am (say, visually) aware of a peccary before me, and can recognise what it is I am thus aware of, then, as we can all recognise, I am aware of what is absolutely incompatible with it not being so that there is a peccary before me. So, if I am in that condition, then, it would seem, on Cook Wilson’s apparently demanding conception of knowledge, I know, or can know, that there is a peccary before me. Such would be a way in which vision can afford me knowledge, even on the most demanding conception of what knowledge would be. But now we may ask whether vision really ever does place me in such positions. Take a situation in which I do see a peccary before me, in plain view. I can recognise, sure enough, that a peccary before me is incompatible with it failing to be so that there is a peccary before me. But can I also recognise that I am aware of what is incompatible with that? And if not, can I meet Cook Wilson’s demanding standard for knowing? After all, a ringer for my situation is conceivable: a situation in which, for all I can tell, I see a peccary before me, but in fact I do not (e.g., because it is a ringer-peccary). And now, despite its being a peccary before me that I am in fact aware of, and despite my preparedness to take what I am aware of for that, am I really in a position to swear, hand on heart, that there is no chance that I am in a ringer situation, that, given all I can see as to what my situation is, it is just inconceivable that my situation is such? Suppose one answers, ``Yes'': there are situations which you can just see not to be ringers, but to be genuine situations of seeing a peccary---and, of course, since there are ringers, situations in which you cannot see this. Then the question is when you are in the one sort of situation, when you are in the other---for, it seems, this is something you must be able to tell. To say the least, there are no ready answers to such questions.

It is, it seems, for just such reasons that Prichard in fact makes it doubtful that perception can be the sort of source of knowledge here envisaged. He says, e.g.:
\begin{quote}
	When knowing, for example, that the noise we are hearing is loud, we do or can know that we are knowing this and so cannot be mistaken, and when believing that the noise is due to a car we know or can know that we are believing and not knowing this. (1950: 89)
\end{quote}
And he suggests as a way for me to know that I do not know this that, on reflection, I see that ``such a noise can be caused by something other than a car, say, an aeroplane.'' (1950: 79) So if we think as above, it can seem at best questionable whether Cook Wilson’s conception of knowledge can make room for vision as a source of knowledge of such things as the whereabouts of peccaries. Which may seem to force on us a highly revisionary view of perception. But, as Cook Wilson himself saw, the trouble does not end here. (Thus, swallowing such a view of perception will not buy us Cook Wilson’s view of knowledge.) For, if one can be under the illusion of seeing a peccary on the path—if there are such ringer situations—one can also be under cognitive illusions. I may be as convinced as can be that I see before me a proof of the Pythagorean theorem when what I in fact see is a bogus proof. I would be willing to swear that what I see is incompatible with the Pythagorean theorem failing to hold. But I would be wrong. Same problems all over again; this time ones which Cook Wilson worried about quite a lot. Perhaps, as he insists, I could (in principle), if I reflected, come to see that what is before me is not a proof. But if I am as certain as can be that I have before me a proof that the continuum hypothesis is independent of the axiom of choice, it is cold comfort to be told that if this is cognitive illusion, it might, in fact, go away if I reflected enough. Do I now know the fact in question, or do I not? And just which cases of my so standing would be ones of my knowing this, which cases not?

A satisfactory account of knowledge should allow that, sometimes, one knows there is a peccary before him because he sees it. This might have been enough motivation for the reworking of Cook Wilson’s view which Austin proposes. But, for reasons Cook Wilson himself acknowledges, that view is in enough trouble anyway. Now, then, for the modifications. Two are crucial. A third follows of its own accord. One of the two crucial points concerns the right way to understand the idea that knowing is not some very high point on a scale of being sure. Austin puts the point in first person terms, making a comparison between the sort of force (typically) indicated by prefacing a statement with ‘I know’ and that indicated by ‘I promise’, used to make a promise. Comparing promising to stating an intention, he says:
\begin{quotation}
	\noindent When I say `I promise', a new plunge is taken: I have not merely announced my intention, but, by using this formula \ldots\ I have bound myself to others \ldots\ Similarly, saying `I know' is taking a new plunge. \ldots\ When I say `I know', I give others my word; I give others my authority for saying that `\( S \) is \( P \)'.
	\\
	When I have said only that I am sure \ldots\ I am not liable to be rounded on in the same way as when I have said `I know'. I am sure for my part, you can take it or leave it \ldots\ that’s your responsibility. But I don’t know `for my part', and when I say `I know' I don’t mean you can take it or leave it (though of course you can take it or leave it). (1946: 99-100)
\end{quotation}
This move of Austin’s has probably attracted more, and more vehement, criticism than any other he made. The complaint, briefly, is that saying what one does in saying `I know' (at that, only on one special sort of occurrence of it) is not offering an account of what knowledge is. Perhaps not. But, considered as a modification of Cook Wilson, it is an interesting move in that direction. The shift it highlights is this. Cook Wilson refers to knowledge as a `frame of mind'. The point about scales, in that context, is then that knowledge is a characteristically different frame of mind from any degree of being sure, or, in Cook Wilson’s terms, from believing, or being of an opinion. The question then would be just what state of mind it is. Presumably, in any case, for any potential object of knowledge (any fact), and any time, presumably a frame of mind which I am in at that time relative to that fact, or not, punkt. Either I am, or I am not, enjoying that sort of awareness which, on the Cook Wilson view, I am supposed to enjoy if I know. Whereas Austin here points us towards viewing knowing as a sort of status which I may be accorded, or enjoy. In saying, `I know that \( P \)', I pretend to a certain sort of authority—to being an authority on, having proof of---\( P \). The question then is whether I am really to be counted as having such authority. This is a question as to what I have done to earn that status; and what would need to be done to earn it. It naturally points in different directions than a question understood as one as to what frame of mind I am in.

Perhaps stating makes a better object of comparison for Austin’s purpose here than promising. (It is, anyway, another which Austin makes.) To state something is to represent oneself (truly or falsely) as a certain way; for a start as oneself judging that which is stated. If I state that that is a peccary beneath the oak, I represent myself as saddled by the world with so thinking; as being unable to think otherwise on that score while pursuing the goal \emph{truth}. I thus offer you relief from certain intellectual (and perhaps other) labour. I say that the cat has been put out. I thus offer you relief from looking for yourself. You may, of course, take up my offer or not, depending on what you think of me—just as Austin says you may, or may not, take my word when I say I know. And you may or may not be right to take up my offer, or to refuse it, depending on the work of discovery I actually have done, and on the work needed. Modulo nuances, so it also is if I say that I \emph{know} that the cat has been put out. There is work that must be done if one is to enjoy such status (\emph{what} work being liable to depend on circumstances). I may have done the needed work or not. I may accordingly be to be accorded, or not, the status in question.

In fact, this may be seen as picking up on another element in Cook Wilson and Prichard. Both insist that to see whether I know I must not try to examine my own mental state to see whether it has some feature which marks it as a state of knowledge---whether, e.g., I am `perceiving clearly and distinctly', where that is understood as an introspectibly detectable sort of awareness for one to have. Rather, I must turn my attention to the objects of my awareness---to the proof, say---and ask myself how they bear on the relevant candidate for knowledge—say, that there is no largest prime. As Prichard puts it, ``there is \ldots\ no special distinguishing characteristic of a belief which is true \ldots\ there is no way of discovering whether some belief is true except that of first obtaining knowledge of the fact to which the belief relates, that knowledge therefore necessarily not having been obtained by considering the truth of the belief'' (1950: 92-93); ``What is known \ldots\ is some part of an independent world \ldots\ The only way in which the nature of anything in this independent world can come to be known is \ldots\ by perceiving it \ldots'' (1950: 101).  Talk of frames of mind tends to lead away from this insight.

The second main point is the application of Austin’s view of language, as in the last section, to the general area of epistemic status. The most concise and full statement of that application is in lecture 10 of \emph{Sense and Sensibilia}:
\begin{quote}
	It seems to be fairly generally realised nowadays that if you just take a bunch of sentences \ldots\ impeccably formulated in some language or other, there can be no question of sorting them out into those that are true and those that are false; for \ldots\ the question of truth and falsehood does not turn only on what a sentence is, nor yet on what it \emph{means}, but on, speaking very broadly, the circumstances in which it is uttered. Sentences are not as such either true or false. But it is really equally clear \ldots\ that for much the same reasons there could be no question of picking out from one’s bunch of sentences those that are evidence for others, those that are `testable', or those that are `incorrigible'. (1962: 110-111)
\end{quote}
Nor, Austin adds, are there sentences which are intrinsically in need of evidence, or knowledge of which must, intrinsically, rest on such-and-such grounds. All of which also applies to what speaks of someone \emph{knowing} such-and-such (an application elaborated in more detail in our other source here, “Other Minds” (1946)). The general point—Austin’s development of Cook Wilson---can be put as follows. Consider a sentence such as `Fawns gambol', or `There is red meat on the white rug'. The first, by virtue of meaning what it does, speaks of, or represents, (on some aspect of those verbs) Fawns as gambollers. But this leaves room for different things to be said—for indefinite variety in what is said—on different occasions, in using that sentence to say fawns to be gambollers; things each of which would vary in the conditions under which they would be true. For example, it may matter to the truth of some such things, but not to that of others, how fawns would behave on ritalin. Similarly, for different uses of that second sentence, raw liver, or lightly seared ribeye, may, or may not, count as red meat; ribeye in butcher’s paper may or may not count as on the \emph{rug}. Similarly, then, for such ascriptions of epistemic status to a fact as that it is evidence for such-and-such, or to a person as that he knows such-and-such. Take, for example, ‘Sid knows that the cat has been put out.’ By virtue of what it means (and reference to Sid, a time, and a given cat), these words speak of \emph{Sid} then knowing that that cat has been put out. But, the point is, in using them to speak of that, one \emph{might} say any of an indefinite variety of things to be so; \emph{would} say different things to be so on different occasions for such use, where each such thing \emph{would} be so under different ranges of conditions. So that some such things may, in fact, be true while others are false. So it is not, in general, simply true that Sid’s condition---his being as he is---\emph{is} his knowing that \( P \), or is his not knowing this, independent of any occasion for asking what he knows.

As it stands, this is a bare skeleton of a position. Again, the flesh Austin places on those bones is, for the most part, in (1946). But it already allows us to understand something, perhaps otherwise puzzling, which Austin says about evidence. He remarks,
\begin{quote}
	The situation in which I would properly be said to have evidence for the statement that some animal is a pig is that, for example, in which the beast itself is not actually on view, but I can see plenty of pig-like marks on the ground outside its retreat. If I find a few buckets of pig-food, that’s a bit more evidence, and the noises and the smell may provide better evidence still. But if the animal then emerges and stands there plainly in view, there is no longer an question of collecting evidence; its coming into view doesn’t provide me with more \emph{evidence} that it’s a pig, I can now just see that it is, the question is settled. (1962: 115)
\end{quote}
The notion of evidence in question here---what Austin takes to be the notion which the English `evidence' expresses---is one on which evidence admits of being stronger or weaker, better or poorer; accordingly one on which even the best evidence is compatible with the non-obtaining of what it is evidence for. Just so, then, not merely seeing the pig, but being able to see \emph{that} there is a pig there, is \emph{not} evidence in this sense. Where I am in this position, there is no logical gap between \emph{how} I know there is a pig there---I see (by sight) that there is---and its being so that there is a pig there, no room at all for me to be in that position while there fails to be a pig. (Nor, where I plainly see the pig before me, is there clearly an answer to the question just how much \emph{evidence} for this those pig-droppings now in fact supply. Against just what sorts of pigless cases of their presence are we supposed to measure the strength of such supposed evidence?) There is here, then, the same relation between my grounds for saying so and what I would thus say to be so as there is in the case of the proof that I grasp that there is no largest prime and there being no largest prime. So if I can be in this position, then I can know that there is a pig before me on Cook Wilson’s demanding conception of what knowledge is. At which point it becomes completely unstartling that, as Cook Wilson, Prichard and Austin all hold, there is no such thing as knowledge by evidence. To deny that there is such a variety of knowledge is now \emph{not} to deny that one may know such things as that there is a pig before him.

Here Austin’s point about occasion-sensitivity applies. What may sometimes count as my seeing that there is a pig before me may also, sometimes not. (‘Sometimes’: on some occasions for the counting.) So \emph{I}, in being as I now am (where I stand before the pig in plain view) may sometimes count, and sometimes not, as seeing that there is a pig before me. There are, accordingly, different things to be said in saying me to see this, and, correspondingly, different things to be said in saying me to know it. For some of these, epistemic status is so to be understood that my position \emph{is} one of seeing that there is a pig before me, for others it is to be so understood that my position is not that. There \emph{are} circumstances in which it matters to an answer to the question whether I can see that there is a pig before me whether I can, by sight, rule out (certain sorts of) ringers. Thus it is that worries arise as to how Cook Wilson’s conception of knowledge leaves room for knowledge of the world we all co-inhabit at all. But there are also circumstances in which the answer to that question does not turn on such things: where it comes to measuring my status, ringer pigs are just not in the cards.

As noted already, this is not just a point about knowledge gained through perception. If I grasp a proof that there is no largest prime---if I see how, and that, what the proof cites proves this---then there is no gap between how I know---what it is which reveals this to me—and the fact that there is no largest prime. But do I see how, and that, the proof proves? Do I see what rules out all possibility of a ringer-proof? Within Austin’s general framework, the answer to that is an occasion-sensitive matter. Which means that the fact that one might sometimes need to worry about ringer-proofs—that ringers always are \emph{conceivable}---does not show that one can never count as seeing of a proof that it is a proof. That idea of occasion-sensitivity which marks Austin’s conception of language and its relation to thought is at work throughout in making Cook Wilson’s conception of knowledge unproblematic, at least in such ways.

Similarly, occasion-sensitivity allows us to understand Austin’s response to Ayer’s idea of incorrigible statements. Here Austin remarks,
\begin{quote}
	If, when I make some statement, it is true that nothing whatever could in fact be produced as a cogent ground for retracting it, this can only be because I am in, have got myself into, the very best possible position for making that statement—I have, and am entitled to have, \emph{complete} confidence in it when I make it. But whether this is so or not is not a matter of what \emph{kind of sentence} I use in making my statement, but of what \emph{the circumstances are} in which I make it. If I carefully scrutinise some patch of colour in my visual field, take careful not of it, know English well, and pay scrupulous attention to just what I’m saying, I may say, `It seems to me now as if I were seeing something pink'; and nothing whatever could be produced as showing that I had made a mistake. But equally, if I watch for some time an animal a few feet in front of me, in a good light, if I prod it perhaps, sniff, and take note of the noises it makes, I may say, `That’s a pig', and this too will be `incorrigible', nothing could be produced that would show that I had made a mistake. (1962: 114)
\end{quote}
Being completely entitled to one’s confidence that \( P \) is an epistemic status there is for one to enjoy (or fail to). Austin’s point then is: on any understanding as to what status this is on which I can count as enjoying it with respect to its being, for me now, as if I were seeing something pink, it is also one I can (or might) count as enjoying with respect to there being a pig before me now---and with anything else one might ever know, or of which one might be ignorant. My condition as I now view the pig before me is, quite likely, one which would sometimes count, and sometimes not, as my enjoying that status---depending on the occasion for the counting. But then, if its being for me as if I were seeing something pink is, genuinely, something to be known, or of which to be ignorant, then the same goes for that. Which makes room for knowing there is a pig before me to be an attainable status.

It is important here to combine the two main points just made. Thinking of knowledge as a frame of mind at least tempts us to think this way: I am now in the frame of mind I am in; either \emph{this} frame of mind is one of knowing there is a pig before me, or it is not. Which is so can only depend on how I am in being as I am. If we take a broad view of frames of mind, then, perhaps how \emph{I} now am depends on the broader circumstances in which I find myself. I may or may not, e.g., have views on water depending on what those broader circumstances are. But this is \emph{all} it can depend on. But suppose, instead, that knowing is a status I may or may not enjoy. To enjoy it, e.g., \emph{in re} whether there is a pig before me is to have done the needed work to count as authoritative on that subject. Whether I have done that work depends on what the needed work would be. And now if we ask just what work \emph{would} be needed (or sufficient) for this, an answer is more than likely to depend on circumstances. Not the circumstances in which \emph{I} find myself. That is not the view of knowledge now on offer. But rather the circumstances in which that question is raised (or might be raised). It is of \emph{me}, as I am now \emph{in re} the pig in plain view before me that there is a variety of things to say to be so in saying me to know that there is a pig before me.

[Which forestalls a common objection to occasion-sensitivity about knowledge. Abstracting from some irrelevant rhetoric about `high' and `low' standards for knowledge, the form of the objection is: as Sid stands before the pig, his circumstances (or `the' ones which then obtain) determine what, in them, would count as knowing there is a pig before him. Suppose, as it happens, they determine standards on which he does know this. (Such had better sometimes happen if occasion-sensitive is going to gain for us knowledge of the empirical world at all. Or so goes the objection.) But then there will be other circumstances which determine other standards, such that, finding ourselves in them, in which we, privy to all that Sid then was, would have to say that we did not know whether there was a pig. (Or, transforming the case, we may be just as well off in our circumstances as Sid was in his, except that, by the standards for knowledge set by our circumstances, we do not count as knowing there is a pig before us.) How can we, not counting as knowing whether there was a pig before Sid (or is one now before us) credit Sid with knowing this? For one thing, if we know that Sid knew there was a pig before him (to take that version) then, presumably, we know that there was a pig before him. But this contradicts the supposition. So, it seems, it is incoherent to suppose knowledge to be an occasion-sensitive matter.]

The first mistake here is to suppose that it is the circumstances I am in which determine what would count as \emph{my} knowing whether there is a pig, and the circumstances you are in which determine what would count as your knowing this. Such a supposition is inconsistent with the present story. On it, it is the circumstances in which knowledge is attributed which determine, first, what would be said to be so in attributing it, so, second, when such an attribution would be correct. There are both true and false things to be said of me, in my circumstances, in saying me to know that there is a pig before me. What remains true is this: someone may have said me to have known, on a certain occasion, that there was a pig before me, and thus have spoken truth, while someone else, at the same time as that first person, or at a later one, may have said me not to have known, in the situation of which the first person spoke, that there was a pig before me, and thus also have spoken truth. Such is allowed for, on the account now on offer, by the fact that what the first of these people said to be so in saying me to have known such-and-such may not be what the second said not to be so in saying me not to know this. The two remarks may be consistent. It is the same here as where one person says there to be meat on the rug, steak in butcher paper counting as on the rug for the purpose of what he says, and another says there to be no meat on the rug, steak in butcher paper \emph{not} counting as on the rug for purposes of what he said.

If we think of knowledge as a status rather than a frame of mind, one aspect of Cook Wilson’s, and Prichard’s story does drop out. Their story makes it essential to knowing that if I know something, I can, by reflection alone, come to see that I know it; and if I do not know something (even if I am fooling myself as to this), I could (in principle) by reflection alone, come to see that this is so. That idea helps show how Prichard and Cook Wilson think of frames of mind. For it takes a certain conception of this for the idea to make sense. Sid faces the pig in plain view. It is not as if he could, by reflection alone, see whether or not, on our occasion for answering the question whether he knows there is a pig, he would (ought to) count as enjoying the relevant status---as having done the work needed to count \emph{there} as authoritative. For that, he would need to be acquainted with \emph{our} circumstances, which, in general, would not be there for him to be acquainted with at all. Nor need it be so that, by reflection alone, he could tell whether, in his circumstances, one would need, e.g., to have taken such-and-such precautions against ringers if one were to count as enjoying the relevant status. Rather, he has done the work he has done; and, as it may be, in given circumstances that does count as work enough for achieving that status. It neither so counts, or fails to, of course, unless those circumstances are such as to make relevant questions as to what Sid knows arise.

Austin thus offers a way of understanding knowing as (nothing less than) having proof, in the way Cook Wilson and Prichard do, while making its instancing coincide with what we are prepared to acknowledge its instancing to be---as envisaged by Moore, though matched with a new view as to what it is that we are prepared to recognize. Austin’s reworking of Cook Wilson, along with his ideas about language and thought, were things up with which (most of) Oxford was not for long prepared to put. There are probably several intersecting reasons for this, among which the rapidly accelerating Americanization of everything, including philosophy. Whatever Austin’s fate, though, Cook Wilson’s idea hardly disappeared. Rather, it, or its main part, lived on, most notably in the work of John McDowell. There, touched by different Austinian ideas, it appeared as an anti-hybridism, or (the same thing) disjunctivism about knowledge. McDowell, like Cook Wilson and Prichard, rejects the idea that knowledge that \( P \) could be some sort of condition in which all that one was actually aware of, so far as that went---whatever it was that might answer the question how he knows, what it is that reveals this to him---was compatible with \( P \) not being so.

McDowell shares Cook Wilson’s conception of knowing as nothing less than having proof---as having that \( P \) revealed to one by that which leaves no room for \( P \) not to be so. McDowell arrives at this point by considering the idea that ``we ought to be able to achieve flawless standings in the space of reasons by our own unaided resources, without needing the world to do us any favours'' (1995/1998: 395-396). This, as he sets things out, is the first step to a hybrid conception of knowledge. Knowledge, on this idea, would consist---at least in part---in having conducted our cognitive affairs according to the highest standard rationality imposes on this, where the responsibility for having so conducted ourselves is \emph{entirely} ours; there is \emph{no} demand for the world to be obliging if we are to reach the mark. But knowledge is factive: you do not know that \( P \) if \( P \) is not the case. So we may ask, for given \( P \), whether there is any way of conducting one’s rational affairs which one can guarantee for himself---see by mere reflection---that he has engaged in---a way for which there simply are no ringers---which, where it has a given terminus, leaves no room whatever for \( P \) not to be the case. Descartes asked that question. On reflection, the answer seems to be ``No''; not just in the case of perceptually based knowledge---most of us know all too well that it is possible for it to seem to one that he has, unmistakably, a proof of a proposition in geometry or number theory when he does not.

So the above requirement on knowing cannot be the \emph{only} requirement. There must be an extra requirement, over and above any such requirement on our conduct of our cognitive affairs, which is, at least in part, that \( P \) be the case. Thus a hybrid conception: knowing that \( P \) is part our own responsibility, in the way just sketched, and part the world’s. McDowell then has this to say about that conception:
\begin{quote}
	In the hybrid conception, a satisfactory standing in the space of reasons is only part of what knowledge is; truth is an extra requirement. So two subjects can be alike in respect of the satisfactoriness of their standing \ldots\ although only one of them is a knower, because only in her case is what she takes to be so actually so. \ldots\ Its being so is conceived as external to the only thing that is supposed to be epistemologically significant about the knower herself, her satisfactory standing in the space of reasons. That standing is not itself a cognitive purchase on its being so \ldots\ Then how can the unconnected obtaining of the fact have any intelligible bearing on an epistemic position that the person’s standing \ldots\ is supposed to help constitute? (1995/1998: 403)
\end{quote}
McDowell concludes that it cannot. It is as with evidence, conceived as what is liable to be stronger or weaker. As Cook Wilson notes, when we reflect on our position where such a thing is how we know that \( P \), we see that our reasons for taking it that \( P \) are consistent with \( P \) not being so; hence that we do not know that \( P \).

McDowell’s remedy (what has come to be known as disjunctivism) is to suppose that there are two sorts of case. No one, of course, can know that \( P \) if \( P \) is not so. But suppose that \( P \) \emph{is} so. Then there are still two (generally attainable) ways of standing towards \( P \) being so. In one way, as McDowell puts it, the fact that \( P \) is made apparent to one: What reveals to you that \( P \) is nothing less than what (recognizably) leaves no gap between itself and \( P \) being so---e.g., I know that there is a pig in the pen because I see it for myself. In the other case, this is not so. I have at best defeasible reasons for \emph{thinking} that there is a pig in the pen. It is thus not \emph{apparent} to me that \( P \); things merely so appear (vide 1982/1998: 386-387). The first sort of case---its being apparent that \( P \)---is knowledge. The second sort---its merely appearing that \( P \)---is not. If I can just \emph{see} the pig in the pen, and can recognise what I see for what it is, I need not worry about the strength of any reason I may have for taking there to be a pig there. We thus make room for experience to be a source of knowledge, within Cook Wilson’s conception of what knowledge is.

But suppose there is now, in plain view before me, a pig in the pen. McDowell suggests that there are two conditions I might thus be in. One of these is seeing for myself, hence knowing, that there is a pig in the pen, the other is not. Which of these conditions \emph{am} I in, in this case now? Suppose that being in the knowledgeable condition required me to be able to tell whether I am actually seeing a pig in a pen, or rather in some sort of mere ringer for this. Ringers being what they are, I would fail this requirement. So my condition would not be one of knowing. But, if McDowell is right, there are again, with respect to the ringers in question, two sorts of case. I might be required to be able to tell whether I am in such a ringer situation or not. But I might not. No such ringer may need ruling out by me in order for it to count as just apparent to me that there is a pig before me. Well, then, which condition am I in? And what guides towards an answer to this does McDowell make available?

Austin offers an answer to this question. It is, ``Bad question''. The question is a bad one for a quite specific reason. It supposes that there are two conditions for me to be in, \emph{in re} that pig in the pen, and that my being as I am just is my being in the one condition or the other. Whereas for Austin---knowledge being the occasion-sensitive sort of phenomenon it is---it is not \emph{my} circumstances as I stand there which decide what it is right (true) to say, but rather the circumstances in which I might be credited with, or denied, knowledge (that the pig is in the pen). To repeat, if you and I are confronting the question whether Sid is an authority as to pigs now being in the pen---if there is now some determinate such question for us to confront---then we confront the issue what would it be, now, for someone to be an authority as to that. Depending on our circumstances, this may or may not involve distinguishing porcine-infested pens from certain sorts of ringers for them. Following Austin thus gives us a principled reason for rejecting the question. But McDowell does not follow Austin, at least on this point. Oxford views of language had, in the interim, moved on. The question is, no doubt, still a bad one on McDowell’s view. But it remains an open question precisely why this is so.


\section{Perception} % (fold)
\label{sec:perception}

A concern for realism motivates a fundamental strand of Oxford reflection on perception. Begin with the realist conception of knowledge.  The question then will be: What must perception be like if we can know something about an object without the mind by seeing it? What must perception be if it can, on occasion, afford us with \emph{proof} concerning a subject matter independent of the mind? The resulting conception of perception is not unlike the conception of perception shared by Cambridge realists such as Russell and Moore. Roughly speaking, perception is conceived to be a fundamental and irreducible mode of sensory awareness of mind-independent objects, a non-propositional mode of awareness that enables those with the appropriate recognitional capacities to have propositional knowledge concerning that subject matter. 

The difference between Oxford and Cambridge realism concerns the extent of this fundamental sensory mode of awareness. Whereas Oxford realists maintained that perception affords us this non-propositional mode of awareness, Cambridge realists maintained that this distinctive mode of awareness has a broader domain. Let experience be the genus of which perception is a species. Cambridge realists maintained that \emph{all} experience, and not just perception, involved this non-propositional sensory mode of awareness. Cambridge realists are thus committed to a kind of \emph{experiential monism}---all experience, perceptual and non-perceptual alike, involves, as part of its nature, a non-propositional sensory mode of awareness. Even subject to illusion or hallucination, there is something of which one is aware. And with that, they were an application of the argument from illusion or hallucination or conflicting appearances away from sense data theory and a representative realism that tended, over time, to devolve into a form of phenomenalism.

Framing the discussion is the fundamental realist (or anti-idealist) commitment common to Cook Wilson and Moore:
\begin{quote}
	The objects of knowledge are independent of the act of knowing.
\end{quote}
Suppose that in seeing the pig Sid is in a position to know various things about it. The pig is the object of Sid's knowledge in the sense that Sid knows something about \emph{it}---that the pig is before Sid, or that the pig is black, say. This is a thesis about knowledge, not perception. What connects the fundamental realist commitment to perception is a doctrine whose slogan might be---\emph{perception is a form of knowing}. Perception, conceived as a form of knowing, is a sensory mode of awareness that makes the subject \emph{knowledgeable} of its object. In being so aware of an object, the subject is in a position to know certain things about it, depending, of course, on the subject's possession and exercise of the appropriate recognitional capacities in the circumstances of perception.

Suppose, then, that perception is a form of knowing in the sense that it makes the subject knowledgeable of its object. If knowledge is always knowledge of a mind-independent subject matter, and the objects of perception are at least potential objects of knowledge, then it follows that the objects of perception are themselves mind-independent and so independent of the act of perceiving. In this way the doctrine that perception is a form of knowing allows the realist conception of knowledge to have implications for how perception is properly conceived in light of it.

Working out the demands of the realist conception of knowledge on the nature of perception was subject to internal and external pressures. 

Internally, the core features of the realist conception of knowledge get differently conceived by different authors, in a process of refinement and extension, and so the demands that conception of knowledge places on the nature of perception are themselves reconceived. Importantly, an independent aspect of Cook Wilson's conception of knowledge, the \emph{accretion}, an aspect endorsed by Prichard and rejected by Austin, turns out to be inconsistent with the idea that perception makes the subject knowledgeable of a mind-independent subject matter. So the development of the realist conception of knowledge involved not merely refinement and extension, but elimination as well.

Externally, Oxford reflection on perception is subject to alien influences, in particular, Cantibrigian and Viennese influences. Thus Price comes to Oxford from Cambridge where he was Moore's student. Paul comes to Oxford from Cambridge as well but studied with Wittgenstein. And Ayer, given Ryle's encouragement, studied for a time with the logical positivists in Vienna. Incorporating the insights and resisting the challenges posed by these alien influences play an important part in the development of philosophy of perception in Oxford.

Cook Wilson never published on perception. The main source of Cook Wilson's \citeyear{Cook-Wilson:1926sf} views on perception is a letter of July 1904 criticizing Stout's \citeyear{Stout:1903zl} ``Primary and Secondary Qualities''. To highlight the connections between his realist conception of knowledge and his views about perception, it is useful to begin, however, with Cook Wilson's \citeyear{Cook-Wilson:1926sf} earlier letter of January 1904 to Prichard. There Cook Wilson discusses two variants of a fundamental fallacy concerning knowledge or apprehension.

The first variant is the idealist attempt to understand knowledge as an activity. If knowledge is an activity, then in knowing something a subject must \emph{do} something to the object known. But this, Cook Wilson claims, is absurd. The object of knowledge must be independent of the subject's knowing it, if coming to know is to be a discovery: 
\begin{quote}
	You can no more act upon the object by knowing it than you can `please the Dean and Chapter by stroking to dome of St. Paul's'. The man who first discovered the equable curvature meant equidistance from a point didn't supposed that he `produced' the truth---that absolutely contradicts the idea of truth---nor that he changed the nature of the circle or curvature, or of the straight line, or of anything spatial. \citep[]{Cook-Wilson:1926sf}
\end{quote}

The second variant is the representative realist's attempt to understand knowledge and apprehension in terms representation. Whereas the idealist attempts to explain apprehension in terms of apprehending, the representative realist attempts to explain apprehension in terms of the object apprehended, in the present instance, an idea or some other representation. The problem is that this merely pushes the problem back a level:
\begin{quote}
	The chief fallacy of this is not so much the impossibility of knowing such image is like the object, or that there is any object at all, but that it assumes the very thing it is intended to explain. The image itself has still to be \emph{apprehended} and the difficulty is only repeated. \citep[]{Cook-Wilson:1926sf}
\end{quote}

In what sense are the fallacies of explaining apprehension in terms of apprehending and in terms of the object of apprehension variants of the same fallacy? The are variants of the same fallacy in that both attempt to \emph{explain} knowledge or apprehension:
\begin{quote}
	Perhaps most fallacies in the theory of knowledge are reduced to the primary one of trying to \emph{explain} the nature of knowledge or apprehending. We cannot \emph{construct knowing}---the act of apprehending---out of any elements. I remember quite early in my philosophic reflection having an instinctive aversion to the very expression `\emph{theory} of knowledge'. I felt the words themselves suggested a fallacy---an utterly fallacious inquiry, though I was not anxious to proclaim <it>. \citep[]{Cook-Wilson:1926sf}
\end{quote}
This is a clear statement of the anti-hybridism or anti-conjunctivism about knowledge that McDowell and Williamson will later defend. So conceived, knowledge is not a hybrid state consisting of an internal, mental state and the satisfaction of some external conditions. Cook Wilson's aversion to the ``theory of knowledge'' is just an aversion to explaining knowledge by constructing it out of elements, and this skepticism will be echoed by Prichard, Ryle, and Austin and in precisely these terms.

Suppose that perception makes the subject knowledgeable of a mind-independent subject matter. Suppose further that the knowledge the subject is in a position to acquire cannot be explained or constructed out of elements. What must perception be like to make us knowledgeable of the environment in that sense? Must perception itself be non-conjunctive? Does Cook Wilson himself endorse anti-hybridism about perception? In his letter to Stout he does defend a conception of perception as the direct apprehension of objects spatially external to the perceiving subject. And in the letter to Prichard he does at one point speak indifferently of knowledge, apprehension, and perception. If the main conclusions of that letter are meant to apply to all three, then Cook Wilson endorses anti-hybridism about knowledge, apprehension, \emph{and perception}. Neither consideration is decisive. More telling, however, is that the variant fallacies of explaining apprehension in terms of apprehending and the object apprehended are echoed in the letter written later that year to Stout on perception and, indeed, form the core of its content. In particular, both idealist and representative realist accounts of perception are criticized in line with the two variant fallacies concerning knowledge or apprehension. Let's consider these in turn.

First, like \citet{Moore:1903uo}, Cook Wilson emphasizes the distinction between the object of perception and the act of perceiving. In perceiving an object, the object appears to the subject, and so the subjective act of perceiving is sometimes described as an \emph{appearance}. Given the distinction between the object perceived and the act of perceiving, an appearance, so understood, is necessarily distinguished from the object. However, Cook Wilson warns against a misleading ``objectification'' of appearing:
\begin{quote}
	But next the \emph{appearance}, though properly the appear\emph{ing} of the object, gets to be looked on as itself an object and the immediate object of consciousness, and being already, as we have seen, distinguished from the object and related to our subjectivity, becomes, so to say, a mere subjective `object'---`appearance' in that sense. And so, as \emph{appearance} of the object, it has now to be represented not as the object but as the phenomenon caused in our consciousness by the object. Thus for the true appearance (=appearing) to us of the \emph{object} is substituted, through the `objectification' of the appearing as appearance, the appearing to us of an \emph{appearance}, the appearing of a phenomenon caused in us by the object.  \citep[796]{Cook-Wilson:1926sf}
\end{quote}

If perceptual appearances are ``the appearing of a phenomenon caused in us by the object'', then it would be impossible for a subject to come to know about the mind-independent object on the basis of its perceptual appearance and hence impossible to discover how things stand with a mind-independent subject matter by perceiving:
\begin{quote}
	It must be observed that the result of this is that there could be no direct perception or consciousness of Reality under any circumstances or any condition of knowing or perceiving: for the whole view is developed entirely from the fact that the object is distinct from our act of knowing it or recognizing it, which distinction must exist in any kind of knowing it or perceiving it. From this error would necessarily result a mere subjective idealism. Reality would become an absolutely unknowable `Thing in Itself', and finally disappear altogether (as with Berkeley) as an hypothesis that we could not possibly justify. \citep[797]{Cook-Wilson:1926sf}
\end{quote}
This straightforwardly parallels the fallacy of explaining apprehension in terms of apprehending. 

Second, Cook Wilson singles out for criticism Stout’s representative realism, in particular his claim that the sensations which mediate knowledge of external qualities such as extension do so only in so far as ``they represent, express, or stand for something other than themselves''. The basis of of his criticism involves negative and positive claims about the nature of representation. The negative claim is that nothing is intrinsically representational:
\begin{quote}
	Nothing has \emph{meaning} in itself. \citep[]{Cook-Wilson:1926sf}
\end{quote}
The positive claim is put as follows: 
\begin{quote}
	Representation is our subjective act. ... It is \emph{we} who mean.  \citep[]{Cook-Wilson:1926sf}
\end{quote}
According to Cook Wilson, then, representation is \emph{personal}. It is \emph{we} who mean. So conceived, representation is something that the subject \emph{does}. 

How, according to Stout, might the sensation of extension ``represent, express, or stand for'' extension? Plausibly it might in two ways: by \emph{resembling} extension or by \emph{necessarily covarying} with the presence of extension. However, the natural relations of mimesis and necessary covariation are \emph{impersonal}---they obtain independently of anything that the subject does. If it is \emph{we} who mean, if representation is something that a subject does, then the natural relations of mimesis and necessary covariation could not be what makes a sensation represent and external quality. These are not two analyses of different notions of representation, mimesis and necessary covariation are merely natural relations that \emph{incline} us to represent things by means of them---they are merely relations that can be exploited by a subject's representational ends:
\begin{quote}
	It is we who make the weeping willow a symbol of sorrow. There may of course be something in the object which prompts us to give it a meaning, e.g., the resemblance of the weeping willow to a human figure bowed over in the attitude of grief. But the willow in itself can neither `mean' grief, nor `represent' nor `stand for' nor `express' grief. \emph{We} do all that.  \citep[770]{Cook-Wilson:1926sf}
\end{quote}
The weeping willow resembles a human figure bowed over in the attitude of grief. This presents a subject with an opportunity to exploit that resemblance for their own representational ends, at least if they are apprised of that resemblance. In using the willow to represent grief, the subject apprehends the content of that representation. And that, according to Cook Wilson, is precisely what prevents representation from figuring in an explanation of perceptual apprehension. Any such explanation would be circular and, hence, no explanation at all. This straightforwardly parallels the fallacy of explaining apprehension in terms of the object of apprehension, an idea or representation more generally.

Thus Cook Wilson's discussion of perception in his letter to Stout, parallels his discussion of knowledge in his letter to Prichard. In particular the two fallacies of explaining apprehension in terms of apprehending and in terms of the object apprehended (a representation) arise in the perceptual case as well. This raises the question whether in the perceptual case these fallacies are variants of the fundamental fallacy of trying to \emph{explain} perception in more fundamental terms. Just as knowledge cannot be explained in terms of belief that meets further conditions, perhaps perception cannot be explained in terms of, say, experience or appearance that meets further conditions. Cook Wilson expresses his skepticism about such explanations in the case of knowledge by denying that there is any such thing as a theory of knowledge. Farquharson in the postscript to \emph{Statement and Inference} reports a similar attitude in the perceptual case: ``He came to think of a theory of Perception as philosophically preposterous''. 

The evidence is not decisive. However, even if we were convinced that Cook Wilson accepted an anti-hybridist conception of perception, we would remain unclear why the realist conception of knowledge requires this. A reason begins to emerge with Prichard's case \emph{against} the idea that perception is a form of knowing. While Prichard opposes the doctrine that links the realist conception of knowledge with the nature of perception, his discussion reveals some of what is required if one were to retain the doctrine distinctive of twentieth century realists that perception makes us knowledgeable of a mind-independent subject matter.

Cook Wilson provides neither a theory of perception nor of the nature of appearances. However, Prichard's \citeyear[]{Prichard:1906gf,Prichard:1909yg} theory of appearing builds on some of Cook Wilson's insights. Following Cook Wilson, Prichard holds that the object of perception, like the object of knowledge, must be independent of the act of perceiving, and that an appearance is properly understood as an appearing of a mind-independent object to the perceiving subject. Prichard thus opposes any conception of appearance, such as Kant's, where appearances are states of a subject produced by external objects. However, from at least since ``Seeing Movement'' written in 1921, Prichard abandons the theory of appearing. Specifically, he comes to deny that the objects of perception are mind-independent objects located in space, coming to favor, instead, a Berkelean conception of perception where the objects of perception depend on our perceptual experience of them. At the heart of this change of mind is a doubt about whether perception could be a form of knowing.

The central argument occurs in Prichard's \citeyearpar{Prichard:1938ve} ``Sense Datum Fallacy''. His main target is the sense data theory of Cambridge realists such as Russell and Moore. Like their Oxford counterparts, the Cambridge realists held that the object of knowledge is independent of the act of knowing, and that perception is a form of knowing. Cambridge realism departs from Oxford realism in its adherence to a further thesis. Cambridge realists held, in addition, that there is something of which a subject is aware in undergoing sense experience whether perceiving or no. According to the theories of Russell, Moore, and Price, sense data are whatever we are aware of in sense experience. So understood, sense data are whatever entities that play this epistemic role. This characterization of sense data is \emph{neutral} in the sense that it assumes nothing about the substantive nature of objects that play this epistemic role. Further argument is required to establish substantive claims about the nature of sense data. Notice, that so conceived, sense data are objects whose substantive nature is open to investigation independent of our acts of awareness of them. It is this consequence of the conjunction of the realist conception of knowledge, the conception of perception as a form of knowing, and the sense data theory that is Prichard's primary target. And Prichard's central thought is that perception could not make one knowledgeable of its object, since the object of perception depends on the subject's experience of it in a way that the object of knowledge could not.

Much of Prichard's case is a variant of Berekeley's critique of Locke. However, two arguments go beyond the familiar Berkelean critique. The first derives from a peculiar feature of the Cook Wilsonian conception of knowledge, the accretion, and the second is explicitly derived from G.A. Paul. Both present important morals for Oxford realism. The moral of the first argument is that the accretion must be abandoned if Oxford realism is to be sustained. The moral of the second argument is that the realist conception of knowledge requires an anti-hybridist conception of perception (though it will take the work of Austin and Hinton to begin to vindicate this).

The first argument can seem like a variant of the argument from illusion (though it really has a very different character): 
\begin{quote}
	\ldots\ if perceiving were a kind of knowing, mistakes about what we perceive would be impossible, and yet they are constantly being made, since at any rate in the cases of seeing and feeling or touching we are almost always in a state of thinking that what we are perceiving are various bodies, although we need only to reflect to discover that in this we are mistaken. \citep[]{Prichard:1938ve}
\end{quote}
The passage is frustrating in its lack of explicitness. Indeed in the last line Prichard seems to echo Hume’s contention that it takes the slightest bit of philosophy to show that naïve realism is false. 

Suppose a pig is in plain view of Sid, and the Sid can recognize as a pig the animal that he sees. It might seem that what Sid is thus aware is incompatible with there not being a pig in before him. In which case, perception affords Sid something akin to proof of a porcine presence. In this way, perception can seem to make the subject knowledgeable of a mind-independent subject matter. Prichard's insight is that this picture is incompatible with a further feature of Cook Wilson's conception of knowledge, \emph{the accretion}. Prichard understands the accretion as follows:
\begin{quote}
	We must recognize that whenever we know something we either do, or at least can, by reflecting, directly know that we are knowing it, and that whenever we believe something, we similarly either do or can directly know that we are believing it and not knowing it. (1950: 86)
\end{quote}
If Sid knows that P, Sid can know upon reflection that the he knows that P. And if Sid has some attitude other than knowledge to that proposition, then Sid can know upon reflection that his attitude is something other than knowledge. Knowledge admits of no ringers---a state indiscriminable upon reflection from knowledge just is knowledge. What would it take for perception to make us knowledgeable of a mind-independent subject matter if there are no ringers for knowledge? If the Sid's seeing the pig makes him knowledgeable of the pig's presence, then Sid must recognize that what he is aware of in seeing the pig is incompatible with the pig's absence. But is Sid in seeing the pig in a position to recognize that? After all, there are situations indiscriminable upon reflection from seeing a pig that that do not involve the pig's presence. The Sid's hallucination of the scene would be indiscirminable upon reflection from his perceiving it. If what Sid is aware of in seeing the pig is not discriminable upon reflection from what, if anything, he is aware of in hallucinating the pig, then it could seem that he is not in a position to recognize that what is aware of in seeing the pig is incompatible with the pig's absence. He would lack proof of a pig before him. Since perception admits of ringers, it could not be a source or form of ringerless knowledge.

This argument reveals a tension within the Oxford realism of Cook Wilson and early Prichard. If Cook Wilson and early Prichard were right in claiming that the objects of knowledge are mind-independent objects, and the objects of perception are at least potential objects of knowledge, then these claims can only be sustained by abandoning the accretion. Indeed, it is telling that Austin jettison's just this feature of Cook Wilson's epistemology.

Prichard's second argument derives from \citet{Paul:1936kd}. Arguably it has ancient roots as well. At the very least, it is a variant of Berkeley's interpretation of the \emph{Theatetus} (\emph{Siris} §§ 253, 304-5). On the Berkelean interpretation, the objects of perception are in a perpetual flux of becoming. In perception, every subject is incorrigibly aware of the sensible qualities whose coming and going constitute the flux since every subject is the ``measure'' of what they perceive. Though perception affords us with an incorrigible awareness of its objects, this mode of awareness could not constitute knowledge since knowledge pertains to \emph{being}, not \emph{becoming}. More prosaically, the objects of perception could not have a continuing identity through time, if every feature they manifest is relativized to a perceiver at a time. Nor could the objects of perception be publicly accessible to different perceivers. But this would preclude the objects of perception from being objects of knowledge if knowledge is to have a mind-independent subject matter. Paul's discussion of sense data is of a piece. Paul, and Prichard following him, emphasize our inability to decide key questions about the persistence and publicity of sense data. If sense data are meant to be objects open to investigation independent of our awareness of them, then such questions should be settled by looking to the sense data themselves. But our inability to decide such questions belies this thought. At best, sense data are shadows cast by experiences that can be elicited by suitably affecting the mind. So conceived, open questions about the nature of sense data are resolved not by investigation but by linguistic distinction. In this last regard, Paul is clearly influenced by Wittgenstein's discussion of sense data in \emph{The Blue Book}:
\begin{quote}
    Queerly enough, the introduction of this new phraseology has deluded people into thinking that they had discovered new entities, new elements of the structure of the world, as though to say “I believe that there are sense data” were similar to saying “I believe that matter consists of electrons”. \citep{Wittgenstein:1958rr}
\end{quote}

Suppose the central claim here is right---that sense data do not have a substantive nature open to investigation independent of our awareness of them in sense experience. There are at least three potential morals:

\begin{itemize}
	\item One might claim that sense data constitutively depend on our awareness of them in sense experience. Sense data would be in this regard like Berkelean ideas. Sense data would lack a substantive nature independent of our awareness of them. Though, Ayer, at least, would regard this Berkelean alternative as piece of substantive metaphysics on a par with Moorean sense data. (Though neither deploy the sense-data vocabulary, Berkeley, later Prichard)
	\item One might deny that there are any substantive facts about the nature of sense data that are open to investigation. (Wittgenstein, Paul, Ayer)
	\item One might retain the conception of perception common to Oxford and Cambridge realists by abandoning the fundamental claim of the sense-datum theory---that there is an object of which we are aware whenever we undergo sense experience---and the experiential monism that came in its wake. (Austin, Hinton)
\end{itemize}

There have been relatively few takers for the Berkelean moral (though see \cite{Foster:00ny}). We will set it aside for now; instead, we will focus on the second and third morals, as represented by the work of Ayer and Austin.

Here we encounter an alien influence on the tradition of Oxford reflection on perception---in the present instance, a Viennese influence. In the \emph{Foundations of Empirical Knowledge}, Ayer takes over from Carnap and the other logical positivists the general idea that there is no substantive metaphysics and that metaphysical disagreements are better understood as practical disagreements about what language or conceptual scheme to adopt. Ayer applies this general idea to sense data and suggests that talk of sense data is just an alternative way of talking about facts that all of us can agree about, namely, facts about appearances. Ayer cites \citet{Paul:1936kd} as an antecedent in the application of this general idea to the perceptual case. However, as previously noted, the most likely proximate influence on Paul is the middle period Wittgenstein and not the logical positivists. Moreover, it is clear that Paul's attitude toward tis claim is more ironic than Ayer's:
\begin{quote}
    The important point is whatever we do is not demanded by the nature of objects which we are calling `sense-data', but that we have a choice of different notations for describing observations, the choice being determined only by the greater convenience of one notation, or our personal inclination, or by tossing a coin.
\end{quote}

Ayer understands the traditional argument from illusion to establish not that there are sense data distinct from material objects that are the objects sensory awareness, if this is to be understood as a substantive metaphysical claim; rather, the argument from illusion highlights the practical need to regiment our perceptual vocabulary. since, according to Ayer, ``see'', ``perceive'', and their cognates have readings that implicate and fail to implicate the existence of the object seen or perceived. Sense data theorists, as Ayer understands them, simply regiment in favor of the existential reading. The practical need to talk of immaterial sense data arises in the context of an epistemological project:
\begin{quote}
    For since in philosophizing about perception our main object is to analyse the relationship of our sense-experience to the propositions we put forward concerning material things, it is useful for us to have a terminology that enables us to refer to the contents of our experiences independently of of the material things they are taken to present.
\end{quote}

According to Ayer:
\begin{enumerate}
	\item (non-analytic) sentences about material objects are empirically testable but do not admit of conclusive verification while 
	\item (non-analytic) sentences about sense data are \emph{observation} sentences---they furnish evidence for other sentences and are themselves incorrigible. 
\end{enumerate}
Each of these claims are instances of more fundamental epistemological commitments that are independent of the positivist setting of Ayer's thought. Moreover, each of these claims will engage with fundamental strands of thought in Oxford epistemology and philosophy of language respectively.

The first thesis involves a commitment to a \emph{Lockean conception of knowledge}:
\begin{quote}
    I believe that, in practice, most people agree with John Locke that ``the certainty of things existing \emph{in rerum natura}, when we have the testimony of our sense for it, is not only as great as our frame can attain to, but as our condition needs.'' \citep[1]{Ayer:1958kx}
\end{quote}
(Could this be what Austin had in mind when he said that, for most philosophical writing, it is all over by the bottom of page one?) The Lockean conception of knowledge is fundamentally opposed to the Cook Wilsonian conception of knowledge as proof. According to Cook Wilson, knowing that P is akin to having a proof that P since a subject only knows that P when he is in a state that is absolutely incompatible with not-P. However, if knowledge only requires as much certainty as our frame can attain to and as our condition needs, then such certainty can fall short of proof (as Ayer acknowledges in conceding that material sentences do not admit of conclusive verification.)

The second thesis involves a commitment to \emph{a form of foundationalism} according to which there are a subclass of sentences (sentences about sense data) that can be incorrigibly known to be true. Moreover, these sentences can serve as the basis of an inferential transition to less certain sentences (sentences about material objects). However, if as Austin maintains, a sentence is only true when uttered on an occasion, there could be no sentence, independent of an occasion of utterance, that is true. And if there could be no sentence that is true independent of the occasion of utterance, then no such sentence could be incorrigibly known to be true.

% section perception (end)

% (end)

\bibliographystyle{plainnat} 
\bibliography{Philosophy}

\end{document}
