%!TEX root = /Users/markelikalderon/Documents/oxford-realism/oxford.tex
\section{Language} % (fold)
\label{sec:language}

Unlike Oxford views of knowledge, and of perception, the most significant Oxford views of language are not ones which persisted throughout the century. Rather, with their roots firmly in Cook Wilson, they flowered from the late '40s until the early '60s, largely thanks to Austin, then more or less disappeared from the Oxford scene. It would be an interesting exercise---perhaps largely in sociology---to explain why this is so. For, we shall suggest, Oxford's most distinctive views of language were borne mostly of necessity. More specifically, they were (or were seen as) what was necessary in order to keep afloat those very views of knowledge and perception which not only bear the Oxford mark, but, moreover, did persist at Oxford into this millennium. A question which is very hard to answer is how, after the early '60s, proponents of these last views thought they could get along without those insights Austin found essential. At best we will do little here towards answering it.

There were, in the last century, two distinctive Oxford views of language. One is a particular conception of the relation of language to thought. The other is, in effect, a methodological strategy. One, one might say, concerns the relation between mind and language, the other is a strategy of minding one's language. We do not normally attend to the ways our words work, but rather to what we hope to work with them. But, the idea is, in philosophy words can all too easily work to block our view of the phenomena we mean to speak of. Clarity as to their workings---how and when they actually apply---often is the best way to see through them to those objects of our study. Both these views are rooted in Cook Wilson, though in somewhat different ways. We begin here with the first.

There is a line of thought in Cook Wilson’s treatment of a notion of a proposition, and its role in logic, which adumbrates a main line in Austin’s view of language, and which may well have influenced it. Cook Wilson was, roughly, a contemporary of Frege. So it is fair to compare the two. Both wrote on logic. On first reading, Cook Wilson---precisely in his concern for the ordinary use of words---may seem to be missing all Frege’s best insights. No doubt he did miss some, though on closer reading perhaps not \emph{quite} so many as first appears. In any event, both agreed in finding a \emph{grammatical} distinction between subject and predicate---a distinction as generated by English or German syntax---of little or no relevance to logic. Frege: writes,
\begin{quote}
	Our logic books still drag in much---for example, subject and predicate---that really does not belong to logic. (1897: 60))
\end{quote}
Rejecting that distinction, he gives fundamental importance to another, that between \emph{object} and \emph{concept}. Cook Wilson writes,
\begin{quote}
	The above analysis [of a statement, or proposition] would make the distinction of subject and predicate, one not of words but of what is meant by the verbal expression. We may call this the strict logical analysis, and the distinction of the words of the sentence into `subject words' and `predicative words' may be called the grammatical analysis. \citep[124]{Cook-Wilson:1926sf}
\end{quote}
Thus, for example, in ``That building is the Bodleian'', ``that building'' is the grammatical subject; in ``Glass is elastic'', ``glass'' is the grammatical subject. But in the first either ``that building'' or ``the Bodleian'' may identify the logical subject, depending on the use being made of that sentence on an occasion. In the second, either ``glass'' or ``elastic'' may identify the logical subject on a use. \emph{Mutatis mutandis} for logical predicates. An instance of the \emph{sentence} ``Glass is elastic'', while meaning just what it does, having precisely the syntax and semantics it does, so while having the same grammatical subject and predicate, might have either of two pairs of strict (or true) logical subject and predicate. So the well-formed part, ``glass'', in the sentence, ``Glass is elastic'', might, on two different uses of that sentence to state something, make either of two different contributions to the stating of what is thus stated. Similarly for other sentences and their grammatical subjects and predicates.

Two different uses of the sentence ``Glass is elastic'', each to say something to be so, may thus form a minimal contrasting pair: in the one, but not the other, ``glass'' is the logical subject in what is said; there is a corresponding difference in logical predicates; and what what each says differs in no way not entailed by these differences. Accordingly, \citet[125]{Cook-Wilson:1926sf} tells us, each use, or what is said on it, requires a different ``logical analysis''. The first use thus says something which admits of the first, but not of the second analysis, and \emph{mutatis mutandis} for the second. Thus, each differs in what is thus said. Perhaps there is something to be said which admits of either analysis, just as for Frege a given thought admits of many different analyses. But here each member of the pair says something which requires an analysis what the other says does not admit of. So that something is not a thought analyzable in either of these two ways. Whether ``glass'' figures as a logical subject or not contributes to determining what it is that is thus said. Does each member of the pair thus express a different thought in Frege’s sense? That depends on whether the different analysis each requires---a ``logical'' analysis in Cook Wilson's sense of this term---is part of an analysis of the thought expressed. For Frege, to bear on the thought expressed is to bear on questions of truth. So the minimal difference between each member of the pair would make for a different thought expressed in each if, but only if, whether ``glass'' was the logical subject bore, somehow, on questions of truth---that is, only if that difference made for a difference in the condition on the truth of what each said, or the conditions under which this might be true. That there is such a difference is yet to be seen. We leave it for the moment as an open question.

In any case, to a Fregean, two or three things may seem to have gone wrong already. One of these lies in something Cook Wilson stresses about the just-mentioned ``logical'' distinction:
\begin{quote}
	Subject and predicate mean not the idea or conception of an object, but the object which is said to be an object of the idea or conception. But, while the things called subject and predicate are objects without anything that belongs to our apprehension of them or our mode of conceiving them, the distinction of them as subject and predicate is entirely founded on our subjective apprehension of them, or our opinion about them, and on nothing in their own nature as apart from the fact that they are apprehended or conceived. It may be said that the distinction is not in them, but in their relation to our knowledge or opinion of them, and so not a relation between what they are in themselves apart from their being sometimes apprehended. (1926: 139)
\end{quote}
\emph{Logical subject} and \emph{logical predicate} may thus seem mere psychological notions, which, for Frege, could have no bearing on logic. Whereas Frege's distinction between \emph{concept} and \emph{object} precisely is a distinction between the sorts of things we designate in expressing the thoughts we do. But Cook Wilson's notions need not be psychological in any sense in which Frege's notion of a thought is not.

A thought, for Frege, is the content of a certain sort of stance for a thinker to take towards the world. In taking such a stance a thinker would expose himself to risk of error, of a sort succumbed to or avoided merely in the world being as it is (thus an \emph{objective} stance). The thought which is the content of that stance is what fixes precisely \emph{what} risk a thinker would thus run; just \emph{when} he would succumb, just \emph{how} the world may matter to whether he has. Stances towards the world are part of a thinker's psychology, on a perfectly good use of that term. Being psychological in this sense \emph{need} not mean that it is a psychological matter what such stances there are to take, and certainly does not mean that it is a psychological matter how such and such stances relate to one another (e.g., which ones stand farther down or up on truth-preserving paths).

Cook Wilson's logical subjects and predicates need not be any more psychological than Frege’s thoughts. A thought, for Frege, identifies a commitment there is for one to make as to how things are; accordingly represents things as a certain way. For Cook Wilson, two statements, otherwise as alike as possible, but differing in whether such-and-such is their logical subject, accordingly differ in what question(s) they are to be understood to answer; and, accordingly, in what one is committed to answering correctly in making them. What commitments there are thus to make, and how the correctness of one such commitment may be related to that of others, need be no more a psychological matter than the corresponding questions for Frege’s thoughts. Of course, so reading Cook Wilson, it is a substantial thesis that there are substantively different commitments for differences in logical subject to mark; a thesis best made out, if possible, by showing that difference in logical subject, and just that, may mark commitments whose correctness is independent of one another. But just this adumbrates the really important issue to come.

In his very dismissal of the grammatical subject-predicate distinction, as well as in many other contexts, Frege insists:
\begin{quote}
	Thus we will never forget that two different sentences can express the same thought, that as to the content of a sentence, what concerns us is only what can be true or false. (1897: 60)
\end{quote}
One sentence, perhaps, can express many thoughts (each on some occasion). But what concerns Frege here is that many sentences can express one thought. As he often stresses, the same thought can be articulated, now this way, now that, so that now this, now that, appears as predicative in it. The same thought can be structured in many different ways out of many different sets of concepts and objects. Intuitively, we can see how we would, in some sense, understand ``That building is the Bodleian'' differently depending on whether it was an answer to the question what that building is, or an answer to the question which building is the Bodleian. But what we have not seen---and what, it seems, Cook Wilson has done nothing towards showing us---is that \emph{that} difference in understanding makes for different thoughts expressed---or, again, exploiting Frege’s above framework, that such a difference could make any difference to when the thought thus expressed would be true.

Frege’s object--concept distinction falls on one side of another distinction, equally fundamental for him, between sense and ``Bedeutung''. One might think of this \emph{Bedeutung}, on Cook Wilson's lines, as what we speak of, on some understanding of speaking of. But it is not the sort of object of discussion that Cook Wilson has in mind. Rather, it is, so to speak, a distillate from things at the level of sense, notably thoughts, of what matters for the sorts of calculations, or relations, of concern to logic, most notably truth-preservation. Frege begins a discussion of his main essay on the sense-reference distinction by remarking:
\begin{quote}
	The fundamental logical relation is that of an object falling under a concept; all relations between concepts reduce to this. (1892-1895: 25[[new page number]])
\end{quote}
He goes on to observe that, waiving some grammatical niceties, there is considerable justice in the view of extensionalist logicians. Having first explained how attempts to name concepts, or what they name, with expressions like, ``the concept A'', or ``What the concept A names'' generally misfire, so that, e.g., in saying ``The concept A is (identical with) the concept B'', we end up speaking of a relation between objects when we really mean to be speaking of one between concepts, he goes on to remark:
\begin{quote}
	If we keep all this in mind, we are indeed in a position to say,  ‘What two concept-words denote is the same just in case the associated extensions of the concepts coincide. And with this, I think, an important concession is made to the extensionalist logicians. (1892-1895: 31)
\end{quote}
If logic is concerned with, as Frege puts it, the laws of being true (\emph{Wahrsein}), then logic is concerned with thoughts, since, as Frege also insists, thoughts just are the things which, in the first instance, are eligible to be true or false (the things which make questions of truth arise). (See 1918: 59-60.) But the business of logic reduces, for most purposes, at least, to operations on the level of \emph{Bedeutung}. The first sentence here is all that is needed, and really all that Cook Wilson demands, to honor his insistence that logic is, in some sense, about thought. The second seems entirely consistent with his views on the role of relations between things as opposed to our manners, on occasion, of apprehending them.

So though, for several reasons, Frege is not prepared to say just what a concept is (here see his 1904), one can think of what is at the level of \emph{Bedeutung} as including such things as mappings from some range of things to others; as the taking on of such-and-such range of values for such-and-such range of arguments. (Again, we may, with Frege, keep grammatical obstacles in mind.) What corresponds to objects and concepts at the level of sense is, to use one of Frege's terms for this, modes of presentation of them: ways of thinking of something which bring some Fregean object, or concept, into play. For example, in speaking of fauns as being gambollers, I bring into play, for purposes of calculating truth preservation, among other things, a function from objects to truth-values which takes on the value true for just those objects which, as it happens, gambol. So speaking of being a gamboller is a way of presenting things which brings that concept into play; accordingly, for Frege, a way of presenting it. What there is not at the level of sense, on Frege’s conception of things, is anything corresponding to logical subjects and predicates, or more pertinently, since something would be a logical subject, or predicate, within some given proposition, or something of that form, there is, for Frege, nothing at the level of sense which has logical subjects and predicates. Certainly thoughts do not. Thoughts, for Frege, articulate into elements---being about certain objects, or was for them to be---only relative to an analysis. If we were to decompose a thought so that its elements were being about the Bodleian, and being about being in the Broad, what we would thus have would be, in effect, a mode of presentation of \emph{that thought}---a way, one among others, of thinking about it. We would have a mode of presentation of a mode of presentation of whatever it is, at the level of \emph{Bedeutung}, that thoughts present (for Frege, a truth-value). If what is to be found at the level of sense always presents something at the level of reference, there is no room for a distinction between logical subject and logical predicate at either of Frege’s levels.

Cook Wilson also has a second level corresponding, in some way, to Frege's level of \emph{Bedeutung}. It is inhabited by the things we talk about, on an ordinary understanding on which this includes, for example, the Bodleian, glass, being in the Broad, and being elastic, and by ``real relations'' between them. So it is not quite inhabited by the same things which belong to Frege's \emph{Bedeutung}. But it might be seen as inhabited by Cook Wilson’s candidates for the things which really matter to the concerns of logic---notably truth-preservation. For he insists that when we say, ``That building is the Bodleian'', no matter what the grammatical, or even logical, subject may be, what we \emph{speak of} is just that building being the Bodleian. Which, one might well think---and Cook Wilson seems sometimes to think---leaves nothing for truth to turn on but whether that building is the Bodleian. But then, why is there \emph{any} interest in the notions of (strict) logical subject and predicate, at least if one’s concern is, like Frege’s, only with that in the understandings take words to bear to which laws of logic might apply? How can whether such-and-such is the logical subject of one’s statement matter to the error one risks in stating it (or in judging what is thus stated), at least where such error is error as to how things are (or are correctly viewed as being)?

One approach to answering these questions would be as follows. Frege, while admitting that there are all sorts of aspects to the ways in which one would understand the words we in fact speak, allows into sense, in his sense, only what bears on questions of truth. That is why no notion corresponding to logical subjects and predicates shows up, for Frege, at the level of sense. The most obvious way to place those notions there would be to show that they \emph{do} bear on truth; that two truth-bearers (proposition, thoughts, statements) which differed \emph{only} in that the logical subject in one was the logical predicate in the other, and vice-versa, might, for all that, differ in when they would be true. Such would require logical subjects and predicates at Frege's level of sense. Such an idea seems to have inspired Austin. His essay, ``How To Talk (Some Simple Ways)'' (1952) is, in effect, a more refined elaboration of Cook Wilson's idea; its object (or one of them) is to show that distinctions of this kind do bear on questions of truth.

In ``How to Talk'', Austin marks two distinctions---two pairs of distinctive features---where Cook Wilson has only one. He distinguishes, first, between ``directions of fit'', and second, between what he calls ``onuses''. The first distinction is illustrated by cases like this: there is a flower, and a battery of kinds of flower it may be. Looking through the chooses, one commits to it being a \emph{dahlia}, and not, say, an iris; by contrast, one is asked, of an array of flowers, which one is the dahlia, and answers, ``This one''. In the first case, one fits the flower to a rubric (in Austin’s terms, ``cap-fitting''. In the second, one fits a rubric to the flower. Austin also calls the first thing ``placing'', and the second, casting. (In this presumably exploratory work he is neither parsimonious, nor elegant, with technical vocabulary.) The contrast in onus is made with examples like the following. There is a color sample---a piece of cloth, say. It is perfectly clear how it is colored. The question is whether being \emph{so} colored is being crimson. (``Can you really call it \emph{crimson} when there is so much blue in it?'') Or it is perfectly clear what it would be for something to be (when it would be) crimson; what is in question is whether \emph{this} sample qualifies. (``Doesn’t it have too much blue in it?'')

Austin's two contrasts yield four possible pairs of distinguishing features---of an onus and a direction---and, correspondingly, four different things to be done in saying such-and-such (that flower) to be such-and-such way or kind (a dahlia, say). Complicating his initial model slightly, these four things to be done become what he calls ``calling'', ``exemplifying'', ``describing'' and ``classing''. At which point he points to the different considerations that would come into play in holding one or another of these performances to be \emph{mistaken}, or \emph{incorrect}:
\begin{quote}
	If we are accused of \emph{wrongly} calling 1228 a polygon \ldots\ then we are accused of \emph{abusing language}. \ldots\ In calling 1228 a polygon \ldots\ we modify or stretch the use of our name \ldots\ If on the other hand we are accused of wrongly \emph{describing}, or of \emph{mis}describing, 1228 as a polygon, we are accused of doing violence to the \emph{facts}. In describing 1228 as a polygon \ldots\ we are simplifying or neglecting the specificity of the item 1228, and we are committing ourselves thereby to a certain view of it. (1952: 147-148)
\end{quote}
Different ways of going wrong, for different combinations of fit and onus, raise the possibility of going wrong in some such combination, in speaking of, say, \emph{this} flower as a \emph{dahlia}, where one would not go wrong in another combination in speaking of precisely \emph{that}. Depending on the sort of wrongness involved here, this might be the very sort of contrasting pair that Cook Wilson would need in order to bring his logical subjects and predicates into the realm of sense---aspects of the understandings we bestow on words which \emph{do} bear on questions of truth. So, for example, if it is France, or a piece of iridescent fabric with the red appearing as behind the blue, or a genetically modified ``dahlia'' which is neon orange, ten feet tall, glows in the dark, eats birds and sometimes small children, etc., then it may not be true to how the thing is---may mislead, or even misinform---to call it, respectively, a polygon, or crimson, or a dahlia. If what you are doing is saying how the \emph{thing} is, then you have chosen at the least very bad terms in which to do it. Whereas if the question is what you \emph{could} call a polygon, or crimson, or a dahlia---what being this these things really \emph{is}---then you can call France a polygon if you ignore enough irregularity, the sample crimson if you ignore the blue sheen the crimson would then be seen through, the flower a dahlia if you do not mind what dahlias might get up to, so long as the DNA is close enough. And, perhaps, there is nothing in the notions \emph{polygon}, \emph{crimson}, \emph{dahlia}, which rules out, absolutely, so viewing things. If to call France a polygon is to take a certain view of France, it being given what France is like, then that may be, at the least, a very bad view to take. Whereas if to allow that polygonal is the sort of thing France just \emph{might} be allowed to be is to take a certain view of being polygonal, that just might not be such a bad view to take of being that.

We are still some distance from making the case that would need making to install logical subjects and predicates (or Austin's more refined successors to them) within the realm of sense. One would need to make out that very bad views, such as one of that monster as a dahlia, may correspond to representing \emph{falsely}, or at least not truly. That would take some work. But we need not pursue this issue further. For lines of thought such as this one suggest a certain generalization, which can be shown on independent grounds: whether one speaks truth in saying things to be a certain way, or a thing to be a certain way (or of a certain sort) depends on the standards to which one is thus to be held; where these standards depend, not only on what the words you use speak of---just what, simply in and by using them, you are saying to be what---but also on the circumstances in which you speak (on such things as what questions you are to be held responsible for answering in \emph{so} speaking). You spoke of that flower as a dahlia, or of France as a polygon. To what standards of correctness are you thus to be held? What would be required for you to be correct in speaking of \emph{that} as a dahlia? That question is \emph{not} answered by all said so far as to what you did. This is the generalization Austin expresses in \emph{Sense and Sensibilia}, in saying,
\begin{quote}
	It seems to be fairly generally realized nowadays that if you just take a bunch of sentences (or propositions, to use the term Ayer prefers) impeccably formulated in some language or other, there can be no question of sorting them out into those that are true and those that are false; for … the question of truth and falsehood does not turn only on what a sentence is, nor yet on what it means, but on, speaking very broadly, the circumstances in which it is uttered. Sentences are not as such either true or false. (1962: 110-11)
\end{quote}
Whether one speaks truth or falsehood in saying that cloth to be crimson, or that fossil a dahlia, depends on the circumstances of one's so speaking, and on the standards for things being the way in question---the conditions on truth---that then and there apply. So one may, on one occasion, speak truly, and on another falsely, in and by saying the very same thing, in the very same condition, to be crimson, or a dahlia, or and so on ad inf. Otherwise put, there are various things being crimson, or being a dahlia, might be understood to be; where one speaks either truly or falsely in speaking of something as a dahlia, there is something this is to be understood to be, where that is just one of an indefinite variety of things this might be.

So the idea of logical subjects and predicates had, by mid-century, in Austin's hands, turned into the idea that there are many things that might be understood by something being some given way---by being a dahlia, or crimson, for example---where, on different such understandings, different ranges of things would count as those of which it was true that they were dahlias, or crimson, or whatever; that a given way (or sort of thing) for things to be, specified no matter how, does not as such pick out any unique range of things as its instances, full stop; but that what counts as instancing it on one way of viewing this is liable not so to count on others. Such is one way in which at least some of Oxford, by mid-century, had built on the foundations Cook Wilson laid in the first decade of the century (or perhaps before). But it would certainly be wrong to suggest that this view of language was ubiquitous in Oxford at mid-century. And, as noted already, it is a curious fact that, by some time in the \'70s, it had more or less died out.

The most significant dissenter, around mid-century, at least, was H.P. Grice. He first broached his counter-view in (Grice, 1961), and then, more fully, in his William James lectures of 1968(?). First we need to note one small corollary of the view just set out. Suppose that, as per that view, words (e.g., ``Fauns gambol'') \emph{underdetermine} what would be said in using them as meaning what they do (since, as per above, that might be any of indefinitely many distinguishable things). Then there is substantial work for circumstances of a speaking to do---again, as Austin insists in \emph{Sense and Sensibilia}. In those circumstances, there must be something which would be to be understood by, e.g., \emph{gambolling}; and this should be substantial enough to make what was said in that speaking truth-evaluable---gambolling, on the required understanding, must be something fauns either do or fail to. It is always possible in principle, and, Austin thinks, it sometimes occurs in practice, that circumstances are just not up to the job. So you cannot expect to say, ``Fauns gambol'' just any time you please and thereby say something either true or false. Or if, through kindness of the world, you might expect this with ``Fauns gambol'', perhaps you will have poorer luck with that strategy for a sentence like ``Sid tried to lift his pen'', or ``Pia did it of her own free will''. It is this corollary of Austin’s view on which Grice focuses.

Grice's case against Austin is centered on the thought that while, in speaking, we may say things that are either true or false, we may also suggest, or imply, or etc., other things which are either true or false. If I say, ``Pia became pregnant and married'', I may certainly at least suggest that the first-mentioned preceded the second---though (importantly for Grice) it is possible to arrange my saying this so that I would not. But the fact that I am likely at least to suggest this is compatible with my not actually having \emph{said} it, with my \emph{at most} suggesting it; and certainly compatible with there being nothing in the meaning of ``and'', or any other feature of the sentence uttered, which concerns temporal order. Grice introduces the technical term ‘implicate’ for all those ways I, or my words, may have related to propositions about temporal order other than stating them.

Now the core idea to be used against Austin is to be: where Austin sees the possibility of \emph{saying} a variety of things in given (unambiguous) words (while meaning what they do), Grice will argue that this variety of things is only implicated, while, in fact, there is some one thing (to be specified) which is what was said. Or rather, this is what Grice needs to argue. He tends, instead, as mentioned, to focus on the corollary, arguing instead that if, in certain circumstances, one would not say, e.g., ``Sid tried to lift his pen'', this may be, not because what one thus said would not be true, but rather because one would implicate something unwanted. It is not clear that Grice really understood what Austin’s point was. If not, this may be because of what proved to be an unfortunate choice of vocabulary by Austin and Austinians. We will come to that issue shortly. In any case, the idea of implicature is arguable ill-suited for the application it would need for it to touch Austin's view. The idea to be countered is: a sentence, say, ``That painting is crimson'', may be used of a given painting, in a given condition, to say different things, some true, some false, where there is no limit, in principle, to the new things new occasions may make available thus to say. The counter would be: these different things are merely implicated. But then, what is implicated, on any such occasion is, on some possible understanding of being crimson, that the painting is crimson. Now what, in addition to \emph{that}, is to be the thing which is \emph{said} throughout all those cases? Surely something to the effect that the painting is crimson. So it is ``crimson'', whatever that comes to (on some understanding so being), and, moreover, for a given occasion, it is what being crimson is to be understood to be on that occasion. But what is this additional thing which being crimson always comes to throughout? And how is \emph{that} compatible with the different things it would be taken to come to on different occasions?

We just mentioned an (as it proved) unfortunate choice of vocabulary---one of which Grice certainly makes much. This choice is most evident in the second methodological point, begun by Cook Wilson, developed by Austin. The idea was: in philosophy, we need to mind our language. This idea is put most clearly and elegantly by Austin:
\begin{quote}
	First, words are our tools, and as a minimum, we should use clean tools: we should know what we mean and what we do not. and we must forearm ourselves against the traps that language sets us. Secondly \ldots\ we need to prise them off the world \ldots\ so that we can realize their inadequacies and arbitrariness, and can re-look at the world without blinkers. Thirdly \ldots\ our common stock of words embodies all the distinctions men have found worth drawing, and the connexions they have found worth making, in the lifetimes of many generations: these surely are likely to be more numerous, more sound, since they have stood up to the long test of the survival of the fittest, and more subtle, at least in all ordinary and reasonably practical matters, than any that you or I are likely to think up in our armchairs of an afternoon \ldots
	\ldots\ When we examine what we should say when, what words we should use in what situations, we are looking again not \emph{merely} at words \ldots\ but also at the realities we use the words to talk about: we are using a sharpened awareness of words to sharpen our perception of, though not as the final arbiter of, the phenomena. (1956-57: 182)
\end{quote}
We should, we are told, mind our language for several reasons. For one thing, philosophical problems often depend on taking some word in its usual (ordinary, English) sense. Has anyone ever seen a tomato? If that is not in question when it is asked whether what we see are things in our environment, or if it is in question only in some technical sense of ``see'' (to be specified), then the question is not obviously as interesting as it initially seems to be, and much more work needs to be done to show it to be interesting at all. That one does not see ``material objects'' was (we thought) meant to be an amazing discovery. Conversely, for another, if philosophers are not to fly off into the empyrean, only to lose their way there, then they need to be held accountable for what they say. Causal relations hold only between mere appearances. Oh, really? So you did not just now fill my glass. Oh, you didn't mean that by ``appearances''? Well, then, what \emph{did} you mean? (This is all too likely to prove to be nothing at all.) Finally, philosophers too often find introducing technical vocabulary, so as for it to make \emph{sense}, an all too easy matter. Seeing the complexities of ordinary vocabulary, and of the task of getting \emph{it} to apply to the world may be sobering. Moreover, it may show us how our \emph{thinking} falls into confusion by failure to note the complexities involved in isolating a phenomenon.

Austin's advice should, perhaps, have been old saws, but in fact reconceives philosophical good faith, changes what a philosopher could say with a straight face from what this would be taken to be by Hume, or Bradley, or the subject at large in the 18th and 19th centuries, and in some quarters (cf., e.g., Sartre) in the 20th. But here the vexatious vocabulary intrudes. Austin speaks of what we should (would) say when. A natural way of speaking if you want to respect the idea that it is intrinsic to words to equip us to say, or do, different things with them in different circumstances, on different occasions for the doing. But ``what we would say'' can be read so as to encompass such things as not saying, ``What’s the vigorish?'' when your neighbor asks to borrow a cup of milk (but perhaps saying this if it concerns a cup of Scotch), or not saying ``That's just autobiography'' to your small niece when she says she wants another biscuit. And this is how Grice is inclined to read it. On the other hand, asking what one would say when \emph{can} be a way of asking what the words one uses in fact apply to, or for doing what they are in fact applicable---what one would say (as what one would describe a thing, for what one would ask, what sort of greeting or condolence one would convey) in using them (for what they \emph{are} for in the language). If one is moved primarily by that main view of language, as developed by Austin from Cook Wilson's seminal idea---that it is not, e.g., English words, but rather their use on an occasion, which determines how they may, or must, be articulated in understanding what they said—then one certainly will read those words ``what we should say when'' in this last way.

% section language (end)
