%!TEX root = /Users/markelikalderon/Documents/oxford-realism/oxford.tex
\section{Introduction} % (fold)
\label{sec:introduction}

This is a story of roughly a century of Oxford philosophy told by two outsiders. Neither of us has ever either studied or taught there. Nor are we specially privy to some oral tradition. Our story is based on texts. It is, moreover, a very brief, and very highly selective, story.  We mean to trace the unfolding, across roughly the last century, of one particular line of thought---a sort of anti-idealism, and also a sort of anti-empiricism. By focussing in this way we will, inevitably, omit, or give short shrift to, more than one more than worthwhile Oxford philosopher. We will mention a few counter-currents to the main flow of 20th century Oxford thought. But much must be omitted entirely.

Our story begins with a turn away from idealism. Frege's case against idealism, so far as it exists in print, was made, for the most part, between \citeyear{Frege:1893fv} (in the preface to \emph{Grundgesetze} volume 1) and \citeyear{Frege:1918lq} (in ``Der Gedanke''). Within that same time span, at Oxford, John Cook Wilson, and his student, H.A. Prichard, developed, independently, their own case against idealism (and for what might plausibly be called---and they themselves regarded as---a form of ``realism''). Because of the way in which Cook Wilson left a written legacy it is difficult at best to give exact dates for the various components of this view. But the main ideas were probably in place by 1904, certainly before \citeyear{Prichard:1909yg}, which marked the publication of Prichard’s beautiful study, ``Kant’s Theory of Knowledge''. It is also quite probably seriously misleading to suggest that either Cook Wilson or Prichard produced a uniform corpus from the whole of their career---uniform either in content or in quality. (For Cook Wilson the issue is synchronic, while for Prichard it is diachronic, and, accordingly, somewhat puzzling.) But if we select the brightest spots, we find a view which overlaps with Frege’s at most key points, and which continued to be unfolded in the main lines of thought at Oxford for the rest of the century.

Frege's main brief against idealism (of the sort which was common currency in Frege's time) could be put this way: it placed the scope of experience (or awareness) outside of the scope of judgement. In doing that, it left us nothing to judge about. A central question about perception is: How can it make the world bear on what one is to think---how can it give me what are then my reasons for thinking things one way or another? The idealist answer to that, Frege showed, would have to be, ``It cannot''. What, in Frege's terms, ``belongs to the contents of my consciousness''---what, for its presence needs someone to be aware of it, where, further, that someone must be me---cannot, just in being as it is, be what might be held, truly, to be thus and so. (This is one point Prichard retained throughout his career, and which, late on, he directed against others who he termed ``sense-datum theorists''. It is also a point Cook Wilson directed, around 1904, against Stout. (See below.)) So, in particular, it was crucial to Frege that a thought could not be an idea (``Vorstellung''), in the sense of ``idea'' in which to be one is to belong to someone’s consciousness. The positive sides of these coins are: all there is for us to judge about---all there is which, in being as it is might be a way we could judge it to be---is that environment we all jointly inhabit; to be a thought is, intrinsically, to be sharable and communicable. All these are central points in Cook Wilson's, and Prichard's, Oxford realism. So, as they both held (early in the century), perception must afford awareness of, and relate us to, objects in our cohabited environment.

There is another point which Prichard, at least, shared with Frege. As Prichard put it:
\begin{quote}
	There seems to be no way of distinguishing perception and conception as the apprehension of different realities except as the apprehension of the individual and the universal respectively. Distinguished in this way, the faculty of perception is that in virtue of which we apprehend the individual, and the faculty of conception is that power of reflection in virtue of which a universal is made the explicit object of thought. \citep[44]{Prichard:1909yg}
\end{quote}
Compare Frege:
\begin{quote}
	\noindent A thought always contains something which reaches out beyond the particular case, by means of which it presents this to consciousness as falling under some given generality. (1882: Kernsatz 4)
	
	\noindent But don’t we see that the sun has set? And don’t we also thereby see that this is true? That the sun has set is no object which emits rays which arrive in our eyes, is no visible thing like the sun itself. That the sun has set is recognised as true on the basis of sensory input. (1918: 64)
\end{quote}
For the sun to have set is a way for things to be; that it has set is the way things are according to a certain thought. A way for things to be is a generality, instanced by things being as they are (where the sun has just set). Recognising its instancing is recognising the truth of a certain thought; an exercise of a faculty of thought. By contrast, what instances a way for things to be, what makes for that thought's truth, does not itself have that generality Frege points to in a thought---any more than, on a different level, which Frege calls ``Bedeutung'', what falls under a (first-level) concept might be the sort of thing things fall under. What perception affords is awareness of the sort of thing that instances a way for things to be. Perception's role is thus, for Frege, as for Prichard, to bring the particular, or individual, in view---so as, in a favourable case, to make recognisable its instancing (some of) the ways for things to be it does. The distinction Prichard points to here is as fundamental both to him and to Frege as is, for Frege, the distinction between objects and concepts.

For all this shared ground between Prichard, Cook Wilson, and Frege, there is still a difference in focus. For Frege, the central notion in his critique of idealism is \emph{truth}, or, correlatively, judging (a truth-evaluable stance towards things). The trouble with idealism, for him, is that it leaves no room for judgement. For Cook Wilson and Prichard, the central notion was \emph{knowledge}. The trouble with idealism (all idealism being, Prichard argued, subjective idealism) is that it leaves no room for knowledge. (It is just restating Frege's core point about ideas to say that ideas, or, in Prichard’s terms, appearances, are not things about which one can be knowledgeable: there is nothing to know about them.) And it is with this focus on knowledge that Cook Wilson’s and Prichard’s brief against idealism continued to shape Oxford philosophy throughout the last century.

Cook Wilson’s and Prichard’s rejection of idealism assumed its finished form in the first decade of the last century. It coincided roughly with several others. Frege's, notably, was in full flower in 1893, again in 1897, and then in his masterful case against idealism in \citeyear{Frege:1918lq}. At Cambridge, Moore’s and Russell's revolution began in \citeyear{Moore:1899sd} with Moore’s ``The Nature of Judgement'', and continued with his ``The Refutation of Idealism'' of \citeyear{Moore:1903uo}, and with various papers by Russell (see notably ``The Nature of Truth'' \citeyear{Russell:1906sm}). Russell's focus, as he himself points out, was a bit different from either Moore's or Cook Wilson's and Prichard's. As \citet[42]{Russell:1959fv} puts it, ``I think that Moore was most concerned with the rejection of idealism, while I was most interested in the rejection of monism.'' Specifically, Russell spent a good deal of time campaigning against a ``doctrine of internal relations'', held by Bradley and others. But, as \citet[42]{Russell:1959fv} also said, both he and Moore were concerned to insist on ``the doctrine that fact is in general independent of experience''. Moore's points coincided with Cook Wilson and Prichard at a number of crucial points. He insisted, for example, 
\begin{quote}
	[T]he existence of a table in space is related to my experience of it in precisely the same way as the existence of my own experience is related to my experience of that. \ldots\ if we are aware that the one exists, we are aware in precisely the same sense that the other exists; and if it is true that my experience can exist, even when I do not happen to be aware of its existence, we have exactly the same reason for supposing that the table can do so also. \ldots\ I am as directly aware of the existence of material things in space as of my own sensations; and what I am aware of with regard to each is exactly the same---namely that in one case the material thing, and in the other case my sensation does really exist. (1903: 453) (ref. of ideal. Mind NS v 12 n 48 (Oct 1903) 433-453)
\end{quote}
Though, for all that, one might reasonably find Cook Wilson and Prichard more relentlessly focussed on the structure of perceptual experience and of knowledge.

Russell reports finding it exhilarating to reject idealism:
\begin{quote}
	I felt it, in fact, as a great liberation, as if I had escaped from a hothouse on to a wind-swept headland. I hated the stuffiness involved in supposing that space and time were only in my mind. I liked the starry heavens even better than the moral law, and could not bear Kant’s view that the one I liked best was only a subjective figment. In the first exuberance of liberation, I became a naïve realist and rejoiced in the thought that grass is really green, in spite of the adverse opinions of all philosophers from Locke onwards. I have not been able to retain this pleasing faith in its pristine vigour, but I have never again shut myself up in a subjective prison. \citep[48]{Russell:1959fv}
\end{quote}
This last sentence is half-right. Neither Russell, nor Moore, nor Prichard (by the 1930s) was able to hang onto the anti-idealist insights with which they began. (If Cook Wilson did, then again, he died in 1915.) Indeed, by 1917, when he delivered the lectures \emph{The Philosophy of Logical Atomism}, Russell had again locked himself up in a thoroughly subjective prison, even insisting that, pace Frege, it was a positively good thing that thoughts could never be exactly communicated. If idealism is a doctrine (or set of them) about the cognitive role of ideas, in Frege’s sense of idea (\emph{Vorstellung})---something coeval with awareness of it, and which it took being so-and-so to be aware of---then nothing could be a more idealist view of the relation between thought and its objects, and of the objects of experience, than Russell’s logical atomism of around that year. By the ‘20s, Moore was himself drawn, reluctantly, into sense-datum theory. As for Prichard, though he remained always opposed to what he called ``sense data'', he did come, some time before \citeyear{Prichard:1938ve}, to believe that the objects of sight were things he called ``colours'', which, whatever else they were, were precisely ideas in Frege’s sense. We think there is a systematic reason why philosophers as insightful as these were uniformly unable to hold onto the realism with which, with the century, they began. It is, in brief, that (like Kant, as per the 4th paralogism) they did not have the tools really to resist a form of the argument from illusion. Those tools came only later, with Austin. We will elaborate this point in due course.

One more initial point. In addition to the realism just sketched, Cook Wilson also contributed to Oxford philosophy a new conception of philosophical good faith (certainly new relative to Hume, to Hegel, and to most of the post-Cartesian tradition). It is a conception perhaps better known as later championed by Moore. Cook Wilson expressed it thus:
\begin{quotation}
	\noindent The actual fact is that a philosophical distinction is prima facie more likely to be wrong than what is called a popular distinction, because it is based on a philosophic theory which may be wrong in its ultimate principles. \ldots\ There is a tendency to regard the linguistic distinction as the less trustworthy because it is popular and not due to reflective thought. The truth is the other way. Reflective thought tends to be too abstract, while the experience which has developed the popular distinctions recorded in language is always in contact with the particular facts.
	
	Now it is not uncommon in philosophic criticism that some popular term, when reflected on, presents great difficulties to the philosopher; difficulties which are often due to some false theory of his which is presupposed. The criticism sometimes ends \ldots\ so that \ldots\ any distinctive use of [the term] is supposed to be an illusion, or the meaning of the term may be pronounced to be altogether an illusion. When the philosopher arrives at such a conclusion it too often happens that he is satisfied with this negative result. \ldots\ We ought under such circumstances to inquire how it is, if the given term only means something else, that language ever developed it, and still so obstinately holds to it, and when we believe that we have explained a term away or shown that it is a mere unnecessary way of disguising some other meaning, we ought to put our result to the test by trying to do without the word criticized and seeing what would happen if we everywhere substituted for it what we suppose to be the truer expression. \citep[875]{Cook-Wilson:1926sf}
\end{quotation}
A philosopher's claims must be answerable to something. If they are, say, claims about seeing, there is nothing better to which they may be answerable than the way the verb ``see'' is actually used. This is one way of putting the foundations of what came to be known as ordinary language philosophy---some decades before there was any. This, though is a point about philosophic methodology. It does not yet identify the main focus of 20th century Oxonian interest in language. 

Despite that salient difference in focus between Frege, on the one hand, and Prichard and Cook Wilson on the other---despite the centrality of knowledge in 20th century Oxford’s concerns---we divide the following discussion into three sections in this order: language, knowledge, and perception.

\nocite{Russell:1985lk}

% section introduction (end)
